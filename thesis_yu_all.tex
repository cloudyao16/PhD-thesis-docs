\documentclass[edeposit,fullpage]{uiucthesis2009}

% Import all packages in Copernicus.
\usepackage[normalem]{ulem}
\usepackage[T1]{fontenc}
\usepackage{textcomp}
\usepackage[utf8]{inputenc}
\usepackage[english]{babel}
\usepackage{array}
\usepackage{tabularx}
\usepackage{graphicx}
\usepackage{overpic}
\usepackage{color}
\usepackage{amssymb}
\usepackage[intlimits,fleqn,tbtags]{amsmath}
\usepackage{amsthm}
\usepackage{url}\urlstyle{same}
\usepackage{accents}
\usepackage{cancel}
\usepackage{multirow}
\usepackage{supertabular}
\usepackage{algorithmic}
\usepackage{float}
\usepackage{algorithm}
\usepackage{caption}
%\usepackage{subfig}
%\usepackage{subfloat}
\usepackage[authoryear,round]{natbib}
\usepackage{rotating}
\usepackage[mathlines,modulo]{lineno}
\usepackage{times}
\usepackage{tikz}
\usepackage{subcaption}  %package for acp paper
\usepackage[version=4]{mhchem} % package for jgr paper
%\usepackage{chemformula}
\usepackage{threeparttable} %% package for jgr paper
\usepackage{booktabs}
\usepackage{soul}
\usepackage{lscape}
\usepackage{threeparttablex}
\usepackage{longtable}
\usetikzlibrary{shapes,arrows,chains}

%
\renewcommand{\topfraction}{0.9}	% max fraction of floats at top
\renewcommand{\bottomfraction}{0.8}	% max fraction of floats at bottom

% TODO commands
\newcommand{\jctodo}[1]{{\color{red} #1}}
\newcommand{\jcedits}[1]{{\color{blue} #1}}

% Graphics path
\graphicspath{{./graphics/}}

\pdfinfo{
   /Author (Yu Yao)
   /Title  (Particle-resolved aerosol modeling on the
regional scale -- Insights into importance of capturing
aerosol mixing state)
%   /CreationDate (D:20040502195600)
}

% Custom settings
\renewcommand\thealgorithm{\thechapter.\arabic{algorithm}} 

% Tables
\newcommand\tophline{\hline\noalign{\vspace{1mm}}}
\newcommand\middlehline{\noalign{\vspace{1mm}}\hline\noalign{\vspace{1mm}}}
\newcommand\bottomhline{\noalign{\vspace{1mm}}\hline}
\newcommand\hhline{\noalign{\vspace{1mm}}\hline\noalign{\vspace{1mm}}}

% Need the GMD unit command
\DeclareRobustCommand*\unit[1]
 {\ensuremath{%
   {\thinmuskip3mu\relax
    \def\mu{\text{\textmu}}\def~{\,}%
    \ifx\f@series\testbx\mathbf{#1}\else\mathrm{#1}\fi}}}

    
\newcommand{\kMax}{K_{\textnormal{up}}}
\newcommand{\kMin}{K_{\textnormal{min}}}
\newcommand{\kOver}{K_{\textnormal{over}}}
\newtheorem{theorem}{Theorem}[chapter]
\newtheorem{corollary}[theorem]{Corollary}

% Adjust the length of the table of contents
% List everything (subsubsections)
%\setcounter{tocdepth}{3}
% Chapters and sections only
\setcounter{tocdepth}{1}
\begin{document}

%%%% Title creation
%\nocopyrightpage
\title{Quantifying cloud chemical processes and aerosol optical properties using a particle--resolved aerosol model}
\author{Yu Yao}
\department{Atmospheric Sciences}
\phdthesis
\committee{ Professor Nicole Riemer, Chair and Director of Research\\
Professor Sonia Lasher-Trapp \\
Associate Professor Matthew West \\
Dr. Matt Dawson\\}
\maketitle


% Begin front matter
\frontmatter

\begin{abstract}

\end{abstract}

\begin{dedication}
To my family
\end{dedication}

\chapter*{Acknowledgments}


% List of acknowledgements:
%
% Advisors
% Department support, specifically committee.
% Riemer group, specifically Joseph Ching, Laura Fierce.
% CSE fellowship
% DOE ASR grant
% Blue Waters standard acknowledgement

%%%%%%
% Next comes ToC, LoT and LoF
%%%%%%

\tableofcontents
%\listoftables
%\listoffigures

%%%%%%
% Begin the main body
%%%%%%
\mainmatter


%%%%%%%%%%%%%%%%%%%%%%%%%%%%%
%%%%%Chapter 1%%%%%%%%%%%%%%%
%%%%%%%%%%%%%%%%%%%%%%%%%%%%%
\chapter{Introduction and motivation}
\label{chapter1}
Aerosol particles are solid or liquid matter suspended in the air, and it receives tremendous focus recently because of the ongoing COVID-19 pandemic caused by SARS-Cov-2 virus, which is airborne and dominant by aerosol transmission \citep{prather2020reducing,zhang2020identifying, miller2021transmission, greenhalgh2021ten}. Actually, as a part of air pollutants, the health effects of aerosol particles, especially fine particulate matter with diameter less than 2.5~$\rm \mu m$ ($\rm PM_{2.5}$), have been investigated tremendously \citep{bell2007spatial, fann2012estimating}. Around 141\,000 premature death in North America due to cardiopulmonary and lung cancer is associated with $\rm PM_{2.5}$ \citep{anenberg2010estimate}. Besides its crucial impacts on health, aerosol particles can also change weather and climate through interacting with solar radiation and clouds, and there still exists large uncertainties to quantify these impacts \citep{IPCC_CHAPTER7, seinfeld2016improving, fan2016review, bellouin2020bounding}. 

The interaction between aerosols and climate can be demonstrated in two ways. On the one hand, aerosol particles can directly alter the earth's radiative balance by scattering and absorbing incoming solar radiation, known as aerosol-radiation interactions (ari). On the other hand, aerosol indirectly affect climate by influencing cloud properties, such as droplets number concentration and cloud life, and this effect is known as aerosol-cloud interactions (aci). Unlike the greenhouse gases, particles are typically with short lifetime and highly varied in space and time, making the interactions remain the largest uncertainties for future climate prediction. This dissertation contributes to understand aerosol climate effects by quantifying cloud chemical processes and aerosol optical properties. This chapter gives a general background about aerosol properties, aerosol climate forcing, aerosol mixing state, mixing state evolution, model representation for aerosol and aqueous processes.  

\section{Aerosol properties}
\label{cha1-1:aerosol-defi}
%Definition of aerosol particles.
Aerosol particles can be directly emitted from different sources, or formed secondary from gas precursors. The major components of particles are inorganic species, carbonaceous species, sea salt and mineral dust. These species can originate from natural or anthropogenic sources, and the formation pathways are also varied. Sea salt particles are generated primarily over ocean region through mechanical processes and strongly depend on wind speed \citep{jaegle2011global, monahan1986model}. Dust is another primary natural species originated from desert regions and can affect remotely through long-range transport \citep{van2018mysterious,yu2021observation}. The dominant pathway of inorganic species are secondary through either nucleation or gas-to-particle partitioning. For example, sulfate can form through nucleation of sulfuric acid with the presence of water vapor \citep{sipila2010role}, and it can also be produced through oxidation of $\rm SO_2$ by OH in gas phase or $\rm H_2O_2$ and $\rm O_3$ in aqueous phase \citep{shao2019heterogeneous, zheng2020multiphase}. Primary and secondary are both prevalent pathways for organics formation. Black carbon (BC) and primary organic aerosol (POA) are usually co-emitted from combustion of fossil fuel and biofuel \citep{bond2007historical}. Formation of secondary organic aerosol (SOA) involves partitioning with semivolatile species, oligomerization and aqueous chemistry \citep{zhu2017mechanism, lim2010aqueous, griffin2013sources, mcneill2015aqueous}. Observed particles are commonly the mixture of different species. 
Figure~\ref{fig:chap1-mixing} shows the transmission electron microscopy (TEM) images of carbonaceous-bearing particles collected in urban Shanghai in 2010, and we can find some particles are mixture of two species (a, b, c), others are mixture of more than three species (f, g, h, i).  

\begin{figure}
	\centering
	\includegraphics[scale=0.40]{chap1_figs/thesis_chap1_fig2.png}
	\caption{TEM images of C-bearing particles collected from urban Shanghai, adapted from \citet{fu2012morphology}.}
	\label{fig:chap1-mixing}
\end{figure}

%Aerosol particles diameters range from several nanometers to over 10 $\rm \mu m$.  

Aerosol diameters vary from molecular aggregates with several nanometers to dust particles with several micrometers \citep{MCMURRY200320}. It is common to represent aerosol size distribution by three modes: Nuclei mode (0.01 -- 0.1 $\rm \mu m$), accumulation mode (0.1 -- 2 $\rm \mu m$) and coarse mode (2 -- 10 $ \rm \mu m$). Nuclei-mode particles are commonly observed near combustion sources, especially roadside atmosphere, and characterized by high number concentration, which can engender efficient coagulation and result in short lifetime for these particles \citep{fushimi2008atmospheric}. Accumulation-mode particles contain most of the secondary species, such as sulfate, nitrate and organics \citep{zhang2005time}, and they can also be produced from coagulation of the nuclei-mode particles. As for coarse-mode particles, it is hard to grow such large by coagulation alone between small particles \citep{friedlander1991scavenging, lee2005size}. They are mostly originated from natural sources and produced by mechanical forces. They suspended shortly in the air due to rapid gravitational settling. 

\section{Aerosol climate forcing}
\label{cha1-2:aerosl-climate}
Aerosol can change radiation balance of Earth system through direct interactions with shortwave and longwave radiation. Globally, average net effects change of direct aerosol effective radiative forcing between 1750 and 2005 is estimated to be $-0.45$ $\rm W$ $\rm m^{-2}$, as shown in Fig.~\ref{fig:chap1-aerosol-climate}. Estimation of the effects relies on the properties of aerosol population, such as size and chemical components, and the wide variety of aerosol particles at different regions leads to spatial variation of this effect. For example, directive radiative forcing over top of atmosphere over Africa is around $-12$ $\rm W$ $\rm m^{-2}$, while it can reach +$30$ $\rm W$ $\rm m^{-2}$ over polluted Indo-Gangetic Plains \citep{subba2020recent}. 

\begin{figure}
	\centering
	\includegraphics[scale=0.80]{chap1_figs/thesis_chap1_fig1.jpeg}
	\caption{Radiative forcing estimation of different agents between 1750 and 2011, adapted from \citet{IPCC_CHAPTER8}.}
	\label{fig:chap1-aerosol-climate}
\end{figure}

As a condensation nuclei, aerosol particles can also indirectly affect climate through interactions with cloud, and the estimated effective radiative forcing from this interaction is $-0.45$~\unit{W\;m^{-2}}(Fig.~\ref{fig:chap1-aerosol-climate}). There are many processes involved in the interactions and they can be illustrated by two main effects: cloud albedo and lifetime effects. Cloud albedo effects, first proposed by \citet{twomey1977influence}, describe the changes of cloud number concentration and surface area at environment with different aerosol loading, and it had been supported by ambient observations. \citet{kaufman2005effect}, by analyzing satellite observation over Atlantic ocean, found liquid clouds coverage in the polluted environment is higher by 0.2$-$0.4 than clean environment, and with smaller droplets. Field observations also confirme that more cloud condensation nuclei (CCN) generate more cloud droplets with smaller sizes in liquid clouds \citep{jia2019distinct,kleinman2012aerosol}. Cloud lifetime effect is the mechanism that the increase of aerosol number concentration will result in smaller cloud droplets and inhibiting precipitation development. Considering smaller droplets are hard to trigger coagulation coalescence, this effect is theoretically reasonable if only considering precipitation initiation. However, several observations and model studies suggested this effect can be problematic under different cloud environment due to the unclear relation between initial cloud droplet size and evolving precipitation efficiency \citep{stevens2009untangling}. 
% May do not need to mention this because we are not going to talk about it. 
%The physical understanding of interactions with liquid cloud improves substantially in past decades, but the interactions with ice and mix-phased clouds is still poor constrained. 

As we can see from Fig.~\ref{fig:chap1-aerosol-climate}, there still exists large discrepancies in the aerosol related radiative forcing amplitudes among different models. For radiative forcing from aerosol radiation interaction, the estimated 5 to 95\% confidence ranges from $- 0.95$ to $+ 0.05$ $\rm W$ $\rm m^{-2}$, while the range is from $-1.2$ to $0.0$ $\rm W$ $\rm m^{-2}$ for aerosol-cloud interaction. The sources of the uncertainty lie in the fact that there are multiple scale processes are involved, from particles as small as 10~nm to 1000~km stratocumulus clouds. Many processes are still unclear, such as aerosol interaction with mix-phase and ice clouds, simple parameterization schemes are applied to describe these processes in the models. Even for those processes that are well-constrained, it is challenging to incorporate all these processes in a single model \citep{seinfeld2016improving,bellouin2020bounding}. By interacting with electromagnetic radiation, and acting as an nuclei for cloud formation, particles are fundamental for quantifying these forcing and accurate description of aerosols can help reduce the model uncertainly. 

\section{Aerosol mixing state}
Mixing state, which defines the distribution of chemical species among the particle population, is a helpful concept to describe aerosols \citep{winkler1973growth}. There are two mixing state extremes for an aerosol population: internal mixture and external mixture. As Fig.~\ref{fig:chap1-chi-climate}(a) shows, a population is considered to be completed internally mixed if each particle is made up of the same species mixtures. For a population with fully external mixture, each particle only contain one single specie. However, in real environment, particles rarely fall into these two categories and they are mostly in the intermediate state. In other words, the number of species and their percentages can differ between particles.

Hygroscopicity and optical properties, which are important factors for droplets formation and radiative behaviour, are varied for different species. Thus, aerosol population with different mixing state will produce varied climate-related properties, as illustrated in Fig.~\ref{fig:chap1-chi-climate}. Figure~\ref{fig:chap1-chi-climate}(a) depicts the effects of mixing state for aerosol activation potential. All three populations are with the same amount of ammonium sulfate and organic, but the species are distributed differently among the particles. For the internally mixed population, all the particles have the same amount of ammonium sulfate and organic. For the population with external mixture, each particle only contains one single species. In real environment, these two species can randomly distribute among the particles. If these three populations are exposed at the same supersaturation environment (ss = 0.3\%), the number of activated particles is varied. All the particles are activated in the population with internal mixture, and only half are activated in the externally mixed population. The activated particles number in real world case are between these two extremes. 

\begin{figure}
	\centering
	\includegraphics[scale=0.60]{chap1_figs/thesis_chap1_fig3.pdf}
	\caption{Aerosol mixing state effects on (a) activation ability and (b) optical values. (a) is adapted from \citet{Riemer2019}.}
	\label{fig:chap1-chi-climate}
\end{figure}

The effects of mixing state on aerosol activation potential can be investigated through closure study. Aerosol/CCN closure study is conducted by comparing the observed and predicted CCN concentration at the same supersaturation level. The prediction is made by applying K$\ddot{\rm o}$hler theory, using the measured dry particle size distributions and components information as input, with assumptions in mixing state to examine its effects.  \citet{broekhuizen2006closure} performed the closure study for aerosol samples from downtown Toronto at 0.58\% supersaturation, and they found the internal mixture assumptions overpredicted the CCN concentrations by 0.12$\pm$0.05. Using the CCN sampled from 2010 CalNex field campaign, \citet{moore2012hygroscopicity} also found internal mixture resulted in 30--75\% overprediction of CCN concentration. 

As for the effects on aerosol optical properties, we can apply the same strategy as Fig~\ref{fig:chap1-chi-climate}(a) to explain. Figure~\ref{fig:chap1-chi-climate}(b) shows the populations with the same amount of absorbing species (BC) and non-absorbing species ($\rm NH_4HSO_4$), but with different mixing state. As a result, the optical properties, including single scattering albedo (SSA), ensemble scattering coefficients ($\beta_{\rm abs}$) and ensemble absorbing coefficients ($\beta_{\rm abs}$), are varied in the populations and can lead to different radiative forcing effects.

The effects of aerosol mixing state on its optical properties can be more complicated if considering the aerosol water absorbing ability and particle shapes. In a humidified environment, water update of a particle depends strongly on its composition because of varied hygroscopicity of the species, which is important for scattering \citep{MichelFlores2012, Zieger2013, Titos2014, Titos2016}. Studies showed, compared with dry environment, scattering ability can be enhanced by 1.6 at the environment with RH of 85\% \citep{Burgos2020}. As for particle shapes, the distribution of the diverse species in a particle is important in determining optical values. For particles with no strong absorber, i.e. BC, volume-mixing rule can be used to calculate the refractive index and when BC is contained in the particle, core-shell configuration is proved to be more accurate \citep{Bond2006}. The absorption enhancement of BC-contained particles due to its surrounding coatings are widely investigated \citep{Moffet2009,Liu2017, wu2020light}, and distribution of the non-absorbing species are found to be the main sources for the discrepancies between the simulated and observed optical values \citep{Fierce2016, Fierce2020}.

\section{Aerosol mixing state evolution}  
Aerosol mixing evolution is a complex procedure and several processes are involved. At the time of particles emitted from the sources, aerosol population can already be the mixture of different species. For example, BC emitted from diesel vehicles are mixed with organics and sulfate, and the sea-spray aerosols are the mixture of sodium chloride and organics \citep{cheung2010emissions, kirpes2018secondary}. Once emitted, mixing state can further be modified by condensation of low-volatility compounds, such as the sulfate produced from oxidation of $\rm SO_2$ by OH in gas-phase. %Heterogeneous reactions between gas-phase reactants and condensed-phase surface can be faster than reactions in gas-phase, and one important reaction is the ozone depletion by chlorine radials produced from heterogeneous reactions between chlorofluorocarbons and polar cloud surface \citep{davies2018heterogeneous}. 
Coagulation between particles is an efficient process in changing particle number concentration and Brownian coagulation had been proved to the main reason for the 
rapid evolution of soot particle size distribution near a highway emitting point \citep{jacobson2004evolution}. Coagulation process alone will make particles more similar and particle population will become more internally mixed \citep{Riemer2013a}. The processes outlined above are for the cloud-free environment, however, since the global cloud coverage is more than 0.6 on average \citep{stubenrauch2013assessment}, evolution in clouds is also an important course during the lifetime of aerosol populations.

Figure~\ref{fig:chap1-aq-proc} shows the chemical and physical processes particles experience in fogs and clouds. When the atmosphere reach supersaturated, particles with critical supersaturation lower than the environment supersaturation are activated as cloud droplets and undergo nucleation scavenging. For these activated particles, the absorbed water facilitates the aqueous chemical reactions. Gas species dissolve in the cloud, and if both liquid and ice phase exist in the cloud, species partition between different phase occurs and the retention coefficients in liquid phase are varied. When cloud droplets grow to rain droplets, they fall out and precipitate. During the falling out procedure, some small interstial particles are collected by these large droplets and impaction scavenging occurs. For clouds with strong convection, gas and particle species are vertically redistributed by convective transport. 

\begin{figure}
	\centering
	\includegraphics[scale=1.0]{chap1_figs/thesis_chap1_fig4.jpeg}
	\caption{Processes schematic invovled in fogs and clouds, adapted from \citet{ervens2015modeling}.}
	\label{fig:chap1-aq-proc}
\end{figure}

Aqueous environment provides an efficient medium for sulfate formation. Globally, more than 50\% of sulfate forms through in-cloud oxidation and the formation rates of aqueous reactions, $\rm SO_2$ oxidation by $\rm H_2O_2$ and $\rm O_3$, are more effective than through OH oxidation in gas phase \citep{kreidenweis2003modification, rasch2000description}. Other aqueous sulfate formation mechanisms, such as the reactions catalyzed by Transition Metal Ions (TMI), can also be important and it is confirmed to be the dominant pathway for the sulfate formation in the samples collected during HCCT-2010 field campaign \citep{harris2012sulfur, harris2013enhanced}. 

%Reaction rates of aqueous processes rely on cloud properties and the importance of different pathways differs at various cloud types.  \citet{straub2007chemical}, analyzing the marine stratocumulus cloud samples collected over eastern Pacific Ocean during DYCOMS-II field campaign, found $\rm SO_2$ oxidized by $\rm H_2O_2$ is the dominant reaction for sulfate production.
Recently, more emphasis transfers to the role of aqueous reactions for SOA formation. Globally, SOA produced from aqueous pathway can reach to 20--30 Tg·$\rm yr^{-1}$. Glyoxal, methylglyoxal, glycolaldehyde and acetic acid can act as the precursors for aqueous SOA \citep{liu2012global}. As for the oxidants, besides hydroxyl radical, which is the dominant oxidant for SOA formation at gas phase, aqueous environment can provide several other efficient oxidants, such as peroxyl radicals, peroxides and triplet excited states of organic compounds ($^{3}\textrm{C}^*$) \citep{mcneill2015aqueous, ervens2011secondary}. Using simulated sunlight UV, \citet{smith2014secondary} found phenols can be rapidly oxidized by $^{3}\textrm{C}^*$ to produce low-volatility SOA . 

Aerosol size distribution evolves after the formation of aqueous inorgaincs and organics. Hoppel minima, which describes the CCN concentration gap between two particle distribution modes peak at 0.02--0.03 and 0.08-0.15~$\rm \mu m$ respectively,  was found by \citet{hoppel1986effect} when analyzing the particles processed by nonprecipitation clouds. This phenomena has further been observed for the particles at different region, such as VOCALS campaign over west Chilean coast at Arica and MASE experiment over central California coast \citep{kleinman2012aerosol, hudson2015cloud}. Besides the aqueous aerosol formation, particle size distribution also changes due to coagulation between interstitial particles and cloud droplets. \citet{pierce2015importance} found this process can reduce total particle number concentration by 10--15\% globally. 

Evolution of mixing state affects aerosol climate-related properties. \citet{Ching2016} found, for those already aged BC-contained particles, increased BC emission can result in more cloud droplet number concentrations. Optical Aerosol optical properties are altered as mixing state evolves, especially for BC-contained particles. The enhancement of optical absorption abilities of BC-contained particles due to the surrounding coatings are widely confirmed by models and laboratory \citep{Moffet2009,Liu2017,wu2020light,Fierce2020}.

Considering the importance of mixing state for describing an aerosol population, qualitative description using the terms, such as internal or external mixture, can potentially lead to uncertainties in quantifying the effects of mixing state. \citet{Riemer2013a}, based on the theory of information-theoretic entropy, developed index $\chi$ to quantify aerosol population mixing state. The index $\chi$ is the ratio between average particle diversity $D_{\alpha}$ and bulk diversity $D_{\gamma}$. By using particle mass measurement data, these metrics were successfully proved to be capable of explaining the aerosol aging processes \citep{Healy2014}.  
\section{Aerosol modeling approaches}
Application of diversified measuring techniques help advance our understanding of aerosol particles, and we know the observed particles are the consequence of multiple processes. By using numerical models, we can distinguish the underlining physical and chemical laws leading to the complexity of aerosol behaviour, and predict what will happen if certain processes change. Considering the wide ranges of species types and sizes for an aerosol population, it is hard for a single model to represent the complete processes the population can experience during its lifetime, from emission to deposition, and balance need to be reached between model accuracy and computational efficiency. This section summaries the current aerosol modeling approaches from simplistic bulk methods used in global models to comprehensive particle-resolved models used in box model, and we will focus on how these models deal with aerosol mixing state and its implication for aerosol related climate properties. 

\subsection{Bulk models}
The simplest aerosol modeling approach is using bulk method. Aerosol population are represented by several common species mass concentration, including sulfate, nitrate, BC, organics, dust and sea salt, and these species are treated as external mixtures in the bulk without detail information about how the species mixed with each other. Rather than tracking the aerosol evolution through microphysical processes, this approach prescribes the aerosol size distribution from climatology data. This approach is computationally efficient and applied by several global models, such as GOCART \citep{chin2000atmospheric} and TM5 \citep{vignati2010sources}. 

\subsection{Modal models}
A more advanced approach is representing the aerosol populations by several modes. Just as shown in Fig.~\ref{fig:chap1-aerosol-model}(a), modal model tracks the size distribution evolution of the modes. Each mode can contain a variety of species and mixing state of the population are represented by the modes. But inside each mode, all species are still assumed to internally mixed. Considering the wide range of particle diameters, it is convenient to use lognormal function to represent the number distribution of each mode as follows:
\begin{equation}
\centering
    n({\rm log}D_p) = \sum_{i=1}^{m}\frac{N_i}{\sqrt{2\pi}{\rm log}\sigma_i}{\rm exp}(-\frac{({\rm log}D-{\rm log}{\overline{D}_i})^2}{2{\rm log}^2\sigma_i}),
\end{equation}
where $m$ is the number of modes, $N_i$ is the number concentration of mode $i$, $\overline{D}_{i}$ and $\sigma_i$ are for the geometric mean diameter and standard deviation of the distribution respectively. The number of modes can be varied among the models. Aitken, accumulation and coarse modes are the essential three modes used in global models, such as MADE used in ECHAM4 global climate model and CMAQ regional chemistry model \citep{lauer2005simulating, binkowski2003models}. Other models apply additional modes to better represent the hydrophilic and hydrophobic species. For example, MAM7 used in CAM5 global model apply additional primary carbonaceous mode, including emitted primary organic matter and BC, which has lower hygroscopicity than accumulation mode \citep{liu2012toward}. For the three parameters used to describe the lognormal distribution, standard deviation of each mode is prescribed, and number concentration and mean diameter change as particles evolve.

\begin{figure}
	\centering
	\includegraphics[scale=0.5]{chap1_figs/thesis_chap1_fig5.pdf}
	\caption{Aerosol numerical approaches:(a) Modal model, (b) Section model, and (c) Particle-resolved model}
	\label{fig:chap1-aerosol-model}
\end{figure}

\subsection{Sectional models}
Similar to modal methods, sectional aerosol models are also distribution-based. Instead of tracking the population by different modes, this approach discrete the population to a certain number of size bins and then track the changes in each bin, as illustrated in Fig.~\ref{fig:chap1-aerosol-model}(b). Particles are internally mixed within the bins and externally mixed between the bins. These aerosol modules are widely applied to large-scale models. For example, MOSAIC with flexible size bin numbers is extensively used in WRF-Chem and proved to capture the summer particulate matter distribution well over Houston region \citep{zaveri2008model,fast2006evolution}. Initially with 30 bins resolving particles with diameter between 0.01 and 10 $\rm \mu m$, TOMAS sectional model is applied to several global chemistry models, such as GEOS-Chem, and improved its ability in simulating aerosol optical properties recently through more detail mixing state representation of BC-contained particles \citep{adams2002predicting,pierce2013weak,kodros2018size}.

Most sectional models apply univariate distribution of aerosol properties with assumptions used to describe aerosol mixing state in each bin, and more advanced schemes have been developed to incorporate more particle details inside the size bin, especially BC mass fraction of the particles. Based on Model of Aerosol Dynamics, Reaction, Ionization, and Dissolution (MADRID) module, \citet{oshima2009aging} developed a two-dimensional sectional scheme MADRID-BC by adding another dimension for BC mass fraction and tracking the fraction changes due to condensation/evaporation processes. Through developing MOSAIC-MIX, \citet{ching2016three} further extended the sectional model representation by incorporating another dimension for hygroscopicity, and coagulation process had been added to evaluate its effects on BC mixing state. 

\subsection{Particle-resolved models}
Even though modal and sectional models provide better representation of particle distribution than bulk models, aerosol mixing state are still simplified by using internal or external assumptions. A more advanced representation of aerosol mixing state is particle-resolved method, as illustrated in Fig.~\ref{fig:chap1-aerosol-model}(c). Rather than tracking particles by distribution, particle-resolved model applies Langrangian method to track the particles individually and with no assumptions about the components of the particle. Specifically, each particle is represented by a $A$-dimensional vector, where $A$ is the total species number. The first particle-resolved aerosol model was developed by \citet{Riemer2009} and it was coupled with MOSAIC module to include gas-particle partitioning and gas chemistry processes. The model was applied to investigate the aging process of BC-contained particles in a urban plume environment and the evolution of ship plume particles \citep{tian2014modeling, Ching2016}. Recently, PartMC-MOSAIC had been further coupled with WRF-chem to track particle evolution at large-scale domain \citep{curtis2017single}. However, MOSAIC only considered aerosol aging process for subsaturated environment (RH < 100\%). It can not be used to investigate the physical and chemical processes in a cloud environment. This is one of the goal for this thesis, and I achieved this by coupling an aqueous module Chemical Aqueous Phase Radical Mechanism (CAPRAM). 

In summary, bulk model approach is the most simplified methods and an universal mixing state will apply for all the particles in the population. Modal and sectional models divide aerosol populations by modes and bins. Its ability to describe aerosol mixing state is limited by the number of modes or bins included in the models. With the particle-resolved approach, each particle is explicitly tracked and the mixing state of aerosol population is accurately represented.  

\section{Models for aqueous chemistry simulation }
One of the topics for the thesis is to investigate the interactions between aerosol mixing state and aqueous chemistry. The previous section discussed the different approaches to simulate aerosol mixing state, and it is different story when comes to incorporating aqueous mechanism for models. From particle activation at manometer scale to stratocumulus structure over thousand kilometers, the magnitude of cloud processes involves around $10^{14}$ orders and it is hard for a single model to incorporate all the related processes. Models with aqueous processes can be divided into two groups: process model with explicit aerosol microphysics and complex aqueous chemical mechanisms, and large-scale model with simplified representation of cloud properties and processes. 

For process models, such as box and parcel models, aerosol microphysics are well-resolved. Droplet activation is explicitly described by using K$\rm \ddot{o}$hler theory. Specifically, particles with critical supersaturation lower than maximum supersaturation reached in the environment are activated as droplets \citep{rothenberg2016metamodeling, ching2012impacts}. Detail aqueous chemistry processes can be incorporated to these cloud parcel models. An example is the cloud parcel model SPACCIM coupled with detail aqueous scheme CAPRAM (492 species and 1087 reactions in version 3.0), and it had been testified using FEBUKO observation data \citep{wolke2005spaccim}. Since particle size distributions are commonly described in process models, the modification of particles after cloud processes can also be captured. The limitation of these models is lacking the interactions with cloud macro dynamics.

It is hard to include such detail information for global models with coarse grids. Rather than resolving cloud microphysics through explicit equations, cloud properties are described by using cloud fraction, with other diagnosed meterological parameters, for each grid. Cloud droplet sizes are assumed to be certain sizes. For example, ECHAM5-HAM assumed two bins for the cloud droplets: one bin for particles with lower ion concentration and the other bin for high ion concentration. As for the effects of cloud droplets representation on cloud chemistry, \citet{barth2006importance}, by camparing sectional and single-size droplet parcel model, found simplified cloud droplet size representation will lead to biases in simulated formic acid and formaldehyde concentration. Simplified aqueous mechanisms are also used and the acidity, which is an important factor for aqueous chemistry, are fixed at constant value. For example, GEOS-Chem used a pH of 4.5 for $\rm SO_2$ oxidation reaction by $\rm O_3$\citep{park2004natural}. Furthermore, assumptions should be made to consider the redistribution of species produced from aqueous reactions. For example, in CMAQ, all the non-volatile aqueous-formed species are added to accumulation mode after cloud dissipates \citep{binkowski2003models, fahey2017framework}. 
 
%\begin{table}
%\setlength\extrarowheight{5pt}
%\centering
%\caption{Overview of acidity, mixing state and optical calculation treatment in different models}
%\label{tab:input}
%\begin{tabular}{ c c c c c}
%	\hline
%	Model acronym  & Model scale &  pH  &  mixing state & Optical calculation  \\
%	\hline
%    CMAQ & Regional & Equilibrium pH & Internal mixture in each mode & \\
%    WRF-Chem (MOSAIC) & Regional  & Equilibrium pH & Internal mixture in each bin &\\
%    GEOS-Chem (TOMAS) & Global & fixed pH at 4.5 & Internal mixture in each bin & \\
%    MOZART & Global & pH based on charge balance &  & \\
%    ECHAM5-HAM & Global & 
%    \begin{tabular}{@{}c@{}}pH based on initial aerosol \\ at small and large droplets\end{tabular} & Internal mixture in each mode & \\
%    PartMC-MOSAIC & 0-D & \begin{tabular}{@{}c@{}}pH based on detailed \\ mechanism at CAPRAM 2.4\end{tabular} & particle-resolved & \\
%	\hline
%   \end{tabular}
% \end{table}
\section{Research questions and thesis organization}
Aerosol particles affect climate forcing through directly altering radiation and indirectly interactions with clouds. These effects are dependent on particle chemical species, in other words, the chemical species mixing state of the aerosol population. The objective of this dissertation includes two parts: quantifying the cloud chemical processes, and quantifying the effects of aerosol mixing state for its optical properties.

For the first part, I will focus on the following scientific questions: (1.1) Is particle-resolved approach capable of simulating the complicated aqueous processes? (1.2) To what extent does cloud processing change the aerosol mixing state of the population that entered the cloud?  (1.3) How does this change the cloud condensation number concentration and optical properties? (1.4) What is the implication of coagulation between the interstitial particles and cloud droplets for mixing state of aerosol population? 

For the second part, the following two questions will be addressed: (2.1) What are the errors in aerosol optical properties introduced by internal mixture assumptions used in sectional models? (2.2) Will the optical properties error sectional models amplified or dampen at humidified environment? 

Chapter 2 answers the first question by coupling particle-resolved model PartMC-MOSAIC with aqueous mechanism CAPRAM. This work designed ensemble cases with three levels of initial gas mixing ratio and temperature lapse rates to find out whether particle-resolved approach is capable of simulating the complicated aqueous sulfate formation reactions and quantify the role TMI-catalyzed oxidation pathway for aqueous sulfate formation. The findings will be concluded in the preparing paper ``Sensitivity of sulfate in-cloud chemistry to aerosol acidity variability and mixing state'' for \textit{Aerosol Science and Technology}. 

Chapter 3 investigates the cloud processing effects on aerosol properties. In this work, I used a typical urban plume particle population to experience four cloud cycles to determine the changes of aerosol mixing state and quantify the changes of aerosol microphysical and optical properties after cloud evaporates. Another simulation is conducted to identify the role of coagulation between cloud droplets and interstitial particles. This chapter will answer questions 1.2--1.4. A paper entitled ``The impacts of cloud processing on resuspended aerosol particles after cloud evaporation'', which is in review for \textit{Journal of Geophysical Research-Atmosphere}, summaries the results on the changes of aerosol microphysical properties after cloud processing. Another paper, which is entitled with ``Quantifying the Effects of Cloud Processing on Aerosol Optical Properties Using a Particle-Resolved Model'' about the changes of optical properties after cloud processing, is going to submit to \textit{Aerosol Science and Technology}. 

Chapter 4 evaluates the error introduced by the internal mixture assumptions used in sectional aerosol models. The error is quantified by comparing the aerosol optical value differences between reference populations created by running particle-resolved model and sensitivity populations with reduced representation of mixing state created by composition-averaging methods. This chapter will answer the two questions in second part and the work is submitted to \textit{Atmospheric Chemistry and Physics}, entitled ``Quantifying the effects of mixing state on aerosol optical properties''.

Chapter 5 summaries the main findings of these works and implications for resolving aerosol mixing state in the future for global models.  
%%%%%%%%%%%%%%%%%%%%%%%%%%%%%
%%%%%%%%%Chapter 2%%%%%%%%%%%
%%%%%%%%%%%%%%%%%%%%%%%%%%%%%
\chapter{Particle-resolved modeling of aqueous-phase chemistry}

This chapter presented the application of PartMC as a particle-resolved aqueous-phase chemistry model. In this work, I first described PartMC-MOSAIC model, and the coupled cloud parcel and aqueous chemistry module. Then, I evaluated the coupled model by comparing with other size-resolved and bulk cloud chemistry models. Furthermore, I designed ensemble scenarios to investigate the efficiency of different aqueous sulfate formation pathways. Another group of ensemble scenarios are conducted by adding transition metal ions (TMI) in the population to determine the efficiency of sulfur oxidation reactions catalyzed by TMI. 

\label{chap2:mon}
%%% Suggested section heads:
\section{Introduction}

Clouds cover around 70$\%$ of the earth \citep{Stubenrauch2013}. The coexistence of species multi-phases in the cloud provides a favorite medium for the interactions between different phases \citep{Deguillaume2005}. Gas can transfer to the liquid droplets through thermodynamic processes and undergoes heterogeneous chemistry, which in return affect the particle activation potentials \citep{Farmer2015, Henning2014}. Oxidation of dissolved gases in the aqueous phase can contribute to a large fraction of particle species, especially for the secondary inorganic and organic species. Globally, over 50\% of sulfate are produced through aqueous reactions \citep{Philip2014, Roth2016}. The goal of this work is to investigate the roles of different aqueous sulfate formation pathways. 

Particles with different composition and size experience different net effects from aqueous chemical processing. First, per-particle properties affect cloud processes by determining which particle can be activated as cloud droplets. The activation potential of particles is determined by its size and compositions, and CCN closure studies indicated the simplified assumptions of particle mixture in the population can lead to simulated CCN concentration bias \citep{Broekhuizen2006, Bhattu2015}. Second, particle diversity affects aqueous reaction rates. Aqueous sulfate formation reactions are highly pH-dependent, and the reaction rates for particles with different acidity can vary in several orders. For example, transition metal ions (TMI), including Fe and Mn ions, which are mostly originated from mineral dust \citep{alexander2009transition}, can catalyze aqueous sulfate oxidation reactions. The related pathways can become dominant when pH is larger than 6.0 \citep{Seinfeld2006a}. It had been proved that at some cloud events, $\rm SO_2$ oxidation reactions catalyzed by coarse mineral dust TMI can be more efficient than $\rm H_2O_2$ pathway \citep{Harris2013a, Harris2014}. 

Despite the importance of particle heterogeneity in aqueous chemistry, most regional or global models used simplified assumptions to simulate aqueous processes. For global chemistry model GEOS-Chem, less than 10 dissolved species are used to calculate acidity \citep{alexander2012isotopic}.
For regional model CMAQ, cloud chemistry processes are tracked by three lognormal modes, neglecting the heterogeneity inside each mode \citep{fahey2017framework}. In this work, I applied the particle-resolved model PartMC-MOSAIC \citep{Riemer2009, Zaveri2010a}, which can track particles individually. This approach is well-suited for this problem because it can track the evolution of compositions and sizes of individual aerosol particles without averaging their composition within size bins or modes. The model had been widely used to study particle aging processes in cloud-free environment \citep{ching2012impacts, Ching2016}, and in current work, I first applied the model with cloud chemistry processes. 

This work was structured as follows: Section~\ref{chap2.2} described the models used in current work and Sect.~\ref{chap2.3} evaluated the particle-resolved aqueous chemistry model by comparing with other widely-used models. Section~\ref{chap2.4} investigated the role of different aqueous sulfate formation pathways by analyzing the ensemble cases and the role of TMI pathway is further explored in Sect.~\ref{chap2.5}. Findings were concluded in Sect.~\ref{chap2.6}. 

\section{Model description and simulations setup}
\label{chap2.2}
Following the two-step strategy described in \citet{ching2012impacts}, I designed the cloud chemistry experiment by exposing aerosol populations in a cloud environment, as figure~\ref{chap2-fig1-frame} illustrates. For the first step, Isimulated urban plume scenarios that produced populations with a wide variety of aerosol mixing states using a particle-resolved model PartMC-MOSAIC in a cloud-free environment. These simulated aerosol populations were then used as the input for cloud parcel simulations, including aqueous chemistry, in step 2. Details of PartMC-MOSAIC, cloud parcel model and aqueous chemistry are described in the following sections.

\begin{figure}[ht]
    \centering 
    \includegraphics[scale=0.4]{chap2_figs/chap2-fig1-frame.pdf}
    \caption{Two-step cloud chemistry experiment framework}
    \label{chap2-fig1-frame}
\end{figure}

\subsection{Particle-resolved model PartMC-MOSAIC}
PartMC-MOSAIC (Particle Monte Carlo-Model for Simulating Aerosol Interactions and Chemistry) is a Lagrangian box model which simulates the evolution of individual particles by different processes. The evolution is simulated in a well-mixed computational volume and the particle spatial positions are not stored. Compositions of each particle are explicitly tracked by representing the particle as an $A$-dimension vector $\vec{\mu}^i \in \mathbb{R}^A$ with components ($\vec{\mu}_1^i,\vec{\mu}_2^i,...,\vec{\mu}_A^i$), where $\vec{\mu}_a^i$ is the mass of species $a$ in particle $i$, with $a$ = $1,...,A$ and $i$ = $1,...,N$. The evolution for aerosol number concentration with species mass $\mu$ at time $t$, $n(\vec{\mu},t)$, is described by: 
\begin{equation}
\begin{aligned}
 \frac{\partial n(\vec{\mu},t)}{\partial t} &= \underbrace{\frac{1}{2}\int_0^{\mu_1}\int_0^{\mu_2}...\int_0^{\mu_A} K(\vec{\mu}',\vec{\mu}-\vec{\mu}')n(\vec{\mu}',t)n(\vec{u}-\vec{u}',t)d{\mu_1}'d{\mu_2}'...d{\mu_A}'}_\text{coagulation gain}\\
& - \underbrace{\int_0^\infty\int_0^\infty...\int_0^\infty K(\vec{\mu},\vec{\mu}')n(\vec{\mu},t)n(\vec{\mu}',t)d\mu_1'd\mu_2'...d\mu_A' }_\text{coagulation loss} + \underbrace{\dot{n}_{\rm emit}(\vec{\mu},t)}_\text{emission} + \underbrace{\lambda_{\rm dil}(t)(n_{\rm back}(\vec{u},t))-n(\vec{\mu},t)}_\text{dilution}\\
&-\underbrace{\sum_{i=1}^{C}\frac{\partial}{\partial\mu_i}(c_iI_i(\vec{\mu},\vec{g},t))n(\vec{\mu,t})}_\text{gas-particle transfer} - \underbrace{\frac{\partial}{\partial\mu_{C+1}}(c_wI_w)(\vec{\mu},\vec{g},t))n(\vec{\mu,t})}_\text{water transfer} + \underbrace{\frac{1}{\rho_{\rm dry}(t)}\frac{d\rho_{\rm dry}(t)}{dt}n(\vec{\mu},t)}_\text{air density change},
\end{aligned}
\label{eq:partmc}
\end{equation}
where $K$ is the coagulation rate between the particles, $\dot{n}_{\rm emit}(\vec{\mu},t)$ is the emitting distribution of species $\vec{\mu}$, $\lambda_{\rm dil}$ is the dilution rate with background species concentration $n_{\rm back}$, $c_i$ is the gas to particle conversion rate, $I_i$ is the gas species condensation flux, $c_{\rm w}$ is the gas to water conversion rate and $I_{\rm w}$ is the water condensation flux. More details of the equation can refer to \citet{Riemer2009}.

The evolution processes involved in the equation~\ref{eq:partmc} alter the aerosol population in two mechanisms. By adding or removing particles from the population, emission, dilution and coagulation processes modify the number concentration of the population and these processes are accomplished by PartMC with a stochastic Monte Carlo approach. While on the other hand, composition of each particle can be altered by species condensation and evaporation, which are explicitly simulated by state-of-art aerosol chemistry module MOSAIC. In MOSAIC, gas phase reactions are accomplished by carbon bond mechanism CBM-Z, 77~species and 142~reactions are included \citep{Zaveri1999}. Key aerosol species are treated, including $\rm SO_4^{2-}$, $\rm NO_3^-$, $\rm NH_4^+$, BC, primary organic aerosol (POA) and secondary organic aerosol (SOA). SOA treatment is based on SORGAM scheme \citep{schell2001modeling}. In current version, model includes four SOA species (ARO1, ARO2, ALK1, OLE1) oxidized from anthropogenic volatile organic compounds (VOCs) and four other SOA species (LIM1, LIM2, API1, API2) oxidized from biogenic VOCs \citep{ching2012impacts}. Activity coefficients of electrolytes and ions in aqueous solutions are estimated by multicomponent Taylor expansion method (MTEM), and intraparticle solid-liquid partitioning are treated by Multicomponent Equilibrium Solver for Aerosols (MESA) \citep{zaveri2005computationally}. However, MOSAIC only considered the reactions occurred in an unsaturated environment (RH < 100\%) and therefore we were not able to address the change within the cloud. This is the contribution of this work. 

PartMC-MOSAIC had been applied to analyze the particle evolution at different environment. For example, \citet{Zaveri2010a} found particle optical absorption increased by 40\% during a 48-hour idealized urban plume condition due to the aging of BC-contained particles. \citet{tian2014modeling} used the model to investigate the processes responsible for the particle number concentration change in a ship plume environment and evaluated the effects of different aging process on particle CCN properties. PartMC-MOSAIC was also used as a benchmark model to quantify the errors in aerosol optical and microphysical properties introduced by simplified mixing state assumptions commonly used in other aerosol models \citep{Zaveri2010a, ching2012impacts, Fierce2017}. PartMC-MOSAIC was further coupled with a cloud parcel model to investigate the effects of mixing state on cloud droplet properties \citep{ching2012impacts, Ching2016}, and the cloud parcel model will be discussed in detail in section~\ref{section:cloud-parcel-model} because this model was applied for the research both in  this chapter and chapter~3. 

\subsection{Cloud parcel model}
\label{section:cloud-parcel-model}
The cloud parcel model coupled with PartMC-MOSAIC is an zero-dimensional adiabatic model. It simulates the particle activation and condensation growth in a rising air parcel and tracks the changes of environment saturation due to the growth of particles and temperature change. Specifically, for a population with $N$ particles of diameter $D_i$, both the growth rate of $D_i$ and the change of environment saturation $S_v$ are diagnosed, which sums to $N$ + 1 state variables to be numerically solved by the model. To focus on the effects of chemical compositions on cloud droplets formation, entrainment, surface tension effects on droplets growth are not included in the model. This section briefly introduces the main equations solved by the parcel model. See \citet{ching2012impacts} for more detailed description of the model. 

During the condensational growth process, the chemical compositions of particle are assumed to be constant and only the water content of the particle alters. The growth rate of particle $i$ is calculated as
\begin{equation}
\centering
\frac{dD_i}{dt} = \frac{G}{D_i}(S_{\rm v}-S_{\rm eq})
\label{eq:cloud-parcel} 
\end{equation}
where the growth coefficient $G$ and droplet equilibrium supersaturation $S_{\rm eq}$ are
\begin{equation}
\centering
\begin{aligned}
 G & = \frac{4D'_{v,i}M_{\rm w}P^0}{\rho_{\rm w}RT}  ,\\
 S_{\rm eq} & = \frac{a_{{\rm w},i}}{1+\delta_i}{\rm exp}(\frac{4M_{\rm w}\sigma_{\rm w}}{\rho_{\rm w}RTD_i}\frac{1}{1+\delta_i} + \frac{\Delta H_{\rm v}M_{\rm w}}{RT}\frac{\delta_i}{1+\delta_i}),
\end{aligned}
\end{equation}
and $D'_{v,i}$ is the modified particle diffusivity, $M_{\rm w}$ and $\rho_{\rm w }$ are the water molecular weight and density, $P^0$ is the saturation vapor pressure, $R$ is the gas constant, $T$ is the environment temperature, $a_{w,i}$ is the water activity of the particle, $\sigma_{\rm w}$ is the water surface tension, $\Delta H_{\rm v}$ is latent heat of vaporization, and $\delta_i$ is defined as 
\begin{equation}
    \delta_i = \frac{\Delta H_{\rm }\rho_{\rm w}}{4k'_{{\rm a},i}T}D_i\frac{dD_i}{dt},
\end{equation}
where $k'_{{\rm a},i}$ is the corrected air thermal conductivity. 

The water activity is calculated by using the parameter $\kappa$ derived by \citet{Petters2007} and can be expressed as
\begin{equation}
\centering
a_{{\rm w},i} = \frac{v_i^{\rm w}}{v_i^{\rm w} + \kappa_i v_i^{\rm dry}},
\label{eq:activity coeff}    
\end{equation}
where $v_i^{\rm w}$ and $v_i^{\rm dry}$ are the volume of water and all the other dry components in particle $i$, respectively. $\kappa_i$ is volume-weighted $\kappa$ of the non-water species. In current model, we used $\kappa$ of 0.65 for ammonium-sulfate-nitrate system, SOA with $\kappa$ = 0.1 and POA with $\kappa$ = 0.001. BC is assumed to be hyrophobic and has $\kappa$ of 0. 

Rather than describing the rising parcel using a constant updraft velocity \citep{Seinfeld2016,rothenberg2016metamodeling}, this parcel model prescribed a constant temperature lapse rate, following the strategy used in \citet{majeed2001microphysics}, to avoid dealing with radiative heating effects and latent heat budget. Pressure is also assumed to be constant. In light of these considerations, the change of environment saturation can be given by
\begin{equation}
    \centering
    \frac{dS_{\rm v}}{dt} = -\sum_{i=1}^{N}\frac{\pi \rho_{\rm w} RT}{2M_{\rm w}P^0V_{\rm comp}}D_i^2\frac{dD_i}{dt} - \frac{1}{P^0}\frac{\partial P^0}{\partial T}S_{\rm v}\frac{dT}{dt}
\end{equation}
where the first term describes the effects due to the radius change of all $N$ particles and the second term is about the temperature change. $V_{\rm comp}$ is the computational volume. 

This coupled model had been applied to investigate the importance of aerosol mixing state for predicting cloud droplets number concentration (CDNC). \citet{ching2012impacts} found, neglecting particle species heterogeneity in size bins resulted in errors in CDNC up to 34\% at the environment with cooling rate of 0.5~\unit{K/min}. By conducting ensemble cloud parcel simulations, \citet{Ching2016} further demonstrated that ignoring BC mixing state can lead to CNDC error of $-$12\% to $+$45\%.

%The coupled model had been used to evaluate the relative importance of particle size and compositions for cloud droplet concentration. 

\subsection{Aqueous chemistry model}
\label{section:aq-chem-model}
In this work, PartMC-MOSAIC was further coupled with an aqueous chemistry module based reduced Chemical Aqueous Phase Radical Mechanism(CAPRAM) version 2.4. The full mechanism in CAPRAM 2.4 includes 439 reactions and 147 species, and a reduced version is also provided to be more computationally efficient, which includes 183 reactions and 113 species \citep{Ervens2003}. The reduced version also contains a comprehensive aqueous mechanism, and deals with the reactions between OH, $\rm HO_2$, $\rm NO_3$, $\rm SO_4$, $\rm Cl_2$,$\rm Br_2$ and $\rm CO_3$ with inorganic (TMI, $\rm NO_3^-$ , $\rm Cl^-$, $\rm Br^-$) and organic reactants with less than two atoms. The constants for thermodynamic and kinetic reactions are listed in Table~\ref{tab:capram}. When coupling CAPRAM with PartMC-MOSAIC, the original gas phase chemistry mechanism, regional atmospheric chemistry modeling (RACM), used in CAPRAM 2.4 is now replaced with CBM-Z in PartMC-MOSAIC. 

\subsection{Ensemble monodisperse settings}
This work is the first application of particle-resolved cloud chemistry model, and I used relative simple aerosol description, that is the monodisperse distribution, to investigate aqueous sulfate processes. More complex populations with lognormal distribution and the role of other cloud processes, such as coagulation between interstitial particles and cloud droplets, are explored in next chapter. 

The goal of this chapter will focus on the following questions: (1) Is particle-resolved approach capable of capturing the various roles of aqueous sulfate formation pathways under different environment? (2) What is the role of TMI pathway for aqueous sulfate formation? 

To answer the first question, I created a reference scenario ensemble by exposing the monodisperse aerosol populations at different chemical and physical environment, denoted as $P_{\rm ref}$. I set the initial particle size at 100 \unit{nm} composed of ammonium sulfate. Emission and dilution were not included in current simulations. Four important gas species for aqueous sulfate formation, including $\rm O_3$, $\rm H_2O_2$, $\rm NH_3$ and $\rm SO_2$, were perturbed at low, medium and high polluted levels, as listed in table~\ref{TMI-setting}. The $\rm O_3$ and $\rm SO_2$ values of these three levels were based on the national wide urban measurements in China \citep{wang2014spatial}, and $\rm NH_3$ levels were determined from the observations of Houston, U.S. \citep{nowak2010airborne} and Seoul, Korea \citep{phan2013analysis}. The values for $\rm H_2O_2$ were based on the measurements in Guangzhou, China \citep{hua2008atmospheric}. I also set three different temperature rates to provide cloud environment with varied liquid water content, and the corresponded cloud updraft velocity were 1.1, 1.5 and 1.8\unit{m\;s^{-1}} respectively, which represented the conditions of strong convective stratus \citep{peng2005importance}. In total, I simulated $3^5$ = 243 cases in this ensemble scenario.

\begin{table}[ht]
\centering
\caption{Settings for $P_{\rm ref}$ ensemble simulation}
\label{TMI-setting}
\begin{tabular}{c  c c  c}
\hline
Parameters & Low  & Medium & High\\
\hline
$\rm O_3$& 25 & 50 & 75 \\
$\rm H_2O_2$& 0.25 & 0.5 &  1\\
$\rm NH_3$ &2 & 4 & 6 \\
$\rm SO_2$&2&5& 10\\
$\frac{dT}{dt}$ &$-$0.3 &$ -$0.4&$-$0.5\\
\hline
\end{tabular}
\end{table}

The initial RH was 99\% and the simulation lasted 20 minutes. As showed in Fig.~\ref{chap2:ensemrh}, cases with higher temperature lapse rates experienced more rapid increase of liquid water content, while the reached maximum supersaturation were close between the cases. Cloud formed in the first minute and the maximum supersaturation was 0.2\%. 

\begin{figure}[ht]
    \centering \includegraphics[scale=0.55]{chap2_figs/chap2_mono_lwc_rh.pdf}
    \caption{Time series of (a) liquid waeter content and (b) relative humidity in the ensemble scenarios. Blue, orange and green lines are for the cases with temperature lapse rate of --0.3 \unit{K/min}, --0.4 \unit{K/min} and --0.5 \unit{K/min} respectively.}
    \label{chap2:ensemrh}
\end{figure}

In order to answer the second question, I created another sensitivity scenario ensemble, denoted as $P_{\rm TMI}$ by adding $\rm Fe(II)$ and $\rm Fe(III)$ in the initial aerosol population. I set the mass fraction of $\rm Fe(II)$ and $\rm Fe(III)$ to be 0.05 and 0.01. \citet{deguillaume2004role} found aqueous hydroxide is important for TMI oxidation pathways and the concentration is in the order of $\rm 10^{-13}$~\unit{M}. I achieved this by emitting gas OH at the rate of $2\times 10^{-8}$ \unit{mole/m^2/s}. I also applied the same perturbed parameters as in $P_{\rm sul}$, which results in another 243 cases for $P_{\rm TMI}$. 

\section{Sulfate mechanism verification}
\label{chap2.3}
Before applying the comprehensive particle-resolved aqueous chemistry model to study the role of different aqueous sulfate pathways, I first evaluated it by comparing with the models used in \citet{kreidenweis2003modification}(hereafter, KS2003). In KS2003, several bulk and size-resolved models were used to simulate aqueous sulfate formation in an adiabatic updraft cloud environment. They found significant differences in $\rm SO_2$ oxidation rates between size-resolved and bulk models. This model comparison work provides a benchmark for checking aqueous sulfate mechanisms, and I used the same simulation setting in that work to verify our particle-resolved aqueous chemistry approach. Physical and chemical conditions, including the updraft velocity, temperature change, initial species concentration and chemical equations are all set to the same with values used in KS2003, as listed in table~\ref{setting}. In the models used in KS2003, cloud formed by a constant updraft velocity. But in our model, as mentioned before, we predefined the temperature lapse rate. In order to get the same updraft velocity, we obtained the temperature values of constant updraft velocity at 0.5 from pyrcel model, a zero-dimensional adiabatic cloud parcel model \citep{rothenberg2016metamodeling}. 

\begin{table}[ht]
\centering
\begin{threeparttable}
\caption{Cloud  chemical and physical conditions}
 \begin{tabular}{l c|c l}
 \hline
  Physical parameters & Value (Units) & Chemical parameters ($t = 0$) & Values Units\\
 \hline
 Temperature at $t =0$ & 285.2 (K)&$\rm SO_2$ & 200 (pptv)\\
 Pressure at $t$ = 0 & 950 (mbar)&$\rm NH_3$ & 100 (pptv)\\
 Updraft velocity & 0.5 ($\rm m\,s^{-1}$)&$\rm H_2O_2$ & 500 (pptv)\\
 Cloud water mixing ratio after 2400 s& 2.17 ($\rm g\,kg^{-1}$)&$\rm HNO_3$  & 100 (pptv)\\
 Air density at the cloud base & 1.15 ($\rm kg\,m^{-3}$) & $\rm O_3$ &50 (ppbv)\\
 Cloud base temperature& 284.2 K&$\rm CO_2$ & 360 (ppmv)\\
 Cloud base pressure& 939 (mbar)& $\rm SO_4^{2-}$ & 2 ($\rm \mu g \, m^{-3}$)\\ 
 Relative humidity at $t$ =0 & 95$\rm \%$ & $\rm NH_4^+$ & 0.375 ($\rm \mu g\, m^{-3}$)\\
 \hline
 \label{setting}
\end{tabular}
\end{threeparttable}
\end{table}

Figure~\ref{chap2:ks2003}(a) shows the simulated cloud liquid water content (LWC). PartMC simulated higher LWC changing rate and it may result from the different cloud parcel model settings. As mentioned before, we prescribed a constant temperature lapse rate. But for models used in KS2003, a constant updraft velocity was described. In current simulation, the input temperature profile produced updraft velocity close to 0.5~\unit{m/s}, but still with some fluctuations around 0.1~\unit{m/s}, and this can be the main reason for the LWC differences. Another possible factor is the treatment of droplet surface temperature $T$, which is used to calculate droplet growth rate and it is common for models to assume $T$ equals to environment temperature $T_\infty$. But in our model, we did not use this assumption and the different treatments may also result in the calculated LWC differences.

\begin{figure}[ht]
    \centering \includegraphics[scale=0.5]{chap2_figs/chap2_fig1_profile.pdf}
    \caption{Vertical profile differences of (a) liquid water content and (b) Gas $\rm SO_2$ concentration (c) Bulk pH between KS2003 and PartMC.}
    \label{chap2:ks2003}
\end{figure}

The simulated $\rm SO_2$(g) depleting rate is almost consistent between PartMC and KS2003 (Fig.~\ref{chap2:ks2003}(b)), with $\rm SO_2$(g) mixing ratio decreasing from 0.2 to 0.025~ppb after 40 minutes, indicating PartMC is capable of simulating of aqueous sulfate formation processes. The bulk acidity calculated by PartMC also shows similar profile with the models in KS2003 but with higher values (Fig.~\ref{chap2:ks2003}(c)). Considering the different simulation approaches, we are expecting the differences between bulk or size-binned based pH and our particle-resolved pH. 

Since sulfate aqueous formation pathways are highly acidity dependent, especially through $\rm O_3$, where reaction rates increase exponentially with increasing pH, the acidity differences imply noticeable changes in sulfate production. At the end of simulation, PartMC simulated sulfate concentration of 207 ppb, higher than the $\sim$175 in size-resolved models and $\sim$145 in bulk models. As explained in KS2003, the excess produced sulfate in the size-resolved models are found to be associated with higher simulated pH and more sulfate formed from $\rm O_3$ oxidation pathway. This explanation is still applicable for our current findings. 

In summary, this comparison simulation proved our particle-resolved aqueous model is capable of capturing the sulfate aqueous chemical processes. 
With more detail particle information, our particle-resolved approach may also provide more comprehensive understanding of different sulfate formation pathways, and this is explored in our next section. 
%\begin{figure}
\section{Contribution of different aqueous sulfate formation pathways}
\label{chap2.4}
This section explored the important factors for higher aqueous sulfate formation by analyzing the 243 aerosol populations in $P_{\rm ref}$.
Figure~\ref{chap2:ens_three} shows the sulfate mixing ratio in the ensemble scenarios, and values are categorized by different factors. There are clear sulfate formation separations between the cases with different ammonia gas mixing ratio, and cases with more sulfate production are associated with high ammonia environment. But this separation is not clear for other three considered factors, because cases with higher sulfate formation can be with all the three levels of temperature lapse rate, $\rm H_2O_2(g)$ and $\rm O_3(g)$mixing ratio. This phenomena indicates ammonia gas concentration is a determined factor for aqueous sulfate formation in current simulated scenarios. Next, we will look into the underlying theories.

\begin{figure}[ht]
    \centering \includegraphics[scale=0.6]{chap2_figs/chap2_fig2_mono_ensemble.pdf}
    \caption{Sulfate time series categorized by four different settings in $P_{\rm ref}$: (a) initial $\rm NH_3(g)$ mixing ratio, (b) temperature lapse rate, (c) initial $\rm H_2O_2(g)$ mixing ratio, and (d) $\rm O_3(g)$ mixing ratio. The three colors and symbols are for the low, medium and high levels of each parameter.}
    \label{chap2:ens_three}
\end{figure}

The dissolved $\rm SO_2$ in cloud can appear in three S(IV) forms: $\rm SO_2\cdot H_2O$, $\rm HSO_3^-$ and $\rm SO_3^{2-}$. In this work, 
I analyzed four key oxidation pathways which transfer these S(IV) species to S(VI):
\begin{align}
\centering
    &\ce{HSO3^- + H2O2{(aq)} + H^+ -> SO4^{2-} + 2H^+ + [H2O](aq)}\\
    &\ce{HSO3^- + O3{(aq)} -> SO4^{2-} + H^+ + O2{(aq)}} \\
    &\ce{SO3^{2-} + O3{(aq)} -> SO4^{2-} + O2{(aq)}} \\
    &\ce{S(IV) + \frac{1}{2}O2 ->[TMI] S(VI)}
\label{eq:chem-eq}
\end{align}
This section will focus on the discussion of the first three oxidation pathways and transition pathway reaction (2.9) will be discussed in next section.

Figure~\ref{chap2:su-acidity} shows the relationship between sulfate mixing ratio and particle acidity at 3~\unit{min}. For the cases with same $\rm SO_2(g)$, sulfate mixing ratio is higher for those with higher $\rm NH_3(g)$. This is because of the higher pH values for these cases. Considering the cases with $\rm SO_2(g)$ of 5~\unit{ppb} and $\rm O_3(g)$ of 50~\unit{ppb}(the cases in the red boxes of Fig~\ref{chap2:su-acidity}), the only differences for these cases are $\rm NH_3(g)$ values. I found for all these three lapse rates, when $\rm NH_3(g)$ increased from 2 to 6~\unit{ppb}, pH increased from 5.0 to over 5.5 and sulfate mixing ratio increased from 1~\unit{ppb} to more than 2.5~\unit{ppb}. This is because the increased reaction rates of $\rm O_3$ pathways (Fig.~\ref{chap2:reac-rates}). In Fig.~\ref{chap2:reac-rates}, I also found for cases with 10~\unit{ppb} $\rm SO_2$, $\rm H_2O_2$ reaction rates are almost constant at all $\rm NH_3$ levels. The different responses of $\rm O_3$ and $\rm H_2O_2$ pathways are consistent with what \citet{Seinfeld2016} found. For $\rm O_3$ pathways, high pH leads to high S(IV) solubility and reaction rates increased. The reactions are self-limiting because the produced sulfate will increase acidity. For $\rm H_2O_2$ pathway, it is insensitive to pH because of the canceling effects between increased S(IV) concentration and reduced reaction constants. I also noticed for the cases with 2~\unit{ppb} $\rm SO_2$, reaction rates are smaller for those with higher $\rm NH_3$ and pH. These are the $\rm SO_2(g)$-limited cases and the less dissolved $\rm SO_2$ are responsible for this.  

\begin{figure}[ht]
    \centering \includegraphics[scale=0.55]{chap2_figs/chap2_fig4_sulfate_pH_3min.pdf}
    \caption{Correlation between sulfate mixing ratio and acidity at 3~\unit{min}. (a) is for cases with temperature lapse rate of $-0.3$~\unit{K/min}. (b) is for cases with temperature lapse rate of $-0.4$~\unit{K/min}. (c) is for cases with temperature lapse rate of $-0.5$~\unit{K/min}. Symbols in the plots are colored by $\rm SO_2$(g) levels and symbol types are for $\rm NH_3$(g) levels. All the cases in the figure are with $\rm H_2O_2(g)$ = 0.5~\unit{ppb}. Red boxes are for the aerosol populations with only differences in $\rm NH_3(g)$ and are analyzed for detail in the text.}
    \label{chap2:su-acidity}
\end{figure}

As for the sensitivity to temperature, there is no significant differences for cases with different temperature lapse rates (Fig~\ref{chap2:reac-rates}). As \citet{Seinfeld2016} proposed, the effects of temperature on reaction rates are the balancing effects between two factors. For one hand, lower temperature leads to higher solubility and higher reactions rates. For the other hand, reaction constants decreased with lower temperature. Thus, when temperature changes, the behaviour of reactions rates are determined by which factor is dominant. 

\begin{figure}[ht]
    \centering \includegraphics[scale=0.55]{chap2_figs/chap2_fig6_sulfate_pH.pdf}
    \caption{Correlation between aqueous sulfate formation pathway rates and particle acidity at 3~min. (a) is for cases with temperature lapse rate of $-0.3$~\unit{K/min}. (b) is for cases with temperature lapse rate of $-0.4$~\unit{K/min}. (c) is for cases with temperature lapse rate of $-0.5$~\unit{K/min}. Symbol types are for different oxidation pathways: circle is for $\rm H_2O_2+ HSO_3^-$, diamond is for $\rm O_3+ HSO_3^-$ and triangle is for $\rm O_3+ SO_3^{2-}$. Color is for $\rm SO_2(g)$ level and Symbol size represents $\rm NH_3(g)$ level. All the cases in the figure are with $\rm O_3(g)$ = 50~\unit{ppb} and $\rm H_2O_2(g)$ = 0.5~\unit{ppb}.}
    \label{chap2:reac-rates}
\end{figure}

%Detail statistics of the three pathways are further explored in Fig.~\ref{chap2:reac-rates}. Sulfate formation through $\rm H_2O_2+ HSO_3^-$ and $\rm O_3(aq)+ SO_3^{2-}$ pathways are 1--2 orders higher than $\rm O_3(aq) + HSO_3^-$. The higher rates of $\rm O_3+ SO_3^{2-}$ than $\rm O_3+ HSO_3^{-}$ can be explained by the different nucleophilic reactivity. The oxidation rates of $\rm HSO_3^-$ by $\rm H_2O_2(aq)$ and $\rm O_3(aq)$ increase rapidly with increasing $\rm SO_2$(g) mixing ratio. For the cases with 6~\unit{ppb} $\rm NH_3$(g), median reaction rates of $\rm H_2O_2(aq) + HSO_3^-$ jump from $10^{-9}$~\unit{M/s} at 2~\unit{ppb} $\rm SO_2$(g) to $10^{-7}$~\unit{M/s} at 10~\unit{ppb} $\rm SO_2$(g). But for oxidation of $\rm SO_3^{2-}$ by $\rm O_3(aq)$, the reaction rate and $\rm SO_2$(g) mixing ratio is negative correlated. That can explain why there is no strong correlation between $\rm SO_2$(g) mixing ratio and high sulfate formation rates. As for $\rm NH_3$(g), oxidation rates of $\rm HSO_3^-$ and $\rm SO_3^{2-}$ by $\rm O_3(aq)$ is higher at the cases with 6~\unit{ppb} $\rm NH_3$(g), and there is no clear change for $\rm H_2O_2(aq) + HSO_3^-$ at different $\rm NH_3$ levels. This is consistent with what we found the most sulfate formation cases are connected with high $\rm NH_3(g)$ cases. The only anomalous cases are for the aerosol populations with 2~\unit{ppb} $\rm NH_3(g)$, where the reaction rates decreased with increasing $\rm NH_3(g)$. This is the $\rm SO_2(g)$-limited cases we mentioned above. 

%\begin{figure}[ht]
%    \centering \includegraphics[scale=0.55]{chap2_figs/chap2_fig5_sulfate_rates_3min.pdf}
%    \caption{Statistics of three aqueous sulfate formation pathway rates:$\rm H_2O_2$ + $\rm HSO_3^-$(a), $\rm O_3$ + $\rm HSO_3^-$(b), $\rm O_3$ + $\rm SO_3^{2-}$. Cases are grouped by $\rm NH_3(g)$ and $\rm SO_2$.}
%    \label{chap2:rates-stats}
%\end{figure}

Based on analysis for the cases in $P_{\rm ref}$, I found the aqueous sulfate formation are most sensitive to the $\rm SO_2(g)$ and $\rm NH_3(g)$ mixing ratios. Cases with most sulfate formation were for the cases with high $\rm NH_3(g)$ values and this can be explained by the higher oxidation rates of $\rm SO_3^{2-}$ by $\rm O_3(aq)$. Cases with higher $\rm NH_3(g)$ but lower sulfate production were because of lower $\rm SO_2(g)$, and sulfate formation was $\rm SO_2(g)$-limited. I also concluded that our particle-resolved aqueous approach was capable of simulating the complex response of $\rm O_3$ and $\rm H_2O_2$ pathways to temperature and acidity changes. 

\section{Contribution of TMI pathways}
\label{chap2.5}
In this section, I will analyze the other 243 cases containing iron components in $P_{\rm TMI}$ and explore the contributions of TMI catalyzed reactions. Figure~\ref{chap2:iron-conc}(a) shows the time series of $\rm Fe^{2+}$, $\rm Fe^{3+}$ and OH(aq) mass concentration in the ensemble cases. $\rm Fe^{2+}$ varies between $10^{-11}$ and $10^{-7}$ and $\rm Fe^{3+}$ ranges between $10^{-7}$ and $10^{-6}$~\unit{M}, both are in the same order with the values in \citep{Deguillaume2005}. By adding TMI, the median sulfate mixing ratio in $P_{\rm TMI}$ significantly increased from 2~\unit{ppb} to 5~\unit{ppb}. Figure~\ref{chap2:iron-contri} further explores the contribution ratio changes of different aqueous sulfate pathways after adding TMI in the population. 

\begin{figure}[ht]
    \centering \includegraphics[scale=0.55]{chap2_figs/chap2_with_tmi_fixOH_mass.pdf}
    \caption{Time series of (a) $\rm Fe^{2+}$, $\rm Fe^{3+}$ and $\rm OH(aq)$ (b) Sulfate mixing ratio in $P_{\rm ref}$ and $P_{\rm TMI}$. Values are the median of all the ensemble cases in $P_{\rm TMI}$. }
    \label{chap2:iron-conc}
\end{figure}

Oxidation of $\rm HSO_3^-$ by $\rm O_3$ and $\rm H_2O_2$ are the dominant pathways for aqueous sulfate formation in $P_{\rm ref}$ (Fig.~\ref{chap2:iron-contri}(a)). These two pathways in total contributed to more than 90\% of sulfate formation along the simulation period, and there is minimum contribution of TMI pathways. But in $P_{\rm TMI}$, these two pathways are dominant at the first 2 minutes and sulfate formation through the reaction between $\rm SO_4^-$ and water is dominant afterwards (Fig.~\ref{chap2:iron-contri}(b)). I also noticed a remarkable contribution from TMI pathway after 4~\unit{min}, and the contribution ratio can be more than 20\% at the end of simulation. 

\begin{figure}[ht]
    \centering \includegraphics[scale=0.6]{chap2_figs/chap2-TMI_contri_factors.pdf}
    \caption{Mean contribution fraction of different sulfate formation pathways in (a)$P_{\rm ref}$ and (b)$P_{\rm TMI}$}
    \label{chap2:iron-contri}
\end{figure}

It is worth mention that gas OH emitted at a constant rate for the cases $P_{\rm TMI}$, while there is no emission in $P_{\rm ref}$. Based on current simulation setting, we can not distinguish the role of OH and TMI for the increased sulfate production rates in $P_{\rm TMI}$. Further simulations are needed to validate the current findings. But what we can still learn from current result is there exists some conditions where sulfate formation through TMI pathways can be important. 

\section{Conclusion}
\label{chap2.6}
In this work, I coupled particle-resolved PartMC-MOSAIC with a comprehensive aqueous chemistry 
module CAPRAM2.4 to investigate the role of different aqueous sulfate formation pathways. 
Model was evaluated by size-resolved and bulk aqueous models, and our model reproduced the similar $\rm SO_2$(g) and 
acidity profiles. 

I designed a reference scenario at various gas conditions to investigate the most efficient aqueous sulfate formation pathways.
I found cases with high $\rm NH_3(g)$ and $\rm SO_2(g)$ produced the most sulfate, and it can be explained by efficient oxidizing reactions between $\rm SO_3^{2-}$ and $\rm O_3(aq)$ because of the higher pH. Our particle-resolved aqueous model was proved to be capable of simulating the complicated responses of $\rm O_3$ and $\rm H_2O_2$ pathways to pH. Analysis of another ensemble scenario with TMI shows S(IV) reactions catalyzed by TMI can contribute up to 20\% of the sulfate production, and the ratio was consistent with the values found by \citet{alexander2009transition}. 

%%%Limitations%%%
Currently, I used monodisperse aerosol populations for analysis and in the future we should generalize our conclusions by more realistic aerosol populations, such as populations with several lognormal distribution. Also, I used rather simple way to represent TMI and OH(g) concentrations. Since Fe from different sources have different solubility \citep{desboeufs2005dissolution}, it would be better to connect Fe soluble fraction and OH(g) emitting rates with aerosol sources to confirm the conditions most favorable for TMI-catalyzed reactions. 

%%%%%%%%%%%%%%%%%%%%%%%%%%%%%
%%%%%%%%%Chapter 3%%%%%%%%%%%
%%%%%%%%%%%%%%%%%%%%%%%%%%%%%
\chapter{Evaluating the impacts of cloud processing on resuspended aerosol particles after cloud evaporation using a particle-resolved model}
\label{chap3}
\section{Introduction}
Atmospheric aerosol particles are complex mixtures of different
chemical species reflecting the fact that they originate from
different emission sources and experience various aging processes in
the atmosphere \citep{Riemer2009, Li2011a, Bondy2018, Healy2014,
  Rissler2014}.  Aging processes include processes in the cloud-free
atmosphere such as coagulation, heterogeneous reactions on the
particles' surface, and the formation of coatings from organic and
inorganic secondary aerosol. They also include processes in clouds
\citep{Lance2017} such as aqueous-phase chemistry within cloud droplets
forming inorganic and organic aerosol material, and
collision-coalescence of particles and droplets within a cloud. When
clouds evaporate, aerosol populations are released into the atmosphere
with modified properties compared to the populations that formed the
cloud. This, in turn, changes the aerosols' impacts on clouds in the
next cloud cycle \citep{Hoose2008}, and therefore this process is
important for 3D chemical transport models to include. At the same
time it poses challenges to be represented in chemical transport
models \citep{Gao2016}.

Specifically, in-cloud processes have been shown to produce a double
peaked size distributions since material from the gas phase and from
smaller particles is transferred to the accumulation mode size range
\citep{Hoppel1986,Noble2019}. It has also been observed that cloud
droplets of different sizes may differ in their acidity
\citep{Collett1994,Pye2020}. This has important implications for the
rates of aqueous-phase sulfate formation \citep{Hoag1999}, which
depends strongly on pH, and this needs to be considered when
representing these processes in cloud microphysics models
\citep{Hegg1990,Barth2006}. Because of the non-linearity of aqueous
chemistry processes, models predict larger rates of sulfate formation
when using a more realistic size-resolved droplet representation
compared to using a prescribed single droplet size.

In this study, we not only considered the variation of aerosol (and
cloud droplet) composition with size, but also the variation of
composition within a narrow size range, commonly referred to as mixing
state \citep{Winkler1973,Riemer2019}. Our goal in this study was to
quantify the changes in aerosol mixing state due to in-cloud 
aqueous-phase chemistry and coagulation processes. Aerosol mixing state
impacts the aerosols' effects on health \citep{Ching2018}, their
absorption and scattering of sunlight \citep{Lesins2002,Fierce2020},
and their ability to act as cloud condensation and ice nuclei
\citep{Broekhuizen2006,Bhattu2015,Knopf2018}.

Mixing state is, on the one hand, a factor in determining which
  particles activate and form cloud droplets, thereby determining
  cloud properties \citep{ching2012impacts,Ching2016}. On the other hand, mixing
  state can be modified by in-cloud processes. For example,
observations using online single-particle mass spectrometry during the
HCCT-2010 field campaign showed that cloud residuals contain more
sulfate and nitrate compared to the below-cloud aerosol
\citep{Roth2016}, resulting in a change of aerosol mixing state.

Model simulations of aerosol mixing state are challenging, and
particularly rare when it comes to simulating in-cloud processes due
to the need for extensive computing resources.  Regional or global
models use simplified assumptions about aerosol activation and aerosol
mass and size changes due to cloud processing, which are determined by
the underlying model representation of aerosol and cloud droplets. For
example, in the CMAQ model, which uses a modal aerosol representation,
the sulfate mass produced by in-cloud chemistry is added to the entire
accumulation mode \citep{Ervens2015, Fahey2017}. In the global
  climate model ECHAM5-HAM, activated particles are grouped into two
  bins, one with low ion concentration and the other with higher ion
  concentration \citep{roelofs2006aerosol} to capture the dependence of
  the sulfate formation pathway on pH.

High-resolution cloud models typically use a size-resolved fixed-bin
microphysical model \citep{Flossmann1994,Feingold1996} and resolve
aerosol particles and cloud droplets by separate distributions, thus
internally mixing all same-sized particles or droplets. Similarly,
accurate parcel models most frequently use a size-resolved moving-bin
approach \citep{Kreidenweis2003,Cooper1997}, again losing aerosol
history and composition information within each size bin. To preserve
some composition information, 2D aerosol models have been used,
resolving cloud droplet size and aerosol dry volume
\citep{Bott1996,Ovchinnikov2010}. However, increasing the dimension
beyond 2D to treat composition variation of the aerosol in more detail
would be computationally prohibitively expensive. Lagrangian cloud
microphysics models have also been developed that track
information on a droplet level \citep{Shima2009,
  Andrejczuk2008,Grabowski2019,Soelch2010,
  Unterstrasser2014,Jaruga2018}. However, their focus has been the
study of cloud microphysics rather than the modification of aerosol
composition. While \citet{Jaruga2018} consider some aqueous-phase
chemistry processes, their representation of the aerosol is
comparatively simple and questions about mixing state have not yet
been addressed.

For our study, we used the aerosol model PartMC-MOSAIC (Particle Monte
Carlo-Model for Simulating Aerosol Interactions and Chemistry)
\citep{Riemer2009,Zaveri2008} as aerosol and cloud parcel model. This
stochastic particle-resolved model explicitly resolves the composition
of individual aerosol particles and cloud droplets in a given
population. Since individual particles and droplets are explicitly
tracked, there is no need to invoke ad hoc aging criteria that move
aerosol mass between bins or modes as is the case with traditional
modal or sectional approaches \citep{Riemer2003,Stier2005,
  Bauer2008,Jacobson2001}. Tracking droplets explicitely is therefore
a well-suited to simulate aerosol mixing state and investigate its
impacts on climate-relevant aerosol properties.

The model was described in \citet{ching2012impacts} and \citet{Ching2016} and
has been used to simulate the mixing state evolution of
black-carbon-containing aerosol in the cloud-free atmosphere, followed
by a process analysis to what extent the aged aerosol is able to
undergo nucleation-scavenging as the particles compete for water vapor
in an updraft. However, these studies did not include the effects of
aqueous-phase chemistry ocurring within the cloud droplets. This is
the motivation for this study, where we extended our modeling framework
to include aqueous-phase chemistry within the cloud droplets that are
forming on a diverse population of particles, common to urban
environments. The contribution of this paper is the first study to
document quantitatively the impact of aqueous phase
chemistry on mixing state.

We focused on the following questions: (1) To what extent does cloud
processing change the aerosol mixing state of the population that
entered the cloud? (2) How does this change the cloud condensation
number concentration? (3) What is the role of coagulation between the
interstitial particles and cloud droplets for mixing state of the
aerosol?

Section~\ref{sec:model} describes the model components, the scenario
setup, and the mixing state metrics used in this
study. Section~\ref{sec:results} presents the analysis of the
simulation results. Section~\ref{sec:conclusions} summarizes our
results.
%

\section{Model description and metrics}
\label{sec:model}
\subsection{Stochastic particle-resolved module PartMC-MOSAIC}
Aerosol physical and chemical processes were simulated by the
stochastic particle-resolved model PartMC-MOSAIC (Particle Monte
Carlo-Model for Simulating Aerosol Interactions and Chemistry-Model
for Simulating Aerosol Interactions and Chemistry, \citep{Riemer2009,
  Zaveri2008}). The PartMC model simulates the evolution of
per-particle composition of a large ensemble of computational
particles in a well-mixed computational volume. In contrast to
Lagrangian droplet models that have become popular in the cloud
microphysics community \citep{Shima2009,Grabowski2019}, PartMC-MOSAIC 
does not track the position of particles and droplets within the computational
volume.

The particle number concentration changes due to coagulation, emission
and dilution, which are simulated by using a stochastic Monte Carlo
sampling method \citep{Riemer2009}. Gas-phase chemistry and
gas-particle partitioning are represented by the aerosol chemistry
model MOSAIC, which includes CBM-Z for gas-phase photochemical
reactions \citep{Zaveri1999}, MTEM for estimating mean activity
coefficient of an electrolyte in a inorganic multicomponent solution
\citep{Zaveri2005} and MESA for intraparticle solid-liquid
partitioning for inorganic aerosols \citep{Zaveri2005a}. The formation
mechansim of secondary organic aerosol (SOA) in MOSAIC is based on
SORGAM \citep{Schell2001} with several parameters adjusted to bring the
simulated values closer to observation \citep{Zaveri2010a}. The model
represents key aerosol species including sulfate, nitrate, ammonium,
black carbon (BC), primary organic aerosol (POA) and several surrogate
secondary organic aerosol (SOA) species. The coupled model
PartMC-MOSAIC was applied in previous studies for simulating aerosol
optical and CCN properties, black carbon aging time-scales and the
black carbon absorption enhancement due to the coatings
\citep{Zaveri2010a, Riemer2010, Fierce2017,Fierce2020}, focusing on
mixing state evolution during cloud-free conditions. The model was
also used for evaluating the impact of aerosol mixing state on cloud
droplet formation \citep{ching2012impacts,Ching2016}, which will be explained
in more detail in the next section.

\subsection{Cloud parcel model and aqueous-phase chemistry}
\citet{ching2012impacts} described the details of the particle-resolved cloud
parcel model, which simulates a population of aerosol particles that
experience cooling at a prescribed cooling rate and subsequent growth
due to the condensation of water vapor. The condensational growth of
the particles is calculated following \citet{Seinfeld2016}. The
driving force of the growth is the difference between droplet
equilibrium saturation vapor pressure and the ambient vapor pressure
of the environment. The equilibrium saturation vapor pressure is
calculated by K\"ohler theory, and the particle hygroscopicity is
determined using the parameterization of aerosol hygroscopicity
developed by \citet{Petters2007}. The hygroscopicity $\kappa$
  of each particle is calculated by using the volume-weighted average
  of the individual $\kappa$ values of the particles' constituent
  species. We used $\kappa$ of 0.65 for ammonium sulfate and ammonium nitrate,
  $\kappa$ of 0.1 for SOA and $\kappa$ of 0.001 for POA.  BC is
  assumed to be hyrophobic and had $\kappa$ of 0.  The hygroscopicity
  of each particle varied as the chemical composition evolved for each
  particle. We currently do not represent any entrainment of
cloud-free air into the cloud, surface tension effects on droplet
growth, or the loss of droplets from the air parcel owing to
sedimentation.

The aim of this paper is to investigate impacts of in-cloud
aqueous-phase chemistry on aerosol mixing state. To this end, we
coupled the reduced Chemical Aqueous Phase Radical Mechanism (CAPRAM)
2.4 to PartMC-MOSAIC. The reduced CAPRAM model includes 183 reactions
(including Henry's Law partitioning, dissociation reactions,
photolysis reactions and other aqueous-phase reactions) and 113
species \citep{Herrmann1999, ervens2003capram}. The mechanism
treats the reactions of common radicals and radical anions, transition
metal ions and organics with less than two carbon atoms. The CAPRAM
mechanism was applied to simulate the cloud processes for the FEBUKO
field campaign and reproduced the aqueous sulfate and organic
compounds oxidation processes well \citep{Tilgner2005, Wolke2005}.
While the aqueous-phase chemistry involving transition metal ions and
organic species is of great interest
\citep{Mayol-Bracero2002,Harris2013a, Alexander2009, Lian2019,
    McNeill2015a, Smith2014, wonaschuetz2012aerosol,wagner2015situ},
our scope for this initial study is the in-cloud production of
  sulfate and nitrate and their relationship to changes in aerosol
  mixing state. We did not consider the co-condensation of nitric acid
  gas due to vapor pressure difference discussed by
  \citet{crooks2018parameterisation}. A subset of the most relevant
Henry's law, aqueous equilibria and chemistry reactions are summarized
in Table~S1.

The original gas-phase chemistry mechanism Regional Atmospheric
Chemistry Modeling (RACM) used in CAPRAM 2.4 was replaced with CBM-Z,
which is the gas-phase mechanism native to PartMC-MOSAIC. In the
current setting, aqueous chemistry, and the evaporation and
condensation of gases (other than water vapor) to aqueous particles
are enabled for particles with liquid water mass larger than 5
$\times$ $\rm 10^{-16}$ kg, which corresponds to solution droplets of
1 $\rm \mu m$ in diameter.

We used the CVODE \citep{Cohen1996} solver of the SUNDIALS
\citep{Hindmarsh2005} package to solve the mass transfer and aqueous
chemistry of the CAPRAM 2.4 reduced mechanism with the Backward
Differentiation Formulas (BDF) and Newton Iteration, which is suitable
for mathematically stiff systems, such as those treating multi-phase
chemistry.  To reduce the stiffness of the system, the Henry's Law
partitioning of the strong acids $\rm H_2SO_4$, HCl, and $\rm HNO_3$
were combined with their first acid dissociation step.

\subsection{The scenario settings}
The scenario setting of this work followed the two-step method
  used in \citet{ching2012impacts}. Step~1 represented the simulation of an
  ``urban plume scenario'' in a subsaturated (RH $<$ 100\%)
  environment using PartMC-MOSAIC. The purpose of this step was to
  generate aerosol populations that cover a wide range of mixing
  states, which can later serve as input populations for our cloud
  parcel simulations. The urban plume simulation had a simulation time
  of 24~h, and the state of the aerosol and gas phase were saved
  hourly. This hourly output was then used as inputs for 25 separate
  30-min particle-resolved cloud parcel simulations (Step 2). Each of
  the 25 cloud parcel simulations was exposed to the same cooling rate
  but differed in their initial conditions of aerosol populations and
  gas-phase concentrations.

The urban plume case environment shown here was adopted from
\citep{Zaveri2010a}, following a Lagrangian box modeling approach,
where we assumed that the air parcel containing background air moved
over a polluted urban environment. We refer to this case for the
remainder of the paper as the ``high-emission'' case. The initial
condition of the aerosol consisted of two lognormal modes, with the
number concentration, geometric mean diameter, standard deviation and
composition for each mode as listed in Table~\ref{tab:emi}. We used
10,000 computational particles to resolve the initial aerosol. Note
that the number of computational particles changed over the course of
the simulation depending on coagulation, particle emissions, and
dilution with the background, but was kept between half and double the
initial number of computational particles using doubling and halving
procedures as described in \citet{Riemer2009}.

\begin{table}[H]
	\centering
	\begin{threeparttable} 
		\caption{Size distributions and compositions of initial,
			background and emitted aerosols for the high-emission case.}
		\vspace*{-5mm}
		\begin{tabular}{c c c c c c}
			\toprule
			Initial/Background & $N$ ($\rm cm^{-3}$)& $D_{\rm g}$ ($\mu{\rm m}$) & $\sigma_{\rm g}$ &  Composition by mass & $D_i^{**}$ \\
			\midrule
			Aitken mode &  1800 & 0.02 & 1.45 & 49.6$\%$ $\rm (NH_4)_2SO_4$ + 49.6$\%$ $\rm API1^*$ + 0.8$\%$ BC & 2.08\\ 
			Accumulation mode & 1500 & 0.116 & 1.65 & 49.6$\%$ $\rm (NH_4)_2SO_4$ + 49.6$\%$ API1 + 0.8$\%$ BC & 2.08 \\
			\midrule
			Emission &  $E$ ($\rm m^{-2} s^{-1}$)& $D_{\rm g}$ ($\mu{\rm m}$) & $\sigma_{\rm g}$  &  Composition by mass &  $D_i$ \\
			\midrule
			Cooking &  9$\times 10^6$ & 0.086 & 1.91 & 100$\%$ POA & 1\\
			Diesel & 1.6$\times 10^8$ & 0.05 & 1.74 & 70$\%$ BC + 30$\%$ POA & 1.84 \\
			Gasoline & 5$\times 10^7$ & 0.05 & 1.74 & 20$\%$ BC + 80$\%$ POA & 1.65\\
			\bottomrule
			\label{tab:emi}
		\end{tabular}
		\vspace*{-5mm}
		\begin{tablenotes}[para,flushleft]
			\small
			\item $*$: Low volatility secondary aerosol product from
				the oxidation of $\alpha$-pinene. \\
			\item $**$: Per-particle diversity, refer to
				Sec.~\ref{sec:mixing_state_metrics} for a detailed
				description.
		\end{tablenotes}
	\end{threeparttable}
\end{table}

Gas-phase initial conditions were set to 50 pbb ozone and low levels
of other trace gases. The plume was diluted with background air at a
rate of 1.5$\times 10^{-5}$ $\rm s^{-1}$ that contained the same gas
mixing ratios and aerosol concentrations as the initial condition. The
simulation started at 6~AM local time and lasted for 24~h with gas and
aerosol emission entering the simulation during the first 12~h. We use
the term ``plume time'', $t_{\rm u}$, to refer to the elapsed time
during this 24-h simulation in cloud-free conditions. The temperature
was prescribed as shown in Figure~\ref{fig:env}. For simplicity we
assumed that the temperature remained constant after the first
6~h. This is consistent with the air parcel staying in the fully
mature mixed layer until sunset and in the residual layer thereafter
\citep{Zaveri2010a}. The resulting relative humidity varied
between 52$\%$ and 95$\%$, assuming that the total water content in
the air parcel was constant.

\begin{figure}[H]
	\centering \includegraphics[width=\textwidth]{chap3_figs/fig1.pdf}
	\caption{Temperature and relativity humidity time series in (a)
		urban plume environment and (b) cloud parcel environment. The
		green dashed line in (b) is RH = 100$\%$.}
	\label{fig:env}
\end{figure}

Aerosol emission sources and their compositions are also listed in
Table~\ref{tab:emi}. Note that the composition of aerosol
  emissions from gasoline vehicles was based on
  \citet{somers2004mobile} and \citet{nam2008analysis}. Current
  gasoline vehicles produce tailpipe emissions with higher BC content
  \citep{liggio2012emissions}. However, since both BC and POA have very
  low hygroscopicities ($\kappa=0$ for BC and 0.001 for POA), we do
  not expect that the exact BC/POA split impacted our results for this
  paper (however this will be important for optical properties).

The mixing state of the aerosol in this simulation evolved because
primary aerosol emissions aged owing to formation of secondary aerosol
and to coagulation processes, while fresh emissions continued to enter
during the first 12 hours of simulation. Overall, this scenario
mimicked the evolution of an air parcel in a polluted urban area, and
we summarized the results, including the mixing state evolution, in
Section~\ref{sec:urban_plume}. The full state, including gas-phase
mixing ratios and composition of all computational particles, was
saved hourly to be used as input for the cloud parcel simulations in
step 2.

In addition to the high-emission urban plume case described
  above, we performed a second simulation with reduced initial aerosol
  number concentration and aerosol emissions (5\% of the high-emission
  case) and gas emissions (15\% of the high-emission case) to
  represent a less polluted urban environment. We refer to this case
  as the ``low-emission case''.  Scale parameters relative to the
  high-emission case are summarized in Table~\ref{tab:low-emi}.


\begin{table}[H]
	\centering
		\caption{Scale parameters of gas emission rates, initial and background particle number concentration and aerosol number fluxes emission rates used for low-emission cases (relative to high-emission case)}
	\begin{tabular}{lcccccl}
		\toprule 
		 &  & \multicolumn{2}{c} {Initial/background number conc.} & \multicolumn{3}{c}{Aerosol emission rates} \\
		 \cmidrule(r){3-4}\cmidrule(r){5-7}
		 & Gas emission rates  & Aitken  & Accumulation & Cooking & Diesel & Gasoline  \\
		\midrule
		Scale parameters & 15\% & 5\% & 5\%  & 5\% & 5\% & 5\% \\
		\bottomrule
	\end{tabular}
        \label{tab:low-emi}
\end{table}

For step~2, the hourly output of the simulated aerosol and gas-phase
mixing ratios was used as input for the cloud parcel simulations,
using a prescribed cooling rate. The temperature decreased for
  the first 10 minutes at a rate of $0.25$~$\rm K \, min^{-1}$, which
  corresponds to an updraft velocity of approximately 1~$\rm m \,
  s^{-1}$. The value is calculated by using equation (17.56) in
  \citet{Seinfeld2006a} without consideration of entrainment. The
cooling rate was kept constant for the next 10 minutes, and increased
at the rate of 0.25~$\rm K \, min^{-1}$ for the last 10 minutes. Hence
one cloud cycle consisted of a total of 30 minutes, and we referred to
the elapsed time within the cloud cycle as ``cloud parcel time'',
$t_{\rm c}$. The initial RH for the cloud parcel was 99~$\%$, and it
reached supersaturation within less than 1~min. The parcel became
subsaturated when the cloud began to evaporate at 20 min, and returned
to RH=99$\%$ at the end of the simulation.

Since it is common for air parcels to undergo several cloud cycles
\citep{Barth2003}, we conducted a total of four cloud cycles for
  the high-emission case, which resulted in a total of $25 \times 4 =
100$ cloud parcel simulations. We initialized the aerosol for the
second, third and fourth cloud cycles using the particle population
from the end of the previous cloud cycle. For the gas-phase
  mixing ratios we always used the values from the beginning of the
  first cloud cycle. That is, in choosing this setup, we explored how
  an already processed aerosol population changes further when it is
  exposed to a given gas environment and cooling rate (especially
  focusing on changes in the mixing state) with the only difference
  between cloud cycles being the input aerosol population. We realize
  that this is an idealized setup since in reality the gas phase
  concentrations are likely not the same for subsequent cloud
  cycles. For the low-emission case, we only present the results for
  one cloud cycle.

Lastly, to explore the effects of coagulation, we performed another
set of cloud parcel simulations that included Brownian coagulation,
using the high-emission case as input. For most of our
analysis, we focus on the difference between the particles at the
start of the cloud parcel simulations and the end of the cloud parcel
simulation, after cloud evaporation.

\subsection{Mixing state metrics}
\label{sec:mixing_state_metrics}
The objective of this paper is to quantify the change of particle
mixing state as a result of cloud processing. The metrics used to
quantify mixing state were developed by \citet{Riemer2013a}. The
mixing state metric $\chi$ is calculated by:
\begin{equation} \label{eq:chi}
    \chi = \frac{{D}_{\alpha}-1}{{D}_{\gamma}-1},
\end{equation}
where $D_{\alpha}$ is the average particle diversity and
$D_{\gamma}$ is the bulk particle diversity.

The calculation of these diversity metrics is based on the
per-particle mixing entropy $H_i$. For an aerosol population of $N$
particles containing $A$ species, the mixing entropy $H_i$ and
particle diversity $D_i$ of particle $i$ are calculated as
\begin{equation}\label{eq:H-i}
    H_i = \sum_{a=1}^{A}-p_i^a {\rm ln}p_i^a  \;\;\;D_i = e^{H_i}, 
\end{equation}
where $p_i^a$ is the mass fraction of species $a$ in particle
$i$. Expanding $D_i$ to the whole population, $D_\alpha$ and
$D_{\gamma}$ are defined as
%  \label{alpha}
  \begin{align}
    H_\alpha &= \sum_{i=1}^N p_i H_i\;\;\;&D_\alpha= e^{H_\alpha}, \label{eq:H-alpha}\\
    H_\gamma &= \sum_{a=1}^A -p_a H_i\;\;\;&D_\gamma= e^{H_\gamma}, \label{eq:H-gamma}
  \end{align}
where $p_i$ and $p_a$ are the mass fractions of particle $i$ and
species $a$ in the population. For externally mixed populations where
particles contain only one species, $D_\alpha =1$ and $\chi =0\%$. For
internally mixed population where each particle has the same
composition as the bulk, $D_\alpha = D_\gamma $ and $\chi =100\%$. In
the ambient atmosphere, aerosols are neither completely internally nor
externally mixed and intermediate mixing states are common
\citep{Healy2014,Ye2018,Ching2019}. For regions close to emission
sources, $\chi$ is expected to be lower, while $\chi$ is larger in air
masses dominated by an aged aerosol. 

The mixing state metrics $\chi$ defined in this paper used the
abundance of model chemical species as the basis for calculating
particle mass fractions in
Equations~(\ref{eq:H-i})--(\ref{eq:H-gamma}), i.e. sulfate, nitrate,
ammonium, POA, etc., excluding aerosol water. Other choices for
defining ``species'' are possible, for example \citet{Ching2017} used
two surrogate species, hygroscopic and non-hygroscopic species, as the
basis for $\chi$. Furthermore, \citet{Zheng2021} compared $\chi$ based
on the mixing of model chemical species, of hygroscopic and
non-hygroscopic species, and of absorbing and non-absorbing species.

\section{Simulation results}
\label{sec:results}
\subsection{Urban plume simulation with PartMC-MOSAIC}
\label{sec:urban_plume}

This section summarizes the results from the urban plume simulations
to provide context for the cloud parcel simulations discussed in the
remainder of the paper. Figure~\ref{fig:urban_plume} shows selected
quantities from the high-emission urban plume simulation. The
total particle number concentration $N_{\rm a}$ increased initially
due to the emission of primary particles, reached a maximum of 15,295
$\rm cm^{-3}$ at $t_{\rm p} = 12$~h, then decreased because the
emissions ceased, and both dilution and coagulation reduced the
particle number concentration. Similarly, BC and POA mass
concentrations increased for the first 12~h due to emission, and
decreased thereafter due to dilution with the background, just as
Figure~\ref{fig:urban_plume}b shows. The time series of the secondary
aerosol species sulfate and SOA were determined by the interplay of
loss by dilution and photochemical production. The ammonium nitrate
mass concentration was determined by the gas concentrations of its
precursors, $\rm HNO_3$ and $\rm NH_3$, temperature and RH. Mixing
ratios of $\rm SO_2$, $\rm O_3$ and $\rm H_2O_2$ are shown in
Figure~\ref{fig:urban_plume}c for reference because they are directly
involved in the in-cloud sulfate formation as discussed in
Section~\ref{sec:cloud}.

Figure~\ref{fig:urban_plume}b only displays the bulk composition of
the aerosol, while the mixing state information available from the
particle-resolved output remains hidden. Figure~\ref{fig:urban_plume}d
provides insight into the evolution of aerosol mixing state as
quantified by the mixing state metrics introduced in
Section~\ref{sec:mixing_state_metrics}. At $t_{\rm u} = 0$, the
particle population was completely internally mixed, and therefore the
mixing state index $\chi$ was initially 100\%.

From Equation~\ref{eq:chi}, we recall that $\chi$ is determined by the
ratio of $D_{\alpha}$ and $D_{\gamma}$. Figure~\ref{fig:urban_plume}d
indicates that both $D_{\alpha}$ and $D_{\gamma}$ started out low,
which is consistent with the aerosol initially only containing a small
number of species, both on a per-particle level and on a population
level, see Table~\ref{tab:emi}. Over the course of the simulation,
both $D_{\alpha}$ and $D_{\gamma}$ increased, but at different rates,
which led to changes in $\chi$ that we can interpret as changes in
mixing state. The initial decrease in $\chi$ to about 50\% was caused
by the emission of fresh combustion particles, containing BC and
POA. These emissions continued for the first 12~h of simulation, but
at the same time coagulation and secondary aerosol formation occurred,
which (at least initially) efficiently increased the average
per-particle diversity $D_{\alpha}$. Overall, this
led to a more internally mixed population, with $\chi$ increasing to
72\% at 10~h. After this, dilution became relatively more important,
introducing background particles, and ammonium nitrate evaporated
almost entirely towards the end of the simulation. These combined
processes resulted in a slow decrease in $\chi$ to 64\% at the end of
simulation.

\begin{figure}
    \centering
    \includegraphics[scale=0.50]{chap3_figs/fig2.pdf}
    \caption{Temporal variation of (a) total number concentration, (b)
      mass concentrations of selected aerosol species, (c) mixing
      ratios of selected gas-phase species and (d) aerosol mixing
      state metrics for the high-emission case. }
    \label{fig:urban_plume}
\end{figure}

The corresponding figure for the low-emission case is shown as
  Figure~S1 in the Supporting Information. As expected, for this case,
  the aerosol number and mass concentrations and the gas-phase
  concentrations were reduced compared to the high-emission case. For
  example, the maximum aerosol number concentration at $t=12$~h only
  reached to about 1000~$\rm cm^{-3}$ and the maximum $\rm SO_2$
  mixing ratio was only 1.2~ppb. The mixing state metrics (Figure~S1d)
  ranged between 40\% and 78\%.

\subsection{Aerosol composition changes during cloud processing}
\label{sec:cloud}
As described in Section~\ref{sec:urban_plume}, for each hourly output
from the urban plume simulations, cloud cycles were simulated using
the same temperature profile, shown in Figure~\ref{fig:env}b.
Each cloud parcel simulation was exposed to the same cooling
  rate but differed in their initial conditions of aerosol populations
  and gas-phase concentrations. Figure~\ref{fig:cases-gases} compiles
  the initial conditions of selected gas-phase species for the cloud
  parcel simulations using the high-emission case. Each color marks a
  specific cloud parcel case, which corresponds to a specific plume
  time $t_{\rm u}$. The ranges were consistent with conditions found
  in polluted urban regions, such as Haikou in southern China, with
  annual average mixing ratios of 2 ppb $\rm SO_2$ and 7 ppb $\rm
  NO_2$ \citep{wang2014spatial}, and Kanpur in Northern India, with
  annual average mixing ratios of 3 ppb $\rm SO_2$ and 5.7 ppb $\rm
  NO_2$ \citep{gaur2014four}. In this section, we first illustrate
the compositional changes during cloud processing, using the aerosol
population from the high-emission case at $t_{\rm u} =12$~h
as the initial conditions for the cloud parcel simulation, and focus
on the first cloud cycle ($N_{\rm cycle}=1$).

 \begin{figure}
    \centering
    \includegraphics[scale=0.6]{chap3_figs/fig3.pdf}
    \caption{Initial conditions for selected gas-phase species
        for the cloud parcel simulations (high-emission case). Colors
        indicate the different plume hours. Note that the mixing ratio
        for $\rm O_3$ is multiplied by 0.1 to be able to use the same
        scale on the ordinate as for the other gases.}
    \label{fig:cases-gases}
\end{figure}

Figure ~\ref{fig:cloud_env}a shows the evolution of several key
variables for this case. The initial RH for each cloud parcel
  simulation was 99\%, and the aerosol water content of each particle
  was adjusted for RH=99\% based on the dry aerosol composition. The
  parcel reached supersaturation within 1~min. During the
first 10~min the liquid water content increased and reached a maximum
of 1.23 $\rm g \, kg^{-1}$ at 10 min. We determined the cloud droplet
number concentration following the strategy used in \citet{ching2012impacts},
where particles with wet diameter larger than 2 $\rm \mu m$ were
classified as cloud droplets. As shown in the figure, the cloud
droplet number concentration (CDNC) was 2011 $\rm cm^{-3}$ at the time
when the maximum supersaturation was reached. This number
concentration decreased somewhat as the relative humidity slowly
relaxed to saturation, which can be explained by the so-called
``inertial effect'' \citep{Chuang1997, Nenes2001}. This refers to
droplets with diameter larger than 2 $\rm \mu m$ that were not truly
activated, i.e., they had a critical diameter larger than 2 $\rm \mu
m$ and this critical diameter was not reached during the simulation
time.  After 20~min, as the RH dropped below 100\%, the cloud droplet
number concentration declined faster and the cloud evaporated.

The CDNC is comparatively high for the example shown in
  Figure~\ref{fig:cloud_env}, since we started out with a large
  aerosol number concentration of over 15,000~$\rm cm^{-3}$. CDNC of
  this magnitude were observed, for example, during the IMPACT field
  campaign \citep{brenguier2011cloud}.  Using aerosols from other plume
  hours as inputs yielded CDNC below 2000 $\rm cm^{-3}$, as shown in
  Figure~S5 (but always larger than 1000~$\rm cm^{-3}$). For the
  low-emission case, the CDNC were much lower, between 100 and 640 $\rm
  cm^{-3}$, consistent with observations in more moderately polluted
  urban environments \citep{ahmad2013long}.

\begin{figure}
    \centering
    \includegraphics[scale=0.50]{chap3_figs/fig4.pdf}
    \caption{The evolution of (a) liquid water content (LWC) and cloud
      droplet number concentration (b) mixing ratios of key gas-phase
      species (c) key aqueous-phase species and (d) number
      concentration with respect to wet diameter. Results are for the
      aerosol population at $t_{\rm u} =12$~h (high-emission
        case) and for $N_{\rm cycle}=1$.}
    \label{fig:cloud_env}
\end{figure}

Figure~\ref{fig:cloud_env}b and Figure~\ref{fig:cloud_env}c show the
evolution of several key gas and aqueous-phase species. Ammonia in the
gas-phase dissolved and immediately formed ammonium. Aqueous-phase
$\rm NH_4^+$ increased from 4.7 to 9.8~$\rm \mu g\;m^{-3}$. Nitrate
increased rapidly from 9.3 to 37.4~$\rm \mu g\;m^{-3}$ at the
beginning due to the uptake of $\rm HNO_3$. These processes are
explained by the R1-R10 reactions in Table~S1. After
this, nitrate was further formed through reaction R15.

The dissolved sulfur dioxide formed $\rm SO_3^{2-}$ and $\rm HSO_3^-$,
and could be oxidized to sulfate by aqueous-phase $\rm H_2O_2$ or $\rm
O_3$ through R11-R15. The sulfate aqueous formation rates are highly
pH-dependent, and the $\rm H_2O_2$ oxidation reaction R13 is dominant
for pH lower than 5, while the $\rm O_3$ pathway R12 becomes more
important for pH higher than 6 \citep{Seinfeld2016, Shao2019}. For the
cases shown here, the cloud droplets were acidic, and therefore the
oxidation by $\rm H_2O_2$ dominated. Sulfur dioxide in the gas-phase
decreased from 4.98 to 1.94~ppb, and the S(VI), including $\rm
SO_4^{2-}$ and $\rm HSO_4^-$, increased from 5.16 to 20.23~$\rm \mu
g\;m^{-3}$ during the simulation period.

The evolution of the number size distributions based on wet diameter
is illustrated in Figure~\ref{fig:cloud_env}d. At $t_{\rm c} =
0$, the size distribution peaked at 0.3~$\rm \mu m$.  As discussed
above, in less than 1~min, a subset of the particles activated to form
cloud droplets and the particle size distribution evolved from
initially unimodal to bimodal, with the first peak representing the
interstitial (not activated) particles and the second peak
representing the cloud droplets. The interstitial aerosol remained
unchanged since this simulation did not include coagulation. We will
investigate the impact of coagulation further in
Section~\ref{sec:coag}. With increasing liquid water content and
aqueous chemistry processes occurring, the cloud droplets continued to
grow and the droplet mode peaked at 11 $\rm \mu m$ at 20~min.

Next, we will turn to the changes in aerosol size distributions. To
provide a summary of the 25 individual cloud parcel simulations, we
show here the average over all 25 scenarios with the variability
amongst cases indicated by the standard deviation (colored band).
Figure~\ref{fig:size_number} shows the number and mass concentration
as a function of dry diameter before entering the cloud and after each
cloud cycle for the high-emission case. After the first cloud
cycle, a second mode appeared for the dry number distribution,
transforming the unimodal number distribution that peaked at a dry
diameter of 0.1~$\rm \mu m$ to a bimodal distribution with a second
peak at 0.3~$\rm \mu m$.  For each additional cloud cycle, the peak of
the second mode kept moving to larger diameters and reached 0.5~$\rm
\mu m$ after the fourth cloud cycle.

The number size distribution of the cloud droplet residuals
  peaked at a dry diameter of 0.3~$\rm \mu m$ after the first cloud
  cycle, which is consistent with field observations
  \citep{fast2019impact, ditas2012aerosols, ge2012effect}. For
  subsequent cloud cycles, the diameters increased further, because
  each cycle added more secondary aerosol mass on the already
  cloud-processed particles. This may not be representative for real
  clouds, where particle growth may be limited by gas precursors or
  oxidants and where entrainment occurs. However, cloud droplet
  residuals as large as 1~$\rm \mu m$ aerodynamic diameter were
  observed for particles collected at a mountain site in Southern
  China \citep{lin2017situ}.

In order to quantify the diameter changes, we define the mean diameter
$\bar{D}$ as
\begin{equation}
	\bar{D}=\frac{\sum_{i=1}^{n} N_i D_i}{N_{\rm total}},
\end{equation}
where $i$ is the particle index, $n$ is the total number of computational
particles, $N_{\rm total}$ is the total number concentration, $N_i$
and $D_i$ are the number concentration and diameter of computation
particle $i$. The increase in diameter was largest for the first
cycle, where $\bar{D}$ increased 24$\%$ and grew from 0.1 to
0.124~$\rm \mu m$. The fourth cycle only led to a 5\% mean diameter
increase. For the mass, as shown in Figure~\ref{fig:size_number}b,
distributions are dominated by the larger particles, as
expected. Similar to the trend seen in the number size distribution,
the mass increased most in the first cycle. The dry mass
  concentration, averaged over all 25 cases, increased from 21.2 to
  67.2~$\rm \mu g$ $\rm cm^{-3}$ in the first cycle, which corresponds
  to 217$\%$. After the fourth cycle, the mass increased from 110.4 to
  152.8~$\rm \mu g$ $\rm cm^{-3}$ (38\%).

\begin{figure}
    \centering
    \includegraphics[scale=0.45]{chap3_figs/fig5.pdf}
    \caption{(a) Number and (b) mass size distribution with respect to
      dry diameter before the first cloud cycle and after each
      subsequent cloud cycle (high-emission case). The solid
      lines are the averaged distributions of all 25 cases for each
      cloud cycle. The shaded area represents the 1$\sigma$ region of
      the average. The grey line is the distribution at $t_{\rm c}= 0$
      min, and the other lines are the distributions at the end of
      each cloud cycle. The composition of the particles was evaluated
      at RH = 99\%. The brown shading in (a) is due to the
        overlapping colors of different cycles.}
    \label{fig:size_number}
\end{figure}

The size distributions for the low-emission case are shown in
  Figure~S2. The average mode diameter of the cloud droplet mode after
  the first cycle is somewhat larger ($\sim0.35$~$\rm \mu m$) than in the
  high-emission case even though the gas-phase precursor
  concentrations were lower. This can be explained by the fact that
  the CDNC was substantially lower than in the high-emission case, and
  the gas-phase concentrations were still high enough to allow more
  secondary aerosol mass to form per droplet.

Figure~\ref{fig:size_mass} shows the size-resolved mass fractions
averaged over the 25 cases for each cloud simulation at the beginning
of the first cloud cycle (grey) and at the end of the fourth cloud
cycle (RH=99\%) in high-emission cases.  As expected, no change in the
size-resolved composition occurred for particles smaller than about
$0.1$~$\rm \mu m$, as these particles remained interstitial
aerosol. For the activated particles, the sulfate mass fraction
increased from $9.6\%$ to $38.7\%$ in the size range of
0.14--0.25~$\rm \mu m$, and the nitrate mass fraction increased from
$2.8\%$ to $56.8\%$ in the size range of 0.21--0.89~$\rm \mu
m$. \citet{ge2012effect} also found that particles were more
  enriched with nitrate and sulfate after aqueous processing from 
  samples collected from Central Valley of California. For the
particles in size range of 0.1--0.2~$\rm \mu m$, the fraction of
inorganic and SOA species decreased and the fraction of POA and BC
species increased after cloud processing. This can be explained by the
fact that there were two groups of particles in this size range, one
group with mainly inorganic species and SOA, and the other group with
mainly BC and POA. Particles in the first group were activated and
grew larger due to cloud processing. Because more ammonium sulfate and
nitrate was produced than SOA, the fraction of inorganics increased
more than the SOA fraction. As a result, in the size range of
0.7--0.9~$\rm \mu m$, particles transferred from organics-dominant to
inorganics-dominant. The reduced SOA mass fraction in this size
  range can also be explained by the simplified aqueous SOA formation
  mechanism in the current study, which for example does not include
  the formation of IEPOX-derived organosulfates that have shown to be
  important in the ambient atmosphere \citep{hatch2011measurements}.
In summary, even in the same size ranges (here 0.1--0.2~$\rm \mu m$),
particles with different compositions can evolve differently. This is
challenging to resolve for models that represent composition with
one-dimensional distributions, assuming internally mixed particles
within one size bin, and further work is needed to investigate
  the bias that may be introduced by this assumption.

For our current work, the cloud simulations were set up using the
temperature profile shown in figure~\ref{fig:env}b. Specifically, we
considered a cloud that was maintained for about 30~min. In the real
environment, clouds may last from minutes to days, depending on cloud
type and the surrounding environment \citep{Cotton2011}, and they may
also experience a range of different updraft velocities. Since the
largest rate of secondary mass formation occured within a few minutes
after the cloud formed, shortening the cloud parcel time did not
impact the conclusions as long as the first few minutes were
captured. With longer cloud lifetime, secondary aerosol mass formation
may continue, provided that the reactants are not depleted. Variations
in updraft velocities/cooling rates will result in different maximum
supersaturations, and hence differences in the subpopulations of
activated particles. We did not explore these sensitivities here to
keep the scope focused on the changes of particle populations after
typical but complete cloud processes.

\begin{figure}
    \centering
    \includegraphics[scale=0.60]{chap3_figs/fig6.pdf}
    \caption{Size-resolved mass fractions of (a) S(VI), (b) nitrate,
      (c) ammonium, (d) POA, (e) BC and (f) SOA at $t_{\rm c}=0$~min
      of the first cloud cycle (grey lines), and $t_{\rm c}=30$~min
      after the fourth cloud cycle (colored lines) for the
        high-emission case.  Solid lines are the average values of
      all 25 cases for each cycle and the shaded band represents the
      range of one standard deviation. The composition was
        evaluated at RH = 99\%.}
    \label{fig:size_mass}
\end{figure}

The previous analyses showed the size-resolved state of the
aerosol. However, even within a certain size range, particles can
exhibit differences in composition, and we introduced this earlier as
the aerosol mixing state. With our particle-resolved modeling approach,
we are able to resolve this detail and quantify how aerosol mixing
state changes with cloud processing. In order to visualize the change
in aerosol mixing state, Figure~\ref{fig:su_2d} displays the
two-dimensional size distribution of sulfate mass fraction $n(D_{\rm
  dry}, w_{\rm SO_4})$ for the example of two plume hours $t_{\rm u} =
0$ and $t_{\rm u} = 12$~h, before entering the cloud simulation and
after four cloud cycles. At the beginning of the urban plume
simulation ($t_{\rm u} = 0$~min), all particles had the same
composition, and $n(D_{\rm dry}, w_{\rm SO_4})$ was 36\% across the
entire population (Figure~\ref{fig:su_2d}a). Over the course of the
urban plume simulation, $n(D_{\rm dry}, w_{\rm SO_4})$ was controlled
by condensation, coagulation and dilution processes, and it became
more diverse as shown in Figure~\ref{fig:su_2d}b. The aerosol primary
emissions consisted of BC- and POA-containing particles (from
gasoline, diesel and cooking emissions), which over time became coated
with sulfate (as well as nitrate and SOA), resulting in particles with
sulfate mass fraction of about 20\% or less.  In
Figure~\ref{fig:su_2d}b, particles in the size range of 0.01--0.2~$\rm
\mu m$ appeared with a sulfate mass fraction of 70\%. These are
background particles that were introduced into the simulation by
dilution with their initial SOA content (model species API1) having
evaporated. The low particle number concentrations in between these
main particle types originate from coagulation events.

Figure~\ref{fig:su_2d}c and Figure~\ref{fig:su_2d}d show the
distributions after four cloud cycles. Note that the initial
concentrations of the gas-phase species between the two plume hours
differed as shown in Figure~\ref{fig:urban_plume}c. Using the
population at $t_{\rm u} = 0$~h, all particles started with the same
composition, and particles with diameters larger than 0.1~$\rm \mu m$
underwent cloud processing, resulting in sulfate mass fractions
between 30 and 60\%. Since we started with particles that were all
identical and after cloud processing, the particles differ in sulfate
mass fractions (and other secondary species), the aerosol population
became more diverse and more externally mixed. The gap in the
  0.1--0.3~$\rm \mu m$ diameter range was caused by the growth of the
  activated particles while the non-activated particles remained
  unchanged. Using the population at $t_{\rm u} = 12$~h, we can still
see the signature of the particles that underwent cloud processing,
however, it is difficult to infer if the population became more
internally or externally mixed. This is where calculating mixing state
metrics will help, which we will investigate next.

\begin{figure}
    \centering
    \includegraphics[scale=0.45]{chap3_figs/fig7.pdf}
    \caption{Two-dimensional number concentration distribution
      $n(D_{\rm p}, w_{\rm SO4})$ before cloud processing for (a)
      $t_{\rm u} = 0$~h and (b) $t_{\rm u} = 12$~h
      (high-emission case), (c) population from (a) after four
      cloud cycles, (d) population from (b) after four cloud cycles.}
    \label{fig:su_2d}
\end{figure}


\subsection{Impacts on mixing state and cloud condensation nuclei concentration}
As shown qualitatively in Figure~\ref{fig:su_2d}, cloud processes can
change the diversity and the mixing state of particle populations. In
this section we will quantify these changes for our cases more
precisely using the metrics described in
Section~\ref{sec:mixing_state_metrics}. Figure~\ref{fig:bulk_chi}
shows the evolution of $D_\alpha$, $D_\gamma$ and $\chi$ after each of
the four cloud cycles. After the first cloud cycle, the average
particle species diversity $D_\alpha$ increased for aerosols that used
plume hours 0 to 6 as inputs for the cloud parcel simulations. These
populations started with relatively low average particle
diversities. In contrast, $D_\alpha$ decreased for aerosols that used
plume hours 7 to 24 as inputs for the cloud parcel simulations. This
illustrates the fact that the addition of aerosol mass (mainly sulfate
and nitrate) to a subset of particles in a population can lead to a
decrease or increase in average particle diversity, depending on what
the starting point is. When $D_{\alpha}$ was initially low, then
adding secondary species led to more diverse particles, and
$D_{\alpha}$ increased.  This was the case when the aerosol consisted
of different types of freshly emitted aerosol and the particles each
only contained few species in the early stages of the urban plume
simulation. When $D_{\alpha}$ was initially high, adding a small
number of secondary species decreased the diversity, since those newly
added species dominated. This was the case when the aerosol consisted
of aged particles where several species commonly existed within one
particle. This argument applies to both $D_{\alpha}$ and
$D_{\gamma}$. Note that these two cases were contrasted in
\citet{Riemer2013a} as ``Prototypical cases 5 and 6''.  Comparing the
different cloud cycles, we observed that each cloud cycle led to a
less diverse population than the previous cloud cycle.

If $D_{\alpha}$ and $D_{\gamma}$ changed at the same rate, then $\chi$
would remain unchanged by cloud processing. However, here $D_{\alpha}$
generally decreased less than $D_{\gamma}$, and therefore $\chi$
increased for each cloud parcel simulation. The freshly emitted
particles experienced the largest changes, especially for urban plume
particle populations at $t_{\rm u}$ = 2~h, with $\chi$ increasing from
50\% to 83\% after four cycles. One exception was the case using the
aerosol at $t_{\rm u} = 0$~h as input, which was 100\% internally
mixed. The first cloud cycle therefore led to a more external
  mixture (but subsequent cloud cycles led to more internal mixtures),
  however this case is likely not relevant for real atmospheric
  conditions as a completely internal mixture is an idealized
  assumption.

\begin{figure}
    \centering
    \includegraphics[scale=0.43]{chap3_figs/fig8.pdf}
    \caption{Evolution of average particle diversity $D_{\alpha}$,
      bulk particle diversity $D_{\gamma}$ and bulk mixing state
      $\chi$ ($\%$) at the beginning of cloud cycle 1 and after each
      of the four cloud cycles (high-emission case).}
    \label{fig:bulk_chi}
\end{figure}

Figure~\ref{fig:chi_diagram} graphs the progression of the diversity
metrics in a $D_{\alpha}$- $D_{\gamma}$ diagram using the aerosol at
$t_{\rm u} = 12$~h as input as an example. After each cloud cycle, the
mixing state index moved closer to the diagonal line which indicates a
complete internal mixture ($\chi = 100$\%), with $\chi$ increasing from
69\% to 87\%. However, a complete internal mixture was never reached
because the interstitial aerosol still contributed diversity.

\begin{figure}
    \centering
    \includegraphics[scale=0.43]{chap3_figs/fig9.pdf}
    \caption{Average particle diversity $D_{\alpha}$ and bulk particle
      diversity $D_{\gamma}$ diagram for the aerosol at $t_{\rm u}$ =
      12~h at the beginning of cloud cycle 1 and after each of the
      four cloud cycles (high-emission case). }
    \label{fig:chi_diagram}
\end{figure}

The change of aerosol mixing state is likely related to the
  fraction of activated particles because only those undergo aqueous-phase chemistry. 
  Figure~\ref{fig:chi-cdnc} compares cloud droplet
  number fraction (number concentration of cloud droplets divided by
  total aerosol concentration at the start of the cloud parcel
  simulation) and change in mixing state for the high-emission case
  and the low-emission case for the first cloud cycle. The cloud
  droplet fraction varied between 10\% and 23\% and between 20\% and
  70\% for the high-emission and low-emission cases, respectively
  (Figure~\ref{fig:chi-cdnc}a). The change in mixing state was indeed
  larger for the low-emission case where $\chi$ reaches values of
  over 90\%  after the first cloud cycle
  (Figure~\ref{fig:chi-cdnc}b). 
\begin{figure}
    \centering
    \includegraphics[scale=0.5]{chap3_figs/fig10.pdf}
    \caption{(a) Cloud droplet fraction for the low-emission
        case and the high-emission case (b) Mixing state before
        (grey lines) and after first cloud cycle (blue lines). Circles
        are for the high-emission case and triangles are for
        low-emission case.}
    \label{fig:chi-cdnc}
\end{figure}

CCN properties are determined by particle size and composition. As
illustrated above, the activated particles grew during cloud
processing and attained high hygroscopicity because of the added
ammonium sulfate and ammonium nitrate. Particles with higher
hygroscopicity and larger sizes have smaller critical supersaturation
and activate at lower supersaturation level. Aerosol hygroscopicity has been 
observed to increase significantly because of the
increased ammonium sulfate and ammonium nitrate mass fraction after
cloud processing \citep{Henning2014,Christiansen2020, Saliba2020}.

Figure~\ref{fig:cycle_ccn}a shows the change of CCN spectrum after
each cloud cycle for the high-emission case. As before, the
solid lines indicate the average over the 25 urban plume cases and the
shaded bands show the range of one standard deviation. For
environmental supersaturations larger than 0.15\%, the CCN/CN ratio
remained unchanged after undergoing cloud processing. For
supersaturations lower than 0.15\%, the CCN/CN ratio increased with
the largest increase occurring after the first cloud cycle. This is
expected, since the interstitial aerosol remained unchanged during all
cloud cycles, and only the particles that formed cloud droplets in the
first cycle become progressively more easily activated (i.e. activate
at lower and lower supersaturations).

Figure~\ref{fig:cycle_ccn}b illustrates the changes in CCN number
concentration for the individual plume hours and an environmental
supersaturation of 0.02\%. Before cloud processing, the CCN number
concentrations for all plume hours were less than 30~$\rm cm^{-3}$ at
this supersaturation level. After cloud processing, the CCN/CN
  ratio increased to about 12\%, which translates to a substantial
  increase of absolute CCN number concentration because of the large
  number concentration present in this scenario. For example, the
  population at $t_{\rm u} = 12$~h experienced an increase of CCN
  concentration from 25 to 547~$\rm cm^{-3}$ after the first
  cycle. While the CCN concentrations for supersaturations above the
  value that was reached in the cloud parcel simulations did not
  change, it was sensitive for supersaturations lower than what
  was reached in the cloud parcel simulations.

CCN concentrations for supersaturation thresholds larger than 0.15\%
did not change as a result of aqueous-phase chemistry. This particular
threshold value was determined by the maximum supersaturation obtained
in our cloud parcel simulations, which then governs the subpopulation
of particles that activate. This threshold value is expected to
increase for larger cooling rates which would result in larger maximum
supersaturations (assuming the same aerosol population).

The results look qualitatively similar for the low-emission
  case (Figure~S4), however here the supersaturation threshold above
  which the CCN/CN did not change was somewhat larger, consistent
  with the overall higher supersaturations that were reached in the
  cloud parcel simulations (Figure~S5a).

Increases in CCN/CN ratio are generally attributed to both
  increased particle size and (potentially) increased
  hygroscopicity. To disentangle the two factors, we performed a
  sensitivity calculation where we set the hygroscopicity parameter
  $\kappa$ of sulfate, nitrate, and ammonium to 0.1 (representative of
  SOA) for the particles that had experienced cloud processing (i.e.,
  we artificially decreased the hygroscopicity of these particles). The
  modified CCN/CN spectrum is shown in Figure~S3. While the addition
  of ammonium sulfate and nitrate increased CCN/CN from 0.01 to 0.11
  for supersaturations lower than 0.04\%, for the sensitivity
  calculation, the ratio only increased to 0.09. That is, only
  increasing the size, but keeping $\kappa$ to a value representative
  of SOA rather than sulfate or nitrate will reduce the effect by
  18\%.

It is worth mentioning that our conclusion that CCN
  concentration increased after cloud processing is based on the
  changes of critical supersaturation. Previous studies showed that
  cloud processing can lead in fact to lower cloud droplet number
  concentrations in subsequent cloud cycles. While an increase in
  particle size due to aqueous-phase chemistry lowers the critical
  supersaturation of these particles, this will suppress the
  environmental supersaturation in the subsequent cloud cycle, overall
  reducing cloud droplet number concentration
  \cite{choularton1997effect, feingold2000does,
    romakkaniemi2006influence,
    ivanova2008aerosol,bower1999modelling}. In our study, the maximum
  environmental supersaturation indeed decreased in the following
  cycles because of the presence of the larger, cloud-processed
  particles (Figure~S5a), and the cloud droplet concentrations
  accordingly decrease in the following cycles (Figure~S5b). This is
  consistent with \citet{feingold2000does}, where decreased cloud
  droplet number concentration in the subsequent cloud cycles was
  found for the environments with larger updraft velocity ($>$ 0.1~$\rm m
  \, s^{-1}$) and does not contradict our current findings that
  processed particles have a higher activation potential.


\begin{figure}
    \centering
    \includegraphics[scale=0.4]{chap3_figs/fig11.pdf}
    \caption{(a) CCN spectrum and (b) CCN number
      concentration at ss = 0.02\% after each of the four cloud
      cycles (high-emission case). Solid lines in (a) are the
      mean distributions of all 25 cases at each cycle and the shaded
      band represents the 1$\sigma$ range. The black line in (a)
      indicates the supersaturation level at 0.02\%, taken here
        as an illustrative example.}
    \label{fig:cycle_ccn}
\end{figure}

\subsection{Effects of in-cloud coagulation on aerosol mixing state} \label{sec:coag}

So far we presented results that did not account for coagulation
events within the cloud. We generally include two different
coagulation mechanisms in PartMC, coagulation due to gravitational
differential settling using the coagulation efficiencies according to
\citet{Hall1980}, and Brownian coagulation
\citep{JACOBSON2005}. For the size ranges of droplets in our
  cloud parcel simulations, which is lower than 20~$\rm \mu m$ and not
  large enough to initiate effective collision-coalescence
  \citep{pruppacher2012microphysics}, coagulation due to gravitational
  differential settling was negligible, and therefore we focus the
  discussion on the impact of Brownian coagulation. This is a process
  that is often ignored in aerosol models
  \citep{Pierce2015}. \citet{Pierce2015} report that interstitial
      coagulation in clouds led to a 15\%--21\% decrease of total number
      concentration globally, associated with a global-mean aerosol
      indirect effect of including interstitial scavenging of $+0.5$
      to $+1.3$~$\rm W \, m^{-2}$.


We use the aerosol from $t_{\rm u} = 12$~h as input, since this
population had the largest total number concentration (see
Figure~\ref{fig:urban_plume}a), and so we expected the impacts of
Brownian coagulation to be maximized for this case. Because
coagulation events were simulated stochastically and introduce an
element of randomness into our simulations, we repeated the cloud
parcel simulation three times and report the average and 95\%
confidence interval.

Figure~\ref{fig:coag_num_brown}a compares the number distribution
before and after the first cloud cycle including only aqueous
chemistry and including aqueous chemistry and coagulation. For the
case without Brownian coagulation, the size distribution did not
change for diameters smaller than 0.1~$\rm \mu m$.  With Brownian
coagulation, the number concentration of particles smaller than
0.1~$\rm \mu m$ decreased slightly. The changes in number size
distributions comparing $t_{\rm c} = 0$ min and 30 min are shown
in Figure~\ref{fig:coag_num_brown}b. Brownian coagulation rates were
higher for particle pairs with large diameter difference. Here,
Brownian coagulation depleted the number concentration of particles
between $0.01-0.1$~$\mu$m (the interstitial aerosol). The effects of
aqueous-phase chemistry occurred at larger sizes and moved the
particles from $0.1-0.3$~$\mu$m to $0.3-0.5$~$\mu$m.

\begin{figure}
    \centering
    \includegraphics[scale=0.45]{chap3_figs/fig12.pdf}
    \caption{(a) Number distribution at initial condition $t_{\rm c} =
      0$ (grey), at $t_{\rm c} = 30$~min with aqueous chemistry only
      (blue) and at $t_{\rm c} = 30$~min with Brownian coagulation and
      aqueous chemistry (orange) (b) the change of number size
      distribution due to aqueous chemistry and Brownian
      coagulation. The solid orange line is the average of three
      realizations of the cloud parcel model simulation and the error
      bar is the 95$\%$ confidence interval.}
    \label{fig:coag_num_brown}
\end{figure}

Since Brownian coagulation in the cloud mainly affected interstitial
particles, the changes to the CCN spectrum were expected to be small
and should be mainly visible for higher supersaturation levels. As
shown in Figure~\ref{fig:coag_ccn_chi}, the CCN spectrum moved left
only for supersaturation level higher than 0.2$\%$.  For example, with
Brownian coagulation, the CCN/CN ratio increased by 4.1\% from 0.74 to
0.77 at ss = 0.5~$\%$. Figure~\ref{fig:coag_ccn_chi}b shows the change
of CN (total aerosol number concentrations) and CCN concentration in
the simulated 30~minutes. Since the decrease of total aerosol
concentration was larger, this led to the reported increase in CCN/CN
ratio. Including Brownian coagulation caused a decrease of
  total aerosol concentration (by 5.9\%) and of CCN concentration (by
  1.7\%) over the 30-min cloud parcel simulation. We expect this
  effect to be more pronounced for aerosol populations that contain a
  nucleation mode \citep{romakkaniemi2006influence}, which our case did
  not include.

Figure~\ref{fig:coag_ccn_chi}c shows the evolution of $D_{\alpha}$
and $D_{\gamma}$ using output from every minute during the cloud
parcel simulation. The trajectory of the case with Brownian
coagulation was almost identical with the case without Brownian
coagulation, indicating negligible effects of coagulation on mixing
state. Although the differences were small, it was noticeable that
including coagulation does not change $D_{\gamma}$, since the aerosol
bulk mass concentrations were not changed. However, it increased
$D_{\alpha}$, since the average particle diversity increases during
coagulation (archetypical Case~4 in \citet{Riemer2013a}).

\begin{figure}
    \centering
    \includegraphics[scale=0.5]{chap3_figs/fig13.pdf}
    \caption{(a) CCN spectrum and (b) total CN and CCN
      concentration at ss = 0.5$\%$ with only aqueous chemistry 
      (blue) and with both Brownian coagulation and aqueous chemistry
      (orange). The vertical line in (a) is for ss = 0.5$\%$, and
      solid and dotted lines in (b) are for total CN and CCN
      concentrations at that level respectively.  (c) Effects of
      coagulation on aerosol mixing state metrics. The points
      correspond to output from every minute during the cloud parcel
      simulation.}
    \label{fig:coag_ccn_chi}
\end{figure}

%
\section{Conclusions}
\label{sec:conclusions}
In this study, we investigated the impact of in-cloud aqueous-phase
chemistry on aerosol mixing state. Using the particle-resolved model
PartMC-MOSAIC, we generated particle populations of different aerosol
mixing states from urban plume simulations representing
polluted conditions. We used these as inputs for 30-min cloud parcel
simulations that included the reduced CAPRAM 2.4 aqueous-phase
chemistry mechanism. Each cloud parcel simulation was driven by the
same temperature profile, with decreasing temperature during the first
10 min, constant temperature during the second 10 min, and increasing
temperature during the third 10 min. While the cloud parcel
simulations were simplified in that we assumed adiabatic conditions,
our results are a first step towards investigating aerosol-cloud
interactions within a particle-resolved framework that allows for
representing aerosol mixing state without simplifying assumptions.

Coming back to the questions that we posed in the introduction, we
conclude the following. We quantified changes in mixing state as a result of in-cloud
processes using the diversity metrics $D_{\alpha}$ (average particle
diversity) and $D_{\gamma}$ (bulk aerosol diversity) and the mixing
state index $\chi$. The aqueous-phase chemistry processes had an
``equalizing effect'' on the diversity metrics, meaning that
$D_{\alpha}$ and $D_{\gamma}$ increased due to aqueous-phase chemistry
when the initial values were low and decreased when the initial values
were high. The first condition applied for plume time hours 0--6 in
our high-emission scenario, when fresh emissions dominated
which tend to be of low diversity, consistent with observational
findings \citep{Healy2014}. Adding secondary species to the activated
particles increased the diversities. The opposite was the case when
$D_{\alpha}$ and $D_{\gamma}$ values started out high, which applied
to plume time hours 7--24, when the plume was aged. Here, adding
secondary species to the activated particles decreased the
diversities.

Whether the overall population became more or less internally mixed
depended on the relative changes in $D_{\alpha}$ and $D_{\gamma}$. For
most populations in our study, aqueous-phase chemistry led to a more
internally mixed aerosol. For example, for the population of plume
hour $t_{\rm u}=12$~h, $\chi$ increased from 69\% to 77\% after the
first cloud cycle, and to 87\% after the fourth cloud cycle. However,
a completely internal mixture was not achieved under the conditions
investigated here, since only a portion of the aerosol population
activated and the remaining interstitial aerosol always contributed
diversity to the population. An exception was the case of plume hour
$t_{\rm u} = 0$, when the initial aerosol population was completely
internally mixed. In this case, aqueous-phase chemistry caused the
population to become more externally mixed.

The change in mixing state index $\chi$ due to cloud processing
  depended also on the fraction of particles that formed cloud
  droplets. In our high-emission case, this fraction remained below
  23\% and $\chi$ did not exceed 90\% even after the fourth cloud
  cycle. In contrast, for the low-emission case, the fraction of
  activated particles ranged between 20\% and 75\%, and $\chi$
  increased to over 90\%  after the first cloud cycle.

The size changes after cloud processing led to significant changes to
aerosol microphysical properties. With ammonium nitrate and ammonium
sulfate added to the activated particles, after the cloud evaporated,
the activation potential of the resuspended aerosol particles
increased remarkably for low supersaturation thresholds. For example,
the CCN concentration for particles from the high-emission case
  at $t_{\rm u}=12$~h and supersaturation level 0.02\% increased by a
factor of 20 from 30~$\rm cm^{-3}$ to 547~$\rm cm^{-3}$ after the
first cloud cycle. For subsequent cloud cycles, the increase was
smaller and by the fourth cycle, it was only 8.8\%.

The effects of coagulation due to gravitational settling were
negligible in our simulations. This can be explained by the fact that
the cloud droplets did not grow large enough for gravitational
settling to take over as main coagulation mechanisms. Brownian
coagulation occurred mainly between the interstitial particles at
around 0.1~$\rm \mu m$ and cloud droplets at 10~$\rm \mu m$. The
number concentration reduction caused by coagulation was up to
5.8~$\%$ in the cases considered while the CCN concentration was
reduced by less than 2\%. This therefore resulted in an increase of
the CCN/CN ratio for supersaturations higher than 0.2\%. The change in
aerosol mixing state caused by coagulation was negligible. It should
be noted that our simulations did not take into account the impact of
phoretic or turbulence effects on coagulation, which could modify the
efficiency of in-cloud coagulation.

%%%%%%%%%%%%%%%%%%%%%%%%%%%%%
%%%%%%%%%Chapter 4%%%%%%%%%%%
%%%%%%%%%%%%%%%%%%%%%%%%%%%%%

\chapter{Quantifying the effects of mixing state on aerosol optical properties}
\label{chap4}
\section{introduction}  %% \introduction[modified heading if necessary]
Aerosol particles scatter and absorb incoming solar radiation, thereby
impacting the global radiative balance and surface temperatures
\citep{mitchell1971effect, charlson1992climate, yu2006review,
  winker2010calipso, oikawa2013study, subba2020recent}. Black carbon
(BC), commonly emitted from combustion, has a direct radiative forcing
of $+0.9$~\unit{W\,m^{-2}}, which is next only to $\rm CO_2$
\citep{bond2013bounding, gustafsson2016convergence} in its warming
impact. At the same time, the overall global average aerosol direct
radiative forcing in the clear-sky environment is
$-1.9$~\unit{W\,m^{-2}} because of the presence of other
non-absorbing aerosol species, which exert a cooling impact
\citep{bellouin2005global}.

Radiative effects of aerosols depend on their optical properties,
which, as a whole, are determined by the individual particles that the
aerosol consists of. As observed in field campaigns, particles are
mixtures of inorganic and organic species, and exhibit significant
spacial and temporal variation in their abundance and composition
\citep{zhang2007ubiquity, bzdek2012single, LASKIN2006260}, with
considerable diversity in composition existing within individual
aerosol populations. The topic of this paper is to quantify the importance
of diversity in composition for aerosol optical properties. 

Aerosol composition impacts aerosol optical properties for several
reasons. First, aerosol species differ in their complex refractive
index. While inorganic species and many organic species have a purely
real refractive index (i.e. only scatter radiation) for wavelength of
visible sunlight, black carbon and some organic carbon species have a
non-zero imaginary part of the refractive index and hence also absorb
radiation \citep{corbin2018brown, Esteve2014, cappa2019light}.
Second, aerosol species differ in their hygroscopicity. This governs
aerosol water uptake in a humidified environment, which is important
for scattering \citep{MichelFlores2012, Zieger2013, Titos2014,
  Titos2016}.

Lastly, the arrangement of the different aerosol species within a
particle is important for determining their scattering and
absorption. For mixed particles without strongly absorbing species,
i.e. BC, a volume-mixing rule can be used to calculate the refractive
index of the entire particle. When BC is contained in the particle, a
core-shell configuration is proven to be more accurate
\citep{Bond2006}. The absorption enhancement of BC-containing
particles due to their non-absorbing coatings has been widely
investigated \citep{Moffet2009,Liu2017,wu2020light, Fierce2020}. It can be 
more complicated if considering BC shapes. By using discrete dipole approximation
model, \citet{scarnato2013effects} found, for BC-NACL mixtures, absorption coefficients 
enhancement is higher for internally mixed compact BC than lacy BC.

To understand the importance of aerosol composition in calculating
aerosol optical properties, it is useful to define the term aerosol
mixing state, that is, the distribution of aerosol species among the
particles in a population \citep{Riemer2009,Riemer2013a}. Aerosol
mixing state in the ambient atmosphere ranges between the two
idealized extremes of an external mixture on the one hand, where
different aerosol species reside in different particles, and an
internal mixture on the other hand, where all particles consist of the
same mixture of species. Aerosols close to emission sources tend to be
more (although not completely) externally mixed \citep{Bondy2018,
  Rissler2014}. After emission, aging processes, such as coagulation
between particles and condensation of gas species on the particles,
transform aerosol populations towards more internal mixtures
\citep{Healy2014, Liu2013, Zaveri2010a}.

Aerosol mixing state is challenging to represent in 3D chemical
transport models, which usually rely on simplifying assumptions for
computational efficiency. These assumptions then determine how aerosol
optical properties are calculated. Optical properties are here
understood by three widely-used parameters: absorption cross section,
scattering cross section and asymmetry parameter \citep{Majdi2020}.
For example, many 3D models use a modal approach to represent aerosol,
such as The Community Multiscale Air Quality Modeling System (CMAQ)
and Modal Aerosol Module (MAM)\citep{Binkowski2007, Appel2017,Liu2012}.
The modes are externally mixed from each other, whereas
within each mode, the aerosol is assumed to be internally mixed. For
BC-containing modes, a core-shell configuration is assumed. \citet{Fierce2016} found,
neglecting coating diversity for BC-contained particles can overestimate absorption by up to 
200\%. Another approach are sectional model representations, which track
size-resolved composition, but not particle composition diversity
within a certain size, such as TwO-Moment Aerosol Sectional (TOMAS) and
GLObal Model of Aerosol Processes (GLOMAP)\citep{kodros2018size, spracklen2005global}. 
Still, mixing state assumptions need to be invoked for each
size bin. The importance of mixing state had been quantified through optical closure studies. 
Using the measured aerosol size distribution and composition collected over East China Sea, \citet{koike2014case}
found neglecting BC mixing state can lead to uncertainty level of $-$18\% to $+$9\% in calculated absorption aerosol optical depth. 

The uncertainties in optical properties introduced by mixing state
assumptions had been also widely evaluated through model sensitivity studies.
Using the AQMEII-2 model inter-comparison framework, \citet{Curci2015} tested
the sensitivity of aerosol optical properties to several parameters,
including aerosol mixing state and size distribution, and they found
aerosol mixing state is the dominant factor introducing uncertainties,
explaining 30--35\% of the aerosol optical depth and single scattering
albedo (SSA) uncertainty. \citet{kodros2018size} also found the
simulated direct radiative forcing (DRF) can vary from $-1.65$ to
$-1.34$ W $\rm m^{-2}$ over the pan-Arctic region in spring depending
on the assumption of internal or external mixture. The variation is
similar when the assumptions are used to calculate DRF at the top of
atmosphere \citep{ma2012aerosol}. An open questions of these
sensitivity studies is that no benchmark simulation exists that
represents the real mixing state, and therefore the importance of
mixing state can only be assessed based on differences between varied
idealized assumptions. By applying a detailed particle-resolved benchmark
model, \citet{Fierce2017} found simple mixing state assumption can result in 
artificial redistribution of BC mass between particles and lead to errors in 
absorption. The redistribution were further confirmed to be main uncertainty sources 
for the discrepancies between the simulated and lab-observed particle optical values \citep{Fierce2020}.

The goal of this study is to systematically quantify the errors in
optical properties due to simplified assumptions for mixing state,
here quantified with the mixing state index $\chi$
\citet{Riemer2013a}. A similar framework had been used to quantify the
error in CCN concentration \citep{Ching2017}, showing that CCN error can range from $-40$\% to
150\% for externally mixed populations when assuming the aerosol was internally mixed, and the error
depended on supersaturation level and aerosol population mixing state. They found, for a given mixing state and supersaturation level,
the CCN error was caused by how many particles were artificially activated and unacitvated after mixing state assumptions used. For this work, 
we want to answer: Given the degree of aerosol mixing state, 
what is the error in aerosol optical values when assuming internal mixture and what are the reasons?


The paper is structured as follows: Model description, scenario design
and the definition of metrics are given in
Sect.~\ref{sec:methods}. Section \ref{sec:dry-case} shows the
relation between the errors in aerosol scattering and absorption and
mixing state for dry aerosol populations, and section
\ref{sec:wet-case} further analyzes the errors for the aerosol
populations at different levels of ambient relative humidity. The
errors in single scattering albedo and its implications for aerosol
direct radiative forcing are analyzed in
Sect. \ref{sec:ssa}. Section \ref{sec:conclusion} summarizes
the main findings.

\section{Model description, scenario libraries and metrics }
\label{sec:methods}
\subsection{Particle-resolved model PartMC-MOSAIC}
The model used to generate the particle populations for this study is
the particle-resolved model PartMC-MOSAIC, and a comprehensive
description of the model can be found in \citet{Riemer2009} for PartMC
and in \citet{Zaveri2008} for MOSAIC. PartMC is a Lagrangian box model
which tracks the evolution of particles in a fully-mixed computational
volume. The processes of emission, coagulation and dilution are
simulated stochastically \citep{Zaveri2010a, ching2012impacts,
  tian2014modeling, Fierce2016, curtis2016accelerated,
  Ching2017}. Gas-phase chemistry and gas-aerosol partitioning are
incorporated by coupling with the deterministic model
MOSAIC. Specifically, MOSAIC uses the carbon bond based mechanism
CBM-Z for gas-phase photochemical reactions \citep{zaveri1999new},
MTEM for calculating electrolyte activity coefficients in aqueous
inorganic mixtures and MESA for calculating the phase states of the
particles \citep{zaveri2005computationally}. The secondary organic aerosol (SOA) treatment follows the SORGAM scheme. 
Aerosol water uptake is calculated using
Zdanovskii‐Stokes‐Robinson (ZSR) method \citep{Zaveri2008,
  zdanovskii1948new,stokes1966interactions} based on the composition
of the inorganic portion of the particles.  By this method, organic
species are treated as hydrophobic, and do not contribute to water
uptake. The impact of this assumption on optical properties was
quantified by \citet{nandy2021water}.

\subsection{Scenario library design}
Following the strategy in \citet{zheng2021estimating} and
\citet{hughes2018machine}, we created a scenario library of a large
number of PartMC-MOSAIC simulations, for this study with a focus on
the aging of carbonaceous aerosol. To produce particle populations
with a wide range of compositions and mixing states, we varied the
model input parameters within the ranges shown in
Table~\ref{tab:input}. We used Latin hypercube sampling
\citep{mckay2000comparison} to create input parameter combinations for
a total of 100 model simulations. The simulation time for each
simulation was 24 hours beginning at 6 am local time with hourly
output. This yielded a total of 2500 particle populations. All
scenarios were run with 10\,000 computational particles.  To create
aerosol initial conditions with realistic mixing states we adopted the
approach described in \citet{zheng2021estimating}: We carried out a first set of
simulations, starting with the aerosol initial concentrations set to
zero for all simulations (the ``initial runs''). We then repeated the
same set of simulations, but replaced the aerosol initial condition
with a randomly sampled population from the initial runs (the
``restart runs''). For the analysis in this paper, we only used the
results from the restart runs.

\begin{table}
\centering
\caption{Baseline and range for the input variables}
\label{tab:input}
\begin{tabular}{ c c c  }
	\hline
	Input parameters  & Baseline & Range  \\
	\hline
	\multicolumn{3}{c}{Enviroment Variables} \\
	\hline
    Relative humidity (RH)&&[0.1, 1) or [0.4, 1) \\
    Latitude & &($70^o$S, $70^o$N) or ($90^o$S, $90^o$N)  \\
    Day of year & & [1, 365]\\
    Temperature & & Based on latitude and day of year\\
    \hline
    \multicolumn{3}{c}{Gas emission rates ($\rm mol$ $\rm m^{-2}$ $\rm s^{-1}$)} \\
    \hline
    Sulfur dioxide ($\rm SO_2$)&8.5$\times$ $10^{-9}$&[0-200\%] \\
    Nitrogen dioxide ($\rm NO_2$) &3.0$\times$ $10^{-9}$&[0-200\%]\\
    Nitrogen oxide ($\rm NO$) &5.7$\times$ $10^{-8}$& [0-200\%]\\
    Ammonia ($\rm NH_3$)&8.9$\times$ $10^{-9}$&[0-200\%] \\
    Carbon oxide (CO)&7.8$\times$ $10^{-7}$& [0-200\%]\\
    Methanol (CH3OH)&2.3$\times$ $10^{-10}$& [0-200\%]\\
    Acetaldehyde ($\rm ALD2$) &1.7$\times$ $10^{-9}$&[0-200\%] \\
    Ethanol (ANOL) &5.3$\times$ $10^{-9}$& [0-200\%]\\
    Acetone (AONE) &7.8$\times$ $10^{-10}$&[0-200\%]\\
    Dimethyl sulfide (DMS) &3.8$\times$ $10^{-11}$&[0-200\%] \\
    Ethene (ETH) &1.8$\times$ $10^{-8}$&[0-200\%] \\
    Formaldehyde (HCHO) &4.1$\times$ $10^{-9}$&[0-200\%] \\
    Isoprene (ISOP) &2.4$\times$ $10^{-10}$&[0-200\%] \\
    Internal olefin carbons (OLEI) &5.9$\times$ $10^{-9}$&[0-200\%]\\
   Terminal olefin carbons (OLET)&5.9$\times$ $10^{-9}$& [0-200\%]\\
   Paraffin carbon (PAR)) &1.7$\times$ $10^{-7}$&[0-200\%] \\
   Toluene (TOL) &6.1$\times$ $10^{-9}$&[0-200\%] \\
   Xylene (XYL) &5.6$\times$ $10^{-9}$&[0-200\%] \\
    \hline
    \multicolumn{3}{c}{Carbonaceous aerosol emission (single mode)} \\
    \hline
    Geometric mean diameter ($D_g$) &&[25, 250]~\unit{nm}\\
    Geometric standard deviation of diameter ($\sigma_g$) && [1.4, 2.5]\\
    BC/OC mass ratio && [0, 100\%] \\
    Particle emission flux && [0, 1.6$\times$ $10^{7}$] $\rm m^{-2}$ $\rm s^{-1}$ \\
	\hline
\end{tabular}
\end{table}

\subsection{Optical properties calculations}
We calculated the optical properties of the particle populations using
Mie calculations \citep{Zaveri2010a}. These properties included the
asymmetry parameter $g$, scattering cross section $\sigma_{\rm scat}$
and absorption cross section $\sigma_{\rm abs}$ for each
particle. Particles were assumed to be spherical, and when BC was
present, a core-shell configuration was assumed, with BC as the core
and non-BC species as the shell. In PartMC-MOSAIC, each chemical
species was assigned a refractive index and the values were the same 
as \citet{Zaveri2010a}, as listed
in Table~\ref{tab:refrc_inex}. The shell refractive index of the
particle was the volume average of all the shell species, including
aerosol water. The absorptivity of brown carbon has been of great
interest in recent years \citep{corbin2018brown, cappa2019light},
however, this was not considered in the current work. We used the
wavelength $\lambda$ of 550~\unit{nm} for our analysis.
\begin{table}
	\centering
	\caption{Refractive indices of aerosol species at $\lambda$ =
          550~\unit{nm}}
	\label{tab:refrc_inex}
	\begin{tabular}{ c  c  c }
		\hline 
		Compounds & Refractive index\\
       \hline
       $\rm H_2SO_4$ & 1.43 \\
       $\rm {(NH_4)}_2SO_4$ & 1.52 \\
       $\rm (NH_4HSO_4$ & 1.47 \\
       $\rm NH_4NO_3$ & 1.5 \\
       $\rm H_2O$ & 1.33 \\
       $\rm BC$ & 1.82 + 0.74$i$ \\
       $\rm SOA$ & 1.45  \\
       $\rm POA$ & 1.45 \\       
       \hline
	\end{tabular}
\end{table}
In PartMC-MOSAIC, all particles are tracked individually in a
well-mixed computational volume, and we obtain the ensemble
optical property values by summing over all particles in the
volume. The ensemble scattering, absorption and extinction
coefficients at wavelength $\lambda$ are given as 
\begin{equation}
\label{eq:opt-cal1}
\beta_{\rm scat}(\lambda) = \sum\limits_i^{N}\sigma_{{\rm scat},i}(\lambda)n_i,
\end{equation}

\begin{equation}
\label{eq:opt-cal2}
\beta_{\rm ext}(\lambda)  = \sum\limits_i^{N}\sigma_{{\rm scat},i}(\lambda)n_i, 
\end{equation}

\begin{equation}
\label{eq:opt-cal3}
\beta_{\rm abs}(\lambda)  = \beta_{\rm ext}(\lambda) - \beta_{\rm scat}(\lambda),
\end{equation}
where $i$ is the particle index, $n_i$ is the number concentration associated with particle $i$ and $N$ is the number of computational
particles in the population. We determined the optical properties of all particle populations of our
scenario libraries using these equations. 

\subsection{Quantifying the impact of mixing state through composition averaging}
To quantify the impacts of mixing state on aerosol optical properties,
we employed the strategy of ``composition-averaging'' similar to
\citet{Ching2016}. The technique is shown conceptually in
Fig.~\ref{fig1:scen}. For each population in our reference scenario
library, we averaged the dry particle compositions within prescribed
size bins. We chose eight size bins between 0.039 and 10~\unit{\mu m},
consistent with the bin structure of the sectional aerosol module
MOSAIC used in WRF-Chem \citep{fast2006evolution}.

The composition averaging procedure preserves the bulk mass
concentration of each species, the total number concentration, and the
particle diameters within each bin. It changes the composition so that
each bin becomes internally mixed, however the composition can vary
between bins. This mimics the assumption frequently made in sectional
models, namely that each size bin contains an internally mixed
aerosol. It is worth mentioning that PartMC-MOSAIC represents
particles outside the MOSAIC bin range, especially for the lower
boundary, and we used an extra bin (bin 0) to preserve the total
number and mass concentrations.

The changes of two important parameters for aerosol optical properties
due to composition averaging are illustrated in Fig.~\ref{fig1:scen},
BC mass fraction and the real part of the refractive index. In the
reference case, a wide range of BC mass fractions exists within the
same size bin. After composition averaging, all particles within a
size bin have the same BC mass fraction.  Since composition averaging
preserves the particle diameters, BC and other species are
redistributed so that all particles within a size bin have the same
mass fractions. Specifically, if a particle has lower BC mass fraction
than the average level in the same size bin, BC is added to this
particle from those with higher BC content. The coating species are
also redistributed after composition averaging and refractive index
varies. Hence, comparing optical properties before and after
composition averaging in the dry population $P_1$ isolates the impact
of mixing state on aerosol optical properties.

%\textcolor{red}{Should we say something about sensitivity to size
%  bins? Could we repeat the calculations for 1 bin (complete internal
%  mixture) and a lot of bins, and report how overall error changes
%  (not include any figures)?}

Since the aerosol water content plays an important role for aerosol
optical properties, we further calculated water uptake for the
reference populations and for the composition averaged poplations for
50\% ($P_2$) and for 90\% ($P_3$) relative humidity, respectively.  At
$\rm RH = 50$\%, depending on the exact composition, some particles
take up water, and at $\rm RH = 90$\%, most particles take up water,
except particles that only contain hydrophobic species, such pure
black carbon or primary organic carbon. Note that while the dry
aerosol mass was conserved by the composition-averaging procedure, the
water content was re-calculated after composition averaging and could
change compared to the reference population. We will discuss the
impact of composition-averaging for dry conditions in
Section~\ref{sec:dry-case} and the impact of water uptake in
Section~\ref{sec:wet-case}.

\begin{figure}
\centering
\begin{subfigure}{0.8\textwidth}
\includegraphics[scale=0.45]{chap4_figs/fig1.pdf}
\end{subfigure}
\begin{subfigure}{0.8\textwidth}
\includegraphics[scale=0.56]{chap4_figs/fig2.pdf}
\end{subfigure}
\caption{Conceptual framework of composition averaging. In the top
  panel, aerosol species and water are shown with different
  colors. Water is in light blue, BC is in black, and other colors are
  for the other species. The middle and bottom panels are the
  two-dimensional distributions of BC mass fraction and real part of
  refractive index for a particle population from reference (a,c) and
  sensitivity (b,d) scenario. The plots are colored by number
  concentration.  Red numbers and grey lines represent the size bin
  ranges, comparable to the bin structure used in WRF-Chem. Bin 0 is
  the extra bin to preserve the total number concentration. The two
  red rectangles are for the analysis in
  section~\ref{sec:dry-case}. This population is taken from scenario
  76 at 1~h, with $\chi$=36\%.}
\label{fig1:scen}
\end{figure}


\subsection{Mixing state metrics}
We quantified the optical properties error introduced by simplified
mixing state representation by using the metrics developed by Riemer
and West (2013). These metrics include the single-particle diversity
$D_i$, the average particle species diversity $D_\alpha$ and bulk
population species diversity $D_\gamma$. For a population with $N$
particles, total mass $\mu$ and $A$ species, we can calculate those
metrics from the total mass of particle $i$, $\mu_i$, total mass of
species $a$ in the population, $\mu^a$, and mass of species $a$ in
particle $i$, $\mu_i^{a}$, for $i$ = 1, ..., $N$ and $a$ = 1,...,
$A$. Mass fraction of species $a$ in particle $i$, $p_i^a$, mass
fraction of particle $i$ in the population, $p_i$ and mass fraction of
species $a$ in the population, $p^a$ are given by
\begin{equation}
\label{eq:frac}
\begin{split}
    p_i^a = \frac{\mu_i^a}{\mu_i},\;\;\;p_i = \frac{\mu_i}{\mu},\;\;\;p^a = \frac{\mu^a}{\mu}.
\end{split}
\end{equation}

Single particle diversity $D_i$ describes the effective species number
in each particle, and is defined as

\begin{equation}
D_i = \prod_{a=1}^A({p}_i^a)^{-p_i^a}.
\end{equation}
For particles containing the same number of species type, particle
diversity $D_i$ reaches its maximum when species are present in equal
amounts. Based on $D_i$, we can construct $D_\alpha$ and $D_\gamma$,
which describes the average effective species number in each particle
and bulk population respectively:
\begin{equation}
D_\alpha = \prod_{i=1}^N (D_i)^{p_i},
\end{equation}

\begin{equation}
 D_\gamma = \prod_{i=1}^A(p^a)^{-p^a}.  
\end{equation}

Finally, the mixing state metric $\chi$ is defined as the affine ratio
of $D_\alpha$ and $D_\gamma$:
\begin{equation}
\chi = \frac{D_\alpha - 1}{D_\gamma - 1}.
\end{equation}
The values of $\chi$ vary between 0\% to 100\%. When $\chi$ = 0\%, it
indicates that the population is fully externally mixed and each
particle only contains one species. The population is internally mixed
when $\chi$ = 100\%, and all particles have the same species mass
fractions.  For this work, our focus is the optical properties of the
particles. Differing from the traditional chemical species mixing
state index, we grouped the aerosol species by absorbing and
non-absorbing species and defined a new index, $\chi_{\rm opt}$. It
still ranges between 0\% to 100\% and signifies the degree to which
absorbing and non-absorbing species are mixed. Since we only consider
two (surrogate) aerosol species, the maximum value of $D_i$,
$D_\alpha$ and $D_\gamma$ is 2. For the remainder of the paper, we
will refer to this index simply as $\chi$.

There were some aerosol populations with unrealistic simulated species concentrations, 
and we applied upper thresholds which were calculated as the sum of $\rm 75^{th}$ percentile 
and 1.5 IQR (interquartil range) for each of the aerosol species. 
At the end, 1809 out of 2500 populations were used for error analysis. 
Figure~\ref{fig2:sce_overview} shows the range of bulk chemical
species concentrations, mixing state index, and optical properties
within the selected scenario library. The simulated aerosol bulk species mass
concentration in the library covered a wide range of urban conditions
(Fig.\ref{fig2:sce_overview}a), and the values were comparable to the
measurements in different locations \citep{jimenez2009evolution,lanz2010characterization}. As for the
distribution of mixing state, most populations had a mixing state
index $\chi$ larger than 40\%, with a median value of 85\%.
The fact that $\chi$ values smaller than 40\% did not occur
in our scenario library was consistent with the notion that BC rarely
existed in a completely external mixture. It is frequently co-emitted
with organic carbon, which form internal mixtures at the time of
emission. Additionally, in urban environments, BC ages quickly,
forming internal mixtures with secondary species.
Figure~\ref{fig2:sce_overview}c shows that the single scattering
albedo (SSA) was larger than 0.4 for all populations, with a median
value of 0.88. While SSA values lower than 0.5 can be considered
extremely low (4\%), most populations (72\%) are with SSA larger than
0.85, which is consistent with fine mode SSA observations from AERONET
\citep{levy2007global}.

\begin{figure}
	\centering
	\includegraphics[scale=0.5]{chap4_figs/fig3.pdf}
	\caption{Distribution of (a) bulk species concentration, (b)
          optical mixing state $\chi$ and (c) SSA in the scenario
          libraries. Error bars in (a) are for $\pm$1 IQR and numbers are the median species
          concentration.}
	\label{fig2:sce_overview}
\end{figure}

The following sections describe the error introduced by
composition-averaging assumptions and how the error depends on mixing
state. Similar to the methods used by \citet{Ching2017}, we stratified
the populations by optical mixing state index $\chi$. In order to
isolate the impacts of mixing state (in the sense of how the chemical
species except for aerosol water are distributed across the
population) from the impacts of water uptake, we first analyzed the
results for the dry population scenarios $P_1$. This quantified the
effects of chemical species redistribution caused by simplified mixing
state assumption used in sectional models. Particles were partially or
fully deliquescent in scenarios $P_2$ and $P_3$, and these populations
will be further analyzed in Section~\ref{sec:wet-case} to quantify the
water redistribution effects on aerosol optical properties resulting
from internally mixing hygroscopic and hydrophobic species.

\section{Errors in aerosol absorptivity and scattering for dry particles}
\label{sec:dry-case}
We quantified the errors in aerosol optical properties by comparing
the values of reference and composition-averaged
populations. The relative error for the aerosol
populations was calculated as
\begin{equation}
  \epsilon(v,\chi) = \frac{v'(\chi) - v(\chi)}{v(\chi)}, 
\end{equation}
where $v$ can be SSA, $\beta_{\rm abs}$ or $\beta_{\rm scat}$, and
$\chi$ is mixing state index. We analyzed the error of these three
parameters separately.

\subsection{Errors in aerosol absorptivity due to composition-averaging}
Absorption was overestimated universally after composition-averaging,
and, as expected, the error was higher for more externally-mixed
populations (low $\chi$ values), with $\epsilon(\beta_{\rm abs})$
reaching up to +70\% for $\chi$ of 30\%
(Fig.\ref{fig3:abs-error}). Each dot represents a particle population
from the scenario library. As shown in the box plot inset, the mean
overestimation was 18\% and the maximum reached over 80\%. The figure
further contains information of BC bulk mass concentration and
relative BC core size changes, which are the two main factors in
determining absorptivity \citep{Bond2006a}, as represented by circle
size and color, respectively. The relative BC core size change is
defined as
\begin{equation}
    \displaystyle \Delta{D^{\rm core}}=\frac{\sum_{i=1}^{N} n_i {D_i^{\rm core}}' - \sum_{i=1}^{N} n_i {D_i^{\rm core}}}{\sum_{i=1}^{N} n_i D_i^{\rm core}},
\end{equation}
where $i$ is the particle index, $n_i$ and $D_i^{\rm core}$ are the
associated number concentration and core diameter in the reference
scenario, and ${D_i^{\rm core}}'$ is the core diameter in the
sensitivity scenario. It is interesting to note that $\Delta
D^{\rm core}$ is always positive, that is, the average core diameter
after composition averaging is larger than the average core diameter
before composition averaging. This is a result of particle mass being
a convex function of particle diameter (assuming spherical
particles). Calculating the new core diameters after composition
averaging will therefore always lead to on-average larger core
diameters than averaging the core diameters before composition
averaging.

The decreasing error with increasing $\chi$ can be explained by the
magnitude of $\Delta{D^{\rm core}}$. Evidently, composition-averaging
caused larger changes of BC core sizes when the populations were more
externally mixed. For example, for $\chi=30$\%, the change in core
sizes was as large than +25\%, while for $\chi=95$\%, the change in
core sizes was less than 5\%.  We also noticed a wide range of errors
for populations with $\chi$ between 60 and 70\%, i.e., partially
internally-mixed populations. In fact, the highest overestimation of
82\% was reached at $\chi= 63$\%. As indicated by the circle size,
these populations contained very little BC (0.01~\unit{\mu g\,m^{-3}})
and even small changes in core sizes can lead to large relative errors
in volume absorption coefficient.
\begin{figure}
	\centering
	\includegraphics[scale=0.55]{chap4_figs/fig4.pdf}
	\caption{Relative error in absorption coefficients
          $\epsilon(\beta_{\rm abs})$ after composition
          averaging. Each dot represents an aerosol population. The
          color denotes the change of BC diameter due to
          composition-averaging, and the marker size represents BC
          bulk mass in the population. The box plot inset shows the
          distribution of the error. The red line shows the median,
          and the edges of the dashed lines are the minimum and
          maximum values. Red numbers are for the minimum, first
          quartile, mean, third quartile and maximum values.}
	\label{fig3:abs-error}
\end{figure}

For some particles, composition-averaging increases the sizes of BC
cores (while at the same time decreasing the coating thickness) and
for other particles it decreases the BC cores sizes (while increasing
the coating thickness). It is therefore not immediately clear that
composition-averaging consistently causes overestimation of aerosol
absorption coefficients. At per-particle scale, 
for particles of the same diameter, 
$\sigma_{\rm abs}$ increases with increasing BC core core, even though
the coating thickness (and hence the absorption enhancement) decreases (Fig.~\ref{fig_sup1}).

%It is important to remember that we preserve the particle size when
%applying composition-averaging. In order to obtain the same species
%mass fraction for particles in the same size bin, particles either
%gain BC and lose coating, or gain coating and lose BC. The change of
%absorption cross section is then the outcome of the balancing effects
%between changes in core size and coating thickness, the latter implies
%changes to the absorption enhancement. To explore the magnitude of
%these two factors, we show the value of absorption cross section
%$\sigma_{\rm abs}$ for one particle as a function of total diameter
%and core ratio, $D_{\rm BC}/D_{\rm dry}$, in
%Fig.~\ref{fig5:abs-exp}a. For particles of a given total diameter,
%$\sigma_{\rm abs}$ increases with increasing BC core size, even though
%the coating thickness (and hence the absorption enhancement)
%decreases.

\begin{figure}
	\centering
	\includegraphics[scale=0.6]{chap4_figs/fig5.pdf}
	\caption{Normalized absorption coefficient as a function of BC mass fraction for five
        monodisperse populations with different sizes. Coating species is ammonium bisulfate of refractive index 1.47.
        Absorption coefficient values are
        normalized by $\beta_{\rm abs}$ of the population with $f_{\rm
          BC} = 1$ (pure BC). Black line is for BC in external
        mixture. Colored lines are for BC in internal mixture of different sizes. Table on the right
        sketches three 300~\unit{nm} internal and external populations 
        with BC mass fraction of 0\%, 50\%, and 100\%. Black is for BC and yellow for coating species.}
	\label{fig5:abs-exp}
\end{figure}

%However, $\epsilon(\beta_{\rm abs})$ is determined by the entire
%population.
%The internal mixture in each size bin is reached by
%moving species from a group of particles to another group of
%particles.  As the BC mass fraction distribution plots in
%Fig.\ref{fig1:scen} shows, there are two major groups of particles in
%the population: Group 1 are particles with higher BC mass fraction,
%and group 2 are particles with lower BC mass. Particles in group 1
%experience decreased absorbing ability because they are losing BC, and
%it's opposite for particles in Group 2.
%It happened simultaneously in
%each size bin and the net effects are explored in

However, $\epsilon(\beta_{\rm abs})$ is determined by the entire
population. The internal mixture in each size bin is reached by
moving species from a group of particles to another group of
particles.  As the BC mass fraction distribution  in
Fig.\ref{fig1:scen} shows, there are two major groups of particles in
the population: Group 1 are particles with higher BC mass fraction,
and group 2 are particles with lower BC mass. Particles in group 1
experience decreased absorbing ability because they are losing BC, and
it's opposite for particles in Group 2.

To further illustrate the effects at population level, we showed the effects of
composition-averaging on the volume absorption coefficient for a
simplified case of five monodisperse populations of different sizes, starting out with
completely externally mixed populations 
consisting of BC and ammonium bisulfate (Fig.\ref{fig5:abs-exp}). Absorption coefficients are normalized by
the absorption coefficient for $f_{\rm BC} = 1$ (pure BC). The black line shows the
normalized volume absorption coefficient for populations when all
particles are externally mixed for bulk BC mass fractions $f_{\rm BC}$
varying between 0 and 100\%. For external mixtures, absorption increases linearly with increasing BC
mass fraction (black line). The linear correlation applies for all the five external mixed populations with
different diameters, so we can only see one black line in the figure. 

The colored lines represent the internally-mixed monodisperse
populations (i.e., after composition-averaging) for different
diameters. These populations all have higher absorption coefficients
compared to the corresponding externally mixed populations. The effect
is more pronounced for larger particles and intermediate BC mass
fractions because of the maximum $\Delta_{\rm core}$ is reached. As the table (Fig.\ref{fig5:abs-exp})shows,
for a 300~\unit{nm} population, the normalized absorption is 0.76 when the particles are internally mixed, higher 
than an external mixed population (0.5). Although this example is an idealized case since our our populations
lie between external and internal mixtures before
composition-averaging and are polydisperse, this finding confirms that
assuming internal mixture will always lead to absorption
overestimation.

The current analysis was based on the populations of the same sizes. Actually, even in the
same size bins, diameters can vary up to several hundred manometers. Cases with 
BC redistribution between particles of different sizes were further explored in  
Fig.~\ref{fig_sup2}. We found internal mixture assumption will still lead to positive biases. 

%The process of composition-averaging is more complicated since the
%particles within a bin are not monodisperse. BC mass is commonly
%transferred from small particles to large particles within the same
%bin, as we can see from the $\Delta{D^{\rm core}}$ being positive in
%Fig.\ref{fig3:abs-error}.
  
%The cases are preliminary explored (see more
%details in Fig.\ref{figsup:abs-exp}), and we found it also produce
%positive bias in absorbing ability when moving BC from several small
%particles with $D$ = 400~\unit{nm} to one large particles with $D$ =
%1200~\unit{nm}.
%At $D$ = 500~nm,
%$\sigma_{\rm abs}$ increases from 2.87$\times10^{-15}$ to
%4.63$\times10^{-14}$ $\rm m^{-2}$ when the core ratio increases from
%0.1 to 0.9.

%Therefore, in the same size bin, composition-averaging moves
%BC from small particles to larger particles to reach the same
%(average) BC fraction. This means that particles in the same size bin
%experience two competing effects: Decreased BC core size in smaller
%particles and increased BC core size in larger particles. The overall
%absorptivity for particles in the same size bin is then determined by
%the net effects of changes in these two group of particles, and the
%effects are explored in
%figure~\ref{fig5:abs-exp}(b). 

%Above all, for the absorption changes, whether BC redistribution
%between particles with same size or different size, absorbing ability
%of an aerosol population is more likely to be overestimated when
%moving from external to internal mixture. With composition averaging,
%the same amount of BC are distributed to more particles and increases
%absorption ability. This redistribution of BC is universal when
%applying internal mixture assumptions, and it explains why absorption
%are overwhelming overestimated in the populations after composition
%averaging. The effects of redistribution can be significant for
%populations that are more externally mixed, or contain less fraction
%of BC where small changes can lead to large biases. It should be
%careful when using internal mixture assumptions for these populations.

\subsection{Error in aerosol scattering due to composition-averaging}
Considering the volume scattering coefficient, composition-averaging
resulted in a negative relative error
(Fig.\ref{fig6:scat-err}). Similar to what we found for
$\epsilon(\beta_{\rm abs}$), the magnitudes of $\epsilon(\beta_{\rm
  scat})$ decreased with increasing $\chi$, but were overall smaller,
with the largest underestimation of $-$32\% for a population with
$\chi$ = 40\% and a median of $-$1.2\%. 

\begin{figure}
	\centering
	\includegraphics[scale=0.50]{chap4_figs/fig6.pdf}
	\caption{Same as Figure~\ref{fig3:abs-error}, but for
          $\epsilon({\beta_{\rm scat}})$. The color is for
          refractive index relative change and the marker size represents BC bulk
          mass in the population.}
	\label{fig6:scat-err}
\end{figure}

There are two factors that can affect particle scattering ability after 
composition-averaging. First, the redistribution of species after composition-averaging 
alters particle scattering volume. In previous section, we already concluded 
that increases in core size were responsible for the overestimation of absorption coefficients.
The increase of core size implied the shrink of the coating volumes. As Fig.~\ref{fig8:scat-exp}(a) shows, 
adding BC core can compress the scattering ability for particles with diameters less than 1200~\unit{nm}, which 
are the typical size ranges of our simulated urban plume particles. That can explain the higher scattering underestimation
is associated with higher mass concentration in Fig.~\ref{fig6:scat-err}. To further explore the effects of coating volume shrink, Fig.~\ref{fig8:scat-exp}(b) shows the size-resolved scattering coefficients before and after composition-averaging for aerosol populations from scenario 77 at 2h, which produced the largest scattering coefficients underestimation ($-$32\%). We found there is significant decrease of $\sigma_{\rm scat}$ in size range of 400--800~\unit{nm} 
in the sensitivity populations, and the core ratio increment in bin 4 is responsible for this decrease (Fig.~\ref{fig_sup3}). 

\begin{figure}
	\centering
	\includegraphics[scale=0.50]{chap4_figs/fig7.pdf}
	\caption{(a) Relation between scattering ability, refractive
          indices and diameter. Blue lines are for non-absorbing
          particles and symbols indicate different refractive
          index. Red line is for absorbing particles including a BC
          core of $0.2D$. (b) Size-resolved scattering coefficients at reference and sensitivity scenario library.
          This population is from scenario 77 at 2h.}
	\label{fig8:scat-exp}
\end{figure}

The change of refractive index is the other factor that affects particle scattering ability. For particles with diameter between 800 and 1200~\unit{nm}, lower refractive index leads to higher scattering ability (Fig.~\ref{fig8:scat-exp}(a)), even though the amplitude is smaller than the change brought by coating volume shrink. Similar to BC core size change $\Delta D^{\rm core}$, we defined a volume-weighted index change, $\Delta{m^{\rm real}}$, to help understand the behaviour of particle scattering ability changes. The index was defined as:
\begin{equation}
    \displaystyle \Delta{m^{\rm real}}=\frac{\sum_{i=1}^{N} V_i {m^{\rm real}}' - \sum_{i=1}^{N} V_i {m^{\rm real}}}{\sum_{i=1}^{N} V_i {m^{\rm real}}},
\end{equation}
where $i$ is the particle index, $V_i$ is the particle volume, 
${m^{\rm real}}$ is the real part of coating refractive index of the particles in reference library,
, and ${m^{\rm real}}'$ is for particles sensitivity library. As shown in Fig.~\ref{fig6:scat-err}, aerosol populations with small scattering change are well explained by the small $\Delta{m^{\rm real}}$. For more external mixed populations (with lower $\chi$), large $\Delta{m^{\rm real}}$ is associated with these populations.

So, as for the effects of composition-averaging for particle scattering ability, we can conclude that at a given value of $\chi$, the amplitude of $\epsilon(\beta_{\rm scat})$ were connected with the coating volume alteration and coating refractive index changes. Decrease of coating volume after composition-averaging is the major factor for the decrease of scattering coefficients. Populations with large underestimation are those with higher BC mass concentration and large refractive index changes.  

\section{The effects of water uptake on aerosol optical properties}
\label{sec:wet-case}
The analysis so far was based on dry aerosol populations. In this
section we investigate the impact of water uptake on the errors in
absorption and scattering by considering RH values of 50\% and 90\%.
As a reminder, we performed composition-averaging on the dry
population first, and then calculated water uptake based on the
averaged composition for 50\% RH and 90\% RH, respectively.

Considering all populations, the range of relative errors in
$\beta_{\rm scat}$ decreases with increasing RH. The median value of
$\epsilon(\beta_{\rm scat})$ decreased from 1.2\% for dry populations
RH0 to 0.2\% for 90\% RH, and the IQR decreased from 4.2\% to 1.5\%
(Fig.\ref{fig9:RH90-RH10-opt-scat}a).  In contrast, the range of
relative errors in $\beta_{\rm abs}$ remains approximately the same
(Fig.\ref{fig9:RH90-RH10-opt-scat})b. The median value of error
$\epsilon(\beta_{\rm abs})$ remained at 13\% and IQR decreased from
18.3\% to 14.9\%.
\begin{figure}
	\centering
	\includegraphics[scale=0.5]{chap4_figs/fig8.pdf}
	\caption{Box plot of (a) scattering relative error
          $\epsilon(\beta_{\rm scat})$ and (b) absorption relative
          error $\epsilon(\beta_{\rm abs})$ at three RH levels (0\%,
          50\% and 90\%). Dots are the populations with values outside Q3 + 1.5IQR. }
	\label{fig9:RH90-RH10-opt-scat}
\end{figure}

%The relationship between the errors for dry conditions and for 90\% RH
%is further explored in Fig.\ref{fig10:RH90-RH10-opt-scat}.  Scattering
%error $\epsilon(\beta_{\rm scat})$ is above the 1:1 line, indicating
%for the same population, the magnitude of the error introduced by
%composition averaging decreased when water uptake is considered. The
%larger the water uptake (indicated by color), the closer the error is
%0\%, and this applies to populations with both low and high scattering
%coefficients (indicated by marker size). The effect of water uptake on
%$\epsilon(\beta_{\rm abs})$ is small, with most populations falling
%close on the 1:1 line. Outliers are populations with very less BC and
%absorbing ability can change a lot even with a small change of the
%values. This is similar to what we found in the dry population, where
%minor distribution of chemical species will resulted in higher error
%when the particles contain less BC.

\begin{figure}
	\centering
	\includegraphics[scale=0.50]{chap4_figs/fig9.pdf}
	\caption{Box plots of (a) Scattering coefficients $\beta_{\rm scat}$, 
	(b) Absorption coefficients $\beta_{\rm abs}$ at the RH levels of 0, 50, 90\%. Dark blue is for populations
	in reference scenario and Dark orange is for sensitivity scenario.}
	\label{fig10:RH90-RH10-opt-scat}
\end{figure}

The different response of $\epsilon(\beta_{\rm \rm abs})$ and
$\epsilon(\beta_{\rm \rm scat})$ after the population become
humidified is due to changes of the coating species after water
uptake, and this is examined in Fig.~\ref{fig10:RH90-RH10-opt-scat}. 
At humidified environment, scattering coefficients increased significantly.
Compared with median $\beta_{\rm scat}$ = 4.42$\times$
$10^{-5}$~\unit{m^{-1}}at RH = 10~\%, $\beta_{\rm scat}$ increased to
5.90$\times$ $10^{-5}$ ~\unit{m^{-1}} and 14.8$\times$ $10^{-5}$
~\unit{m^{-1}} at RH = 50~\% and at RH = 90~\%, respectively.
The enhancement ratio, defined by the $\beta_{\rm scat}$ values in higher RH cases
and dry case, are 1.33 at RH = 50\% and 3.35 at RH = 90\% in our
scenario populations, which are in accordance with the previous
studies \citep{Titos2016, Burgos2020}. However, the differences between the 
reference and sensitivity are small and keep almost unchanged at higher RH 
environment, which implies the water absorption ability is not affected by species redistribution. 
The increase of scattering coefficients and insignificant change of the differences explain the decrease of $\epsilon(\beta_{\rm \rm abs})$ with 
increasing RH. As for absorption coefficients in humidified environment, the differences 
between reference and sensitivity cases stay the same and the effect of RH in absorption coefficients 
is minimal. 

\section{Errors in single scattering albedo and implications for directive radiative forcing}
\label{sec:ssa}
The changes of scattering and absorption coefficients lead to changes
in SSA. Combining the definition of SSA, we can calculate the absolute
error $\Delta$SSA as:
\begin{equation}{\label{eq8:ssa_err}}
\Delta {\rm SSA} = \frac{\beta_{\rm \rm scat}'}{\beta_{\rm \rm scat}' + \beta_{\rm \rm abs}'} - \frac{\beta_{\rm \rm scat}}{\beta_{\rm \rm scat} + \beta_{\rm \rm abs}} = \frac{\beta_{\rm \rm scat}'\beta_{\rm \rm abs} - \beta_{\rm \rm scat}\beta_{\rm \rm abs}'}{(\beta_{\rm \rm scat}' + \beta_{\rm \rm abs}')(\beta_{\rm \rm scat} + \beta_{\rm \rm abs})}
\end{equation}
where $\beta_{\rm scat}'$, $\beta_{\rm abs}'$ are for the scattering
and absorption coefficients after composition averaging. Based on the
previous analysis, we know that $\beta_{\rm scat}'$ tends to be lower
than $\beta_{\rm scat}$ and $\beta_{\rm abs}'$ greater than
$\beta_{\rm abs}$. Combining these changes with
equation~\ref{eq8:ssa_err}, these variations will result in negative
values for $\Delta$~SSA and the relative error $\epsilon(\rm SSA)$,
which is confirmed by Fig.~\ref{fig12:ssa-err}.


In order to connect $\epsilon(\rm SSA)$ with $\epsilon(\beta_{\rm
  scat})$ and $\epsilon(\beta_{\rm scat})$, we sorted the populations
by $\epsilon(\rm scat)$ and $\epsilon(\rm abs)$ ranges and calculated
the mean $\epsilon(\rm SSA)$ for each $\epsilon(\rm
scat)$-$\epsilon(\rm abs)$ bin. For all three RH levels, $\epsilon(\rm
SSA)$ was negative, meaning that composition-averaging causes an
underestimation of SSA. The largest $\epsilon(\rm SSA)$ ( $-22$\%
occured for the largest underestimation in $\epsilon(\beta_{\rm
  scat})$ in the RH = 0\% environment. Populations with $\epsilon(\rm
SSA)$ lower than $-10$\% were related to populations with large
negative magnitudes of $\epsilon(\beta_{\rm scat})$. Relative errors
in SSA decreased in more humidified environment, accompanied by
decreasing errors in scattering coefficients. The median $\epsilon(\rm
SSA)$ decreased from $-7.5$\% for RH = 0\% to $-2.2$\% in RH =
90\%. 

\begin{figure}
	\centering
	\includegraphics[scale=0.50]{chap4_figs/fig10.pdf}
	\caption{Relation between errors in SSA, scattering and
          absorption coefficients. Color represents the mean
          $\epsilon$(SSA) in the corresponding $\epsilon(\beta_{\rm
            scat})$ and $\epsilon(\beta_{\rm \rm abs})$ histogram.}
	\label{fig12:ssa-err}
\end{figure}

The underestimation of SSA can have significant impacts in calculating
direct radiative forcing.  \citet{mccomiskey2008direct} evaluated the
response of directive radiative forcing to changes of several
measurement quantities, including aerosol optical depth, single
scattering albedo and other related factors. They found the total
uncertainties in directive radiative forcing ranged from 0.2 to 3.1
\unit{W\,m^{-2}} and SSA introduced the largest uncertainties. Through
perturbation analysis, \citet{loeb2010direct} also found the SSA to be
the dominant factors for direct radiative forcing uncertainties and
with SSA perturbed $\pm$ 0.06 over ocean and $\pm$ 0.03 over land, the
resulted uncertainties in direct aerosol radiative forcing range
between $-$0.55 and 1.11 \unit{W\,m^{-2}}. Considering the SSA error of
$-$7.5\% and $-$2.2\% in dry and humidified environment in the current
simulation, combining the median SSA values (see
Fig.~\ref{fig_sup4} in the supplement), these errors
translate to $-$0.069 and $-$0.021 perturbation level in SSA,
respectively, and lead to the same order of magnitude of direct
radiative forcing uncertainties as \citeauthor{loeb2010direct} found.

\section{Conclusion and discussion}
\label{sec:conclusion}

Simplified representation of aerosol mixing state used in current
regional or global models may introduce errors in simulating aerosol
optical properties, thus leading to uncertainties in calculating
directive radiative forcing. In this study, the errors introduced by
internal mixture assumptions used in sectional aerosol models were
systematically quantified. We created a reference scenario library
with 1809 aerosol populations by performing particle-resolved aerosol model
simulations with PartMC-MOSAIC. We constructed a sensitivity library
where particles were internally mixed in a prescribed set of size bins
by applying composition-averaging. Aerosol populations from the
reference and sensitivity library were then exposed to three different
RH levels to understand the relative role of chemical species and
water redistribution introduced by the internal mixture assumption.

%Errors in dry environment
The internal mixture assumption generally lead to an overestimation of
the volume absorption coefficients and an underestimation of the
volume scattering coefficients. Populations with higher absorption
overestimation tend to be more externally mixed (lower $\chi$
values). The relative error $\epsilon(\beta_{\rm abs})$ reached +70\%
at $\chi$ = 30\% for dry populations. The magnitude was smaller for
the scattering error. The coating volume shrink due to the increasing
BC core and refractive index alteration after species redistribution 
lead to change of scattering coefficients, and the first factor
is the major reason for underestimating scattering coefficients. Most
populations show scattering underestimation after composition averaging, and
the largest underestimation ($-$32\%) is associated with the less mixed
populations with $\chi$ = 40\%.

%Errors in wet environment
In a humidified environment, the internal mixture assumption causes a
redistribution of the dry aerosol species and of aerosol water. The
bulk aerosol water content was almost identical for the aerosol
populations in reference and sensitivity libraries. The relative error
in the volume absorption coefficient $\epsilon(\beta_{\rm abs})$
displayed a similar pattern for RH of 50\% and 90\% compared to the
dry environment. The relative error in the volume scattering
coefficient $\epsilon(\beta_{\rm scat})$ decreased for higher relative
humidities because of the enhanced scattering ability through
hygroscopic growth. 

%Errors in SSA and implication for radiative forcing
The absorption overestimation and scattering underestimation resulted
in a decreasing trend in SSA. Populations with the largest
underestimation of SSA were associated with populations with the
largest underestimation in scattering. At RH = 90\%, decreasing
scattering error also leads to smaller SSA inaccuracy. Based on
previous studies in the literatrue these SSA error magnitudes
translate to uncertainty ranges between $-$0.55 and 1.11 \unit{W\,
  m^{-2}} in direct aerosol radiative forcing.
  
%limitation 1: shape
It is worth emphasizing that we used Mie theory with core-shell
configuration to calculate optical properties assuming spherical
particle shapes. Our results are therefore most representative of
BC-containing populations, where the BC core is collapsed rather than
a fractal aggregate \citep{china2013morphology, china2015morphology}. More accurate methods, such as
discrete dipole approximation (DDA) should be used to represent these
more irregular particle shapes \citep{scarnato2013effects,
  curtis2008laboratory,luo2019optical, wu2020light}.
%\edits{We also
%  acknowledge that for more bare BC particles, i.e. more
%  externally-mixed BC, the impact of mixing state on optical
%  properties (absorption and scattering) is believed to be limited}.

%limitation 2: refractive indices
The species complex index should be also explored further. For current
work, we used refractive index value 1.82 + 0.74$i$ for BC, which is
close to the medium index suggested by \citet{stier2007aerosol}, but
it can vary among different measurements.  Furthermore, we did
  not consider the absorbing abilities of organic carbons, and several
  studies had lately been conducted to constrain the imaginary index
  values of brown carbon \citep{liu2020lifecycle}. \citet{Esteve2014}
has showed the index of organic aerosols are the biggest uncertainties
in quantifying aerosol absorbing abilities. At last, for current
aerosol populations, we focused on fine mode particles and ignored sea
salt and dust particles, and it should also be important to include
these particles for reducing the uncertainties in direct radiative
forcing.

%%%%%%%%%%%%%%%%%%%%%%%%%%%%%
%%%%%%%%%Chapter 5%%%%%%%%%%%
%%%%%%%%%%%%%%%%%%%%%%%%%%%%%

\chapter{Conclusions}

\appendix
% Reset the algorithm counter
\setcounter{algorithm}{0}

\chapter{Appendix to Chapter~\ref{chap2:mon}}
\label{tab:capram}
\section{List of aqueous reactions coupled to PartMC-MOSAIC}
This appendix shows the thermodynamic and kinetic data for the aqueous chemistry reactions 
coupled with PartMC-MOSAIC. It is based on the reduced CAPRAM 2.4 mechanism, 
and the full mecahnism can refer to \citet{Ervens2003}.
Table~\ref{Henry} lists the coefficients for Henry's Law.
%%%%%%%Table A1%%%%%%%%
\begin{table}[ht]
\centering
\caption{Henry's Law coefficients} \centering
\label{Henry}
\begin{threeparttable}
\begin{tabular}{ c l c c}
\toprule Henry's Law & Equilibrium & $K_{298}$ (M $\rm atm^{-1}$)$^*$& $-\Delta H/R$ (K) \\ 
\midrule
H1  & \ce{CO_2{(\rm g)}  <=> CO_2{(\rm aq)}} & 3.1$\times 10^{-2}$& 2423 \\ 
H2 & \ce{O_3{(\rm g)} <=> O_3{(\rm aq)}} &1.14 $\times 10^{-2}$ & 2300 \\ 
H3  & \ce{HO_2{(\rm g)}  <=> HO_2{(\rm aq)}} & 9.0$\times 10^{3}$& 0 \\ 
H4  & \ce{OH{(\rm g)}  <=> OH{(\rm aq)}} & 25 & 5280 \\ 
H5  & \ce{H_2O_2{(\rm g)} <=> H_2O_2{(\rm aq)}} &1.02 $\times 10^{5}$ & 6340 \\ 
H6  &\ce{NO_2{(\rm g)} <=> NO_2{(\rm aq)}} &1.2 $\times 10^{-2}$ & 1263\\
H7  &\ce{HONO{(\rm g)} <=> HONO{(\rm aq)}} & 49 & 4880\\
H8  & \ce{HNO_3{(\rm g)} <=> NO_3^- + H^+} &4.62 $\times 10^{6}$& 10500\\
H9  &\ce{NO_3{(\rm g)} <=> NO_3{(\rm aq)}} &6 $\times 10^{-1}$ & 0\\ 
H10  &\ce{N_2O_5{(\rm g)} <=> N_2O_5{(\rm aq)}} &1.4 $\times 10^{0}$ & 0\\ 
H11 & \ce{NH_3{(\rm g)}  <=> NH_3{(\rm aq)}} & 60.7 & 3920 \\ 
H12 & \ce{HCL{(\rm g)}  <=> CL^{-} + H^{+}} & 1.89$\times 10^6$ & 8910 \\ 
H13 & \ce{HCHO{(\rm g)}  <=> HCHO{(\rm aq)}} & 2.5 & 7216 \\ 
H14 & \ce{ORA{1}{(\rm g)}  <=> ORA{1}{(\rm aq)}} & 5.53$\times 10^3$ & 5630 \\ 
H15 &\ce{SO2{(\rm g)}  <=> SO2{(\rm aq)}} & 1.24 & 3247  \\ 
H16 &\ce{OP{1}{(\rm g)}  <=> OP{1}{(\rm aq)}} & 310 & 5607  \\ 
H17 &\ce{ORA{2}{(\rm g)}  <=> ORA{2}{(\rm aq)}} & 5.5$\times 10^3$ & 5890  \\ 
H18 &\ce{MO{2}{(\rm g)}  <=> MO{2}{(\rm aq)}} & 310 & 5607  \\ 
H19 &\ce{ETHPX{(\rm g)}  <=> ETHPX{(\rm aq)}} & 340 & 87  \\ 
H20 &\ce{ETOH{(\rm g)}  <=> ETOH{(\rm aq)}} & 190 & 6290  \\ 
H21 &\ce{CH{3}OH{(\rm g)}  <=> CH{3}OH{(\rm aq)}} & 220 & 5390  \\ 
H22 &\ce{ALD{(\rm g)}  <=> ALD{(\rm aq)}} & 4.8 & 6254  \\ 
H23 &\ce{BR{2}{(\rm g)}  <=> BR{2}{(\rm aq)}} & 0.758 & 3800  \\ 
H24 &\ce{CL{2}{(\rm g)}  <=> CL{2}{(\rm aq)}} & 9.15$\times 10^{-2}$ & 2490  \\ 
H25 &\ce{SULF{(\rm g)}  <=> HSO_4^- + H^{+}} & 8.7$\times10^{14}$ & 0 \\
H26 &\ce{HNO4{(\rm g)}  <=> HNO4{(\rm aq)}} &3$\times 10^4$& 0 \\ 
H27 &\ce{ACO3{(\rm g)}  <=> ACO3{(\rm aq)}} &6.69$\times 10^2$& 5893 \\ 
H28 &\ce{GLY{(\rm g)}  <=> GLY{(\rm aq)}} &1.40& 0 \\ 
H29 &\ce{[O_2]^{**}{(\rm g)}  <=> O_2{(\rm aq)}} &1.3$\times 10^{-3}$& 1700 \\ 
\bottomrule
\end{tabular}
\end{threeparttable}
\end{table}

% Table continued on next page
\addtocounter{table}{-1}
\begin{table}[ht]
\centering
\begin{threeparttable}
\caption{Continued.}
\begin{tabular}{ c l c c}
\toprule Henry's Law & Equilibrium & $K_{298}$ (M $\rm atm^{-1}$) & $-\Delta H/R$ (K) \\ 
\midrule
H30 &\ce{CLNO2{(\rm g)}  <=> CLNO2{(\rm aq)}} &0.024& 0.0 \\ 
H31 &\ce{BRNO2{(\rm g)}  <=> BRNO2{(\rm aq)}} & 0.3 & 0.0 \\ 
H32 &\ce{BRCL{(\rm g)}  <=> BRCL{(\rm aq)}} &0.94& 0.0 \\ 
H33 &\ce{NO{(\rm g)}  <=> NO{(\rm aq)}} &1.9$\times 10^{-3}$& 0.0 \\ 
\bottomrule
%\vspace*{5mm}
\end{tabular}
\begin{tablenotes}[para,flushleft]
      \small
      \item $*$: $K = K_{298} {\rm exp}(-\frac{\Delta H}{R}(\frac{1}{T}- \frac{1}{298}))$\\
      \item $**$: Specie with square bracket indicates its concentration is constant. 
\end{tablenotes}
\end{threeparttable}
\end{table}

Table~\ref{aq-ox} lists the coefficients for aqueous chemical reactions. 
%%%%%%Table A2%%%%%%%%%%%%
\begin{table}[ht]
\centering
\caption{Aqueous chemical reactions} \centering
\label{aq-ox}
\begin{threeparttable}
\begin{tabular}{ c l c c}
\toprule Aqueous chemistry & Reaction & $ K_{298}$ (${\rm M}^{-n}$ $\rm s^{-1}$) & $-E/R$ (K) \\ 
\midrule
A1 & \ce{FEOH^{2+} -> FE^{2+} + OH(\rm aq)} & 4.76$\times 10^{-3}$ & 2.20 \\
A2 & \ce{NO3^{-} -> NO2(\rm aq) + OH(\rm aq) + OH^{-}}& 4.57$\times 10^{-7}$ & 2.59 \\
A3 & \ce{H2O2(\rm aq) -> OH(\rm aq) + OH(\rm aq)} & 7.64$\times 10^{-6}$ & 2.46 \\
A4 & \ce{FEC2(O4)_2^{-} -> FE^{2+} + C2O4^{2-} + CO2(\rm aq) + CO2^{-}} &2.47 $\times 10^{-2}$& 1.96 \\
A5 & \ce{H2O2(\rm aq) + FE^{2+} -> FE^{3+} + OH(\rm aq) + OH^{-}} & 50.0 & 0.0 \\
A6 & \ce{H2O2(\rm aq) + Cu^+ -> Cu^{2+} + OH(\rm aq) + OH^-} & 7000.0 & 0.0 \\
A7 & \ce{O2^- + FE^{3+} -> FE^{2+} + O2(\rm aq)} & 1.5$\times 10^8$ & 0.0 \\
A8 & \ce{HO2(\rm aq) + FE(OH)^{2+} -> FE^{2+} + O2(\rm aq) + [H2O](\rm aq)} &1.3$\times 10^5$& 0.0 \\
A9 & \ce{O2^{-}(\rm aq) + FE(OH)^{2+} -> FE^{2+} + O_2(\rm aq) + OH^-} & 1.5$\times 10^8$ & 0.0 \\
A10 & \ce{O2^- + FE^{2+} -> FE^{3+} + H2O2(aq) + 2OH^- - 2[H2O](aq)} & 1.0$\times 10^7$ & 0.0 \\
A11 & \ce{HO2(aq) + FE^{2+} -> FE^{3+} + H2O2(aq) + 2OH^- -2[H2O](aq)} & 1.2$\times 10^6$& --5050.0 \\
A12 & \ce{OH(aq) + FE^{2+} -> FE(OH)^{2+}} & 4.3$\times 10^8$ & --1100 \\
A13 & \ce{O2^- + Cu^+ -> Cu^{2+} + H2O2(aq) + 2OH^- -2[H2O](aq)} & 1.0$\times 10^{10}$& 0.0 \\
A14 & \ce{HO2(aq) + Cu^+ -> Cu^{2+} + H2O2(aq) + OH^- -[H2O](aq)} & 2.3$\times 10^9$& 0.0 \\
A15 & \ce{HO2(aq) + Cu^{2+} -> Cu^+ + O2(aq) + H^+} & 1.0$\times 10^8$ & 0.0 \\
A16 & \ce{O2^- + Cu^{2+} -> Cu^+ O2(aq)} & 8$\times 10^9$ & 0.0 \\
A17 & \ce{FE^{3+} + Cu^+ ->= FE^{2+} + Cu^{2+}} & 1.3$\times 10^7$ & 0.0\\
A18 & \ce{FE(OH)^{2+} + Cu^+ -> FE^{2+} + Cu^{2+} + OH^-} & 1.3$\times 10^7$ & 0.0 \\
A19 & \ce{O3(\rm aq) + O2^- -> O3^- + O2(rm aq)} & 1.5$\times 10^9$ & 0.0 \\
A20 & \ce{HO3(aq) -> OH(\rm aq) + O2(\rm aq)} & 330.0 & --4500.0 \\
A21 & \ce{H2O2{(\rm aq)} + OH{(\rm aq)} -> HO_2{(\rm aq)} + H_2O} &3.0 $\times 10^{7}$& --1680 \\  
A22 & \ce{HSO3^- + OH{(\rm aq)} -> SO3^{-} + H2O}&2.7 $\times 10^{9}$& 0 \\ 
A23 & \ce{Cu^+ + O2(\rm aq) -> Cu^{2+} + O2^-} & 4.6$\times 10^5$& 0.0 \\
A24 & \ce{FE^{2+} + O3(aq) -> FEO^{2+} + O2(aq)} & 8.2$\times 10^5$ & --4690.0 \\
A25 & \ce{FEO^{2+} + Cl^- -> FE^{3+} + CLOH^- + OH^- - [H2O](aq)} & 100.0 & 0.0 \\
\bottomrule
\end{tabular}
\end{threeparttable}
\end{table}

% Table continued on next page
\addtocounter{table}{-1}
\begin{table}[ht]
\centering
\begin{threeparttable}
\caption{Continued.}
\begin{tabular}{ c l c c}
\toprule Aqueous chemistry & Reaction & $ K_{298}$ (${\rm M}^{-n}$ $\rm s^{-1}$) & $-E/R$ (K) \\ 
\midrule
A26 & \ce{FEO^{2+} + FE^{2+} -> 2FE^{3+} + 2OH^-} & 7.2$\times 10^4$ & --842.0 \\
A27 & \ce{N2O5(aq) -> NO2^+ + NO3^-} & 1.0$\times 10^9$ & 0.0 \\
A28 & \ce{NO2^+ + [H2O](aq) -> NO3^- + H^+ + SO3^-} & 8.9$\times 10^7$ & 0.0 \\
A29 & \ce{NO3(aq) + HSO3^-  -> NO3^- + H^+ + SO3^-} &1.3 $\times 10^{9}$& --2000.0\\ 
A30 & \ce{NO3(aq) + SO4^{2-}  -> NO3^- + SO4^-} & 1.0 $\times 10^{5}$& 0.0\\
A31 & \ce{NO4^-  -> NO2^- + O2{\rm (aq)}} & 4.5$\times10^{-2}$ & 0.0 \\ 
A32 & \ce{HNO4{(aq)} + HSO3^- -> HSO4^- + H^+ + NO3^-} &3.3 $\times 10^{5}$& 0.0 \\ 
A33 & \ce{NO2^+ + Cl^- -> CLNO2(aq)} & 1.0$\times 10^{10}$& 0.0 \\
A34 & \ce{NO2^+ + Br^- -> BRNO2(aq)} & 1.0$\times 10^{10}$& 0.0 \\
A35 & \ce{CLNO2(aq) + Br^- -> NO2^- + BRCL(aq)} & 5.0$\times 10^6$ & 0.0 \\
A36 & \ce{BRNO2(aq) + Br^- -> BR2(aq) + NO2^-} & 2.55$\times 10^4$ & 0.0 \\
A37 & \ce{BRNO2(aq) + Cl^- -> NO2^- + BrCl(aq)} & 10.0 & 0.0 \\
A38 & \ce{HMS^- + OH(aq) -> CHOHSO3^- + [H2O](aq)} & 3.0$\times 10^8$ & 0.0 \\
A39 & \ce{O2CHOHSO3^- + O2(aq) -> O2CHOHSO3^-} & 2.6$\times 10^9$& 0.0 \\
A40 & \ce{O2CHOHSO3^- -> HO2(aq) + CHOSO3^-} & 1.7$\times 10^4$ & 0.0 \\
A41 & \ce{O2CHOHSO3^- -> O2CHO(aq) + HSO3^-} & 7$\times 10^3$ & 0.0 \\
A42 & \ce{CHOSO3^- + [H2O](aq) -> HSO3^- + ORA{1}(aq)} & 1.26$\times 10^{-2}$ & 0.0 \\
A43 & \ce{O2CHO(aq) + [H2O](aq) -> ORA{1}(aq) + HO2(aq)} & 44.32 & 0.0 \\
A44 & \ce{HSO3^- + H2O2{(aq)} + H^+ -> SO4^{2-} + 2H^+ + [H2O](aq)} &7.2 $\times 10^{7}$& $-$4000.0\\ 
A45 & \ce{HSO3^- + O3{(aq)} -> SO4^{2-} + H^+ + O2{(aq)}} &3.7 $\times 10^{5}$& $-$5530.0 \\ 
A46 & \ce{SO3^{2-} + O3{(aq)} -> SO4^{2-} + O2{(aq)}} &1.5 $\times 10^{9}$& $-$5280.0 \\ 
A47 & \ce{SO5^{-} + FE^{2+} -> HSO5^- + FEOH^{2+}} & 2.65 $\times 10^7$ & -- 5809.0 \\
A48 & \ce{HSO5^- + FE^{2+} -> SO4^- + FEOH^{2+}} & 3.0$\times 10^4$ & 0.0 \\
A49 & \ce{FE^{2+} + SO4^- -> FEOH^{2+} + SO4^{2-} + H^+} & 4.6$\times 10^9$ & 2165.0 \\
A50 & \ce{SO5^- + SO5^- -> SO4^- + SO4^- + O2(aq)} & 2.2 $\times 10^8$ & --2600.0 \\ 
A51 & \ce{SO5^- + HO2{(aq)} -> SO5O2H^-} & 1.7$\times 10^9$ & 0.0 \\
A52 & \ce{SO5O2^{2-} -> HSO5^- + O2(aq) + OH^- -[H2O](aq)} & 1.2$\times 10^3$ & 0.0 \\
A53 & \ce{SO3^- + O2{(\rm aq)} -> SO5^-} & 2.5$\times10^9$ & 0.0 \\ 
A54 & \ce{SO4^- + [H_2O](aq) -> SO4^{2-} + OH{(\rm aq)} + H^+} & 11.0 & -1110.0 \\
A55 & \ce{HSO5^- + HSO3^- + H^+ -> 2SO4^{2-} + 3H^+ } & 7.14$\times 10^6$& 0.0 \\
A56 & \ce{CH3OH(aq) + OH(aq) -> CH2OH(aq) + [H2O](aq)} & 1.0$\times 10^9$ & 0.0 \\
A57 & \ce{CH2OH(aq) + O2(aq) -> O2CH2OH(aq)} & 2 $\times 10^9$ & 0.0 \\
A58 & \ce{O2CH2OH(aq) + O2CH2OH(aq) -> CH2OH(aq) + O2(aq) + aHCHO} & 1.05$\times 10^9$ & 0.0 \\
A59 & \ce{ETOH(aq) + OH(aq) -> CH3CHOH(aq) + [H2O](aq)} & 1.9$\times 10^9$ & 0.0 \\
A60 & \ce{CH3CHOH(aq) + O2(aq) -> O2CH3CHOH(aq)} & 2.0$\times 10^9$ & 0.0 \\
A61 & \ce{O2CH3CHOH(aq) + ALD(aq) -> HO2(aq)} & 52.0 & --7217.0 \\
A62 & \ce{CH2OH2(aq) + OH(aq) -> CHOH2(aq) + [H2O](aq)} & 1.0 $\times 10^9$ & --1020.0 \\
A63 & \ce{CHOH2(aq) + O2(aq) -> HO2(aq) + ORAQ1(aq)} & 2.0$\times 10^9$ & 0.0 \\
A64 & \ce{CH3CHOH2(aq) + OH(aq) -> CH3COH2(aq) + [H2O](aq)} & 1.2$\times 10^9$ & 0.0 \\
A65 & \ce{ALD(aq) + OH(aq) -> CH3CO(aq) + [H2O](aq)} & 3.6$\times 10^9$ & 0.0 \\
A66 & \ce{ORA{1}(aq) + OH(aq) -> CO2H(aq) + [H2O](aq)} & 1.3$\times 10^8$ & $-1000.0$ \\
A67 & \ce{HCOO^- + OH(aq) -> CO2H(aq) + OH^-} & 3.2 $\times 10^9$ & $-1000.0$ \\
A68 & \ce{ORA2(aq) + OH(aq) -> CH2COOH(aq) + [H2O](aq)} & 1.5$\times 10^7$ & $-1330.0$ \\
A69 & \ce{MCOO^- + OH(aq) -> CH2COO^- + [H2O](aq)} & 1.5$\times 10^7$ & --1800.0 \\
A70 & \ce{CH2COOH(aq) + O2(aq) -> ACO3(aq)} & 1.7 $\times 10^9$ & 0.0 \\
A71 & \ce{MO2(aq) + MO2(aq) -> CH3OH(aq) + HCHO(aq) + O2(aq)} & 7.4$\times 10^7$ & 0.0 \\
A72 & \ce{MO2(aq) + MO2(aq) -> Ch3O(aq) + CH3O(aq) + O2(aq)} & 3.6$\times 10^7$ & --2200.0 \\
A73 & \ce{ACO3(aq) + ACO3(aq) -> 2MO2(aq) + 2CO2(aq) + O2(aq)} & 1.5$\times 10^8$ & 0.0 \\
A74 & \ce{MO2(aq) + HSO3^- -> OP1(aq) + SO3^-} & 5.0$\times 10^5$ & 0.0\\
\bottomrule
%\vspace*{5mm}
\end{tabular}
\end{threeparttable}
\end{table}

% Table continued on next page
\addtocounter{table}{-1}
\begin{table}[ht]
\centering
\begin{threeparttable}
\caption{Continued.}
\begin{tabular}{ c l c c}
\toprule Aqueous chemistry & Reaction & $ K_{298}$ (${\rm M}^{-n}$ $\rm s^{-1}$) & $-E/R$ (K) \\ 
\midrule
A75 & \ce{ETHPX(aq) + ETHPX(aq) -> CH3CH2O(aq) + CH3CH2O(aq) + O2(aq)} & 1.0$\times 10^8$ & 750.0 \\
A76 & \ce{CH3CH2O(aq) -> CH3CHOH(aq)} & 1.0$\times 10^6$ & 0.0 \\
A77 & \ce{OH(aq) + HC2O4^- -> C2O4^- + [H2O](aq)} & 3.2$\times 10^7$ & 0.0 \\
A78 & \ce{OH(aq) + C2O4^{2-} -> OH^- + C2O4^-} & 5.3$\times 10^6$ & 0.0 \\
A79 & \ce{C2O4^- + O2(aq) -> CO2(aq) + O2^- + CO2(aq)} & 2$\times 10^9$ & 0.0 \\
A80 & \ce{OH(aq) + CHOH2CHOH2(aq) -> COH2CHOH2(aq) + [H2O](aq)} &1.1$\times 10^9$  & --1516.0 \\
A81 & \ce{COH2CHOH2(aq) + O2(aq) -> aO2COH2CHOH2(aq)} & 1.38$\times 10^9$ & 0.0 \\
A82 & \ce{O2COH2CHOH2(aq) -> HO2(aq) + CHOH2COOH(aq)} & 2$\times 10^9$ & 0.0\\
A83 & \ce{HO(aq) + CHOH2COOH(aq)  ->  COH2COOH(aq) + [H2O](aq)} & 1.1$\times 10^9$& --1516.0 \\
A84 & \ce{COH2COOH(aq) + O2(aq) -> O2COH2COOH(aq)} & 2.0$\times 10^9$ & 0.0 \\
A85 & \ce{O2COH2COOH(aq)  -> HO2(aq) + H2C2O4(aq)} & 2.0$\times 10^9$ & 0.0 \\
A86 & \ce{CH3COH2(aq) + O2(aq) -> CH3COH2OO(aq)} & 2.0$\times 10^9$& 0.0 \\
A87 & \ce{CH3COH2OO(aq) -> H^+ + H^+ + MCOO^- + O2^-} & 1.0$\times 10^5$ & 0.0\\
A88 & \ce{CH3O(aq) -> CH2OH(aq)} & 1.0$\times 10^6$ & 0.0 \\
A89 & \ce{CH2COO^- + O2(aq) -> O2CH2COO^-} & 2.0$\times 10^9$ & 0.0 \\
A90 & \ce{O2CH2COO^- + O2CH2COO^- -> 2CHOH2COO^- + H2O2(aq)} & 2.0$\times 10^7$& 0.0\\
A91 & \ce{CO2^- + O2(aq) -> CO2(aq) + O2^-} & 4.0$\times 10^9$ & 0.0 \\
A92 & \ce{Cl2^- + FE^{2+} -> 2 Cl^- + FE^{3+}} & 1.0$\times 10^7$ & --3030.0 \\
A93 & \ce{Cl2^- + HO2(aq) -> 2Cl^- + H^+ + O2(aq)} & 1.3$\times 10^{10}$ & 0.0\\
A94 & \ce{Cl2^- + HSO3^- -> 2Cl^- + H^+ + SO3^-} & 1.7$\times 10^8$& --400.0 \\
A95 & \ce{Cl2(aq) + [H2O](aq) -> H^+ + Cl^- + HOCL(aq)} & 0.4 & --7900.0 \\
A96 & \ce{Cl2^- + [H2O](aq) -> H^+ + Cl^- + Cl^- + HO(aq)} & 23.4 & 0.0 \\
A97 & \ce{Br^- + SO4^- -> SO4^{2-} + Br(aq)} & 2.1$\times 10^9$ & 0.0 \\
A98 & \ce{Br^- + NO3(aq) -> NO3^- + Br(aq)} & 3.8$\times 10^9$ & 0.0 \\
A99 & \ce{Br2^- + Br2^- -> Br2(aq) + 2Br^} & 1.7$\times 10^9$ & 0.0 \\
A100 & \ce{Br2^- +  FE^{2+} -> 2Br^- + FE^{3+}} & 3.6$\times 10^6$ & --3330.0 \\
A101 & \ce{Br2^- + H2O2(aq) -> 2Br^- + H^+ + HO2(aq)} & 1.0$\times 10^5$ & 0.0 \\
A102 & \ce{Br2^- + HO2(aq) -> 2BR^- + O2(aq) + H^+} & 6.5$\times 10^9$ & 0.0 \\
A103 & \ce{Br2^- + HSO3^- -> 2BR^- + H^+ + SO3^-} & 5.0$\times 10^7$ & --780.0 \\
A104 & \ce{Br2(aq)  + [H2O](aq) -> H^+ + Br^- + HOBr(aq)} & 0.031 & --7500.0 \\
A105 & \ce{BrOH^- -> Br(aq) + OH^-} &4.2$\times 10^6$ &0.0 \\
\bottomrule
%\vspace*{5mm}
\end{tabular}
\end{threeparttable}
\end{table}

Table~\ref{aq-equi} lists constants for equilibrium equations. 

\begin{table}[ht]
\centering
\caption{Equilibrium reactions} \centering
\label{aq-equi}
\begin{threeparttable}
\begin{tabular}{ c l c c}
\toprule Aqueous equilibria & Reaction & $K_{298}$ (${\rm M}^{-n}$ $\rm s^{-1}$) & $-\Delta(H)/R$ (K) \\ 
\midrule
D1 & \ce{H_2O{(aq)} <=> OH^- + H^+} &1.8 $\times 10^{-16}$& --6800.0\\
D2 & \ce{CO_2{(aq)} <=> HCO_3^- + H^+} & 4.3 $\times 10^{-7}$& --913.0\\
D3 & \ce{NH_3{(aq)} + H_2O <=>  NH_4^+ + OH^-} &3.17 $\times 10^{-7}$& --560.0 \\
D4 & \ce{HO_2{(aq)} <=> O_2^- + H^+} & 1.6$\times 10^{-5}$& 5.0$\times 10^{10}$\\
D5 & \ce{HONO(aq) <=> NO2^- + H^+} & 5.3$\times 10^{-4}$ & --1760.0 \\
D6 & \ce{HNO_4{(\rm aq)} <=> NO_4^- + H^+} & 1$\times 10^{-5}$& 5$\times 10^{10}$ \\
D7 & \ce{NO_2{(\rm aq)} + HO_2{(\rm aq)} <=>  HNO_4{(\rm aq)}} &2.2 $\times 10^{9}$ &4.6$\times 10^{-3}$ \\
D8 & \ce{SO_2{(\rm aq)} + H_2O <=>  HSO_3^- + H^+} &3.13 $\times 10^{-4}$& 1940.0 \\
D9 & \ce{HSO_3^- <=>  SO_3^{2-} + H^+} &6.22 $\times 10^{-8}$& 1960.0\\
D10 & \ce{HSO_4^- <=> H^+ + SO_4^{2-}} &1.02$\times 10^{-2}$& 2700.0\\
D11 & \ce{ORA{1}(aq) <=> H^+ + HCOO^-} & 1.77$\times 10^{-4}$ & 12.0 \\
D12 & \ce{ORA{2}(aq) <=> H^+ + MCOO^-} & 1.75$\times 10^{-5}$ & 46.0 \\
D13 & \ce{FE^{3+} + [H2O](aq) <=> FEOH^{2+} + H^+} & 1.1$\times 10^{-4}$ & 4.3$\times 10^8$ \\
D14 & \ce{HCHO(aq) + [H2O](aq) <=> CH2OH2(aq)} & 36.0 & 4030.0 \\
D15 & \ce{ALD(aq) + [H2O](aq) = CH3CHOH2(aq)} & 2.46$\times 10^{-2}$& 2500.0 \\
D16 & \ce{HSO3^- + HCHO(aq) <=> HMS^-} & 790.0 & --3293.0 \\
D17 & \ce{HMS^- <=> HSO3^-  + HCHO(aq)} & 1.197$\times 10^{-7}$ & --5831.0 \\
D18 & \ce{SO3^{2-} + HCHO(aq) <=> HMS^- + OH^- - [H2O](aq)} & 2.5$\times 10^7$ & --2752.0 \\
D19 & \ce{HMS^ <=> HCHO(aq) + SO3^{2-} + H^+} & 3.79$\times 10^{-3}$& --5290.0 \\
D20 & \ce{Cl(aq) + Cl^- <=> Cl2^-} & 1.4$\times 10^5$ & 6$\times 10^4$ \\
D21 & \ce{Br + Br^- <=> Br2^-} & 6.32$\times 10^5$ & 1.9$\times 10^4$ \\
D22 & \ce{Cl^- + HO(aq) <=> ClOH^-} & 0.7 & 6.1$\times 10^9$ \\
D23 & \ce{ClOH^- + H^+ <=> Cl(aq) + [H2O](aq)} & 5.1$\times 10^6$& 4100.0 \\
D24 & \ce{Br^- + HO(aq) <=> BrOH^-} & 333.0 & 3.3$\times 10^7$ \\
D25 & \ce{BrOH^- + H^+ <=> Br(aq) + [H2O](aq)} &1.8$\times 10^{12}$ & 2.45$\times 10^{-2}$ \\
D26 & \ce{HO3(aq) <=> H^+ + O3^-} & 6.3$\times 10^{-9}$ & 5.2$\times 10^{10}$ \\
D27 & \ce{CHOHSO3^-  <=> CHOSO3^{2-} + H^+} & 1.34$\times 10^{-6}$ &4.4$\times 10^{10}$ \\
D28 & \ce{SO5O2H^- <=>  H^+ + SO5O2^{2-}} & 1.5$\times 10^{-5}$ & 5.0$\times 10^{10}$ \\
D29 & \ce{HC2O4m = C2O4mm + Hp} <=> & 6.25$\times 10^{-5}$ & 5.0$\times 10^{10}$ \\
D30 & \ce{H2C2O4(aq) <=> HC2O4^- + H^+} & 6.4$\times 10^{-2}$& 5.0$\times 10^{10}$ \\
D31 & \ce{CHOH2COOH(aq) <=> H^+ + CHOH2COO^-} & 3.16$\times 10^{-4}$ & 2.0$\times 10^{10}$ \\
D32 & \ce{GLY(aq) + [H2O](aq) <=>  CHOH2CHOH2(aq)} & 3900.0 & 5.5$\times 10^{-3}$ \\
D33 & \ce{FE^{3+} + C2O4^{2-} = FEC2O4^+} &2.9$\times 10^9$ & 3.0$\times 10^{-3}$ \\
D34 & \ce{FEC2O4^+ + C2O4^{2-} <=> FEC2(O4)2^-} & 6.3$\times 10^6$ & 3.0$\times 10^{-3}$ \\
D35 & \ce{SO4^- + CL^- <=> SO4^{2-} + CL(aq)} & 1.2 & 2.1$\times 10^8$ \\
D36 & \ce{NO3(aq) + CL^- <=> NO3^- + CL(aq)} & 3.4 & --4300 \\
D37 & \ce{CH3CO(aq) + [H2O](aq) <=> CH3COH2(aq)} & 367.0 & 0.0 \\
D38 & \ce{ACO3(aq) <=> H^+ + O2CH2COO^-} & 1.75$\times 10^{-5}$ & 46.0 \\
D39 & \ce{Na^+ + Na^+_C <=> Na^+ + Na^+_C} & 0.0 & 0.0\\
\bottomrule
\end{tabular}
\end{threeparttable}
\end{table}

\chapter{Appendix to Chapter~\ref{chap3}}
\label{tab:AppB}
\begin{landscape}
\begin{ThreePartTable}
  \begin{TableNotes}
    \raggedright
    \begin{hyphenrules}{nohyphenation}
    \item[]
    *: Note that Henry's Law partitioning is treated kinetically with an uptake rate constant 
     calculated as in equation 1 of Ervens et al. (2003), and then using the equilibrium constant 
     to calculate the evaporation rate constant.
    \end{hyphenrules}
  \end{TableNotes}
\begin{longtable}{ c l c c} 

	%\settablenum{S1}
	%\centering
	\caption{Thermodynamic and kinetic data for a subset of CAPRAM 2.4 mechanism} \\
	\hline
	Henry's Law & Equilibrium & $K_{298}^*$ (M $\rm atm^{-1}$) & $-\Delta H/R$ (K) \\ 
	\hline
	\endfirsthead
	
	\caption{Thermodynamic and kinetic data for a subset of CAPRAM 2.4 mechanism (Continued)} \\
	%\centering 
	\hline
 	Aqueous chemistry & Reaction & $ K_{298}$ (${\rm M}^{-n}$ $\rm s^{-1}$) & $-E/R$ (K) \\ 
 	\hline
	\endhead
	
	\hline \multicolumn{3}{r}{{Continued}} \\ 
	
	\hline \endfoot
	
	\insertTableNotes
	\endlastfoot
		
	H1  & \ce{CO_2{(\rm g)}  <=> CO_2{(\rm aq)}} & 3.1$\times 10^{-2}$& 2423 \\ 
	H2  & \ce{O_3{(\rm g)} <=> O_3{(\rm aq)}} &1.14 $\times 10^{-2}$ & 2300 \\ 
	H3  & \ce{HO_2{(\rm g)}  <=> HO_2{(\rm aq)}} & 1.14$\times 10^{-2}$& 2300 \\ 
	H4  & \ce{OH{(\rm g)}  <=> OH{(\rm aq)}} & 9$\times 10^{3}$& 0 \\ 
	H5  & \ce{H_2O_2{(\rm g)} <=> H_2O_2{(\rm aq)}} &1.02 $\times 10^{5}$ & 6340 \\ 
	H6  &\ce{NO_2{(\rm g)} <=> NO_2{(\rm aq)}} &1.2 $\times 10^{-2}$ & 1263\\ 
	H7  & \ce{HNO_3{(\rm g)} <=> NO_3^- + H^+} &4.62 $\times 10^{6}$& 10500\\
	H8  &\ce{NO_3{(\rm g)} <=> NO_3{(\rm aq)}} &6 $\times 10^{-1}$ & 0\\ 
	H9  &\ce{N_2O_5{(\rm g)} <=> N_2O_5{(\rm aq)}} &1.4 $\times 10^{0}$ & 0\\ 
	H10 & \ce{NH_3{(\rm g)}  <=> NH_3{(\rm aq)}} & 60.7 & 3920 \\ 
	H11 & \ce{HCL{(\rm g)}  <=> CL^{-} + H^{+}} & 1.89$\times 10^6$ & 8910 \\ 
	H12 &\ce{SO_2{(\rm g)}  <=> SO_2{(\rm aq)}} & 1.24 & 3247  \\ 
	H13 &\ce{SULF{(\rm g)}  <=> HSO_4^- + H^{+}} & 8.7$\times10^{14}$ & 0 \\
	H14 &\ce{HNO_4{(\rm g)}  <=> HNO_4{(\rm aq)}} &3$\times 10^4$& 0 \\ 
	H15 &\ce{O_2{(\rm g)}  <=> O_2{(\rm aq)}} &1.3$\times 10^{-3}$& 1700 \\ 
	H16 &\ce{NO{(\rm g)}  <=> NO{(\rm aq)}} &1.9$\times 10^{-3}$& 0 \\ 
	\hline
	Aqueous equilibria & Reaction & $K$ (${\rm M}$) & $-\Delta H/R$ (K) \\ 
	\hline
	D1 & \ce{H_2O{(\rm aq)} <=> OH^- + H^+} &1.8 $\times 10^{-16}$& --6800\\
	D2 & \ce{CO_2{(\rm aq)} <=> HCO_3^- + H^+} & 1.72 $\times 10^{6}$& --913\\
	D3 & \ce{NH_3{(\rm aq)} + H_2O(aq) <=>  NH_4^+ + OH^-} &3.17 $\times 10^{-7}$& --560 \\
	D4 & \ce{HO_2{(\rm aq)} <=> O_2^- + H^+} & 3.17$\times 10^{-7}$& 5.0$\times 10^{10}$\\
	D5 & \ce{HNO_4{(\rm aq)} <=> NO_4^- + H^+} & 1$\times 10^5$& 5.0$\times 10^{10}$ \\
	D6 & \ce{NO_2{(\rm aq)} + HO_2{(aq)} <=>  HNO_4{(\rm aq)}} &2.2 $\times 10^{9}$ &4.6$\times 10^{-3}$ \\
	D7 & \ce{SO_2{(\rm aq)} + H_2O(aq) <=>  HSO_3^- + H^+} &3.13 $\times 10^{-4}$& 1940 \\
	D8 & \ce{HSO_3^- <=>  SO_3^{2-} + H^+} &6.22 $\times 10^{-8}$& 1960\\
	D9 & \ce{HSO_4^- <=> H^+ + SO_4^{2-}} &1.02$\times 10^{-2}$& 2700\\
	\hline
	Aqueous chemistry & Reaction & $ K_{298}$ (${\rm M}^{-n}$ $\rm s^{-1}$) & $-E/R$ (K) \\ 
	\hline
	A1  & \ce{H_2O_2{(\rm aq)} + OH{(\rm aq)} -> HO_2{(\rm aq)} + H_2O} &3.0 $\times 10^{7}$& $-$1680 \\  
	A2 & \ce{HSO_3^- + OH{(\rm aq)} -> SO_3^{-} + H_2O}&2.7 $\times 10^{9}$& 0 \\ 
	A3 & \ce{N_2O_5(\rm aq)  -> NO_3^- + NO_2^+} & 1.0 $\times 10^{9}$& 0 \\
	A4 & \ce{NO_2^+ + H_2O(aq) -> NO_3^- + H^+ + H^+} & 8.9 $\times 10^{7}$& 0 \\ 
	A5 & \ce{NO_3(\rm aq) + HSO_3^-  -> NO_3^- + H^+ + SO_3^-} &1.3 $\times 10^{9}$& --2000\\ 
	A6 & \ce{NO_3(\rm aq) + SO_4^{2-}  -> NO_3^- + SO_4^-} & 1.0 $\times 10^{5}$& 0\\
	A7 & \ce{NO_4^-  -> NO_2^- + O_2{\rm (aq)}} & 4.5$\times10^{-2}$ & 0 \\ 
	A8 & \ce{HNO_4{(\rm aq)} + HSO_3^- -> HSO_4^- + H^+ + NO_3^-} &3.3 $\times 10^{5}$& 0 \\ 
	A9 & \ce{HSO_3^- + H_2O_2{(\rm aq)} + H^+ -> SO_4^{2-} + 2H^+ + H_2O} &7.2 $\times 10^{7}$& $-$4000\\ 
	A10 & \ce{HSO_3^- + O_3{(\rm aq)} -> SO_4^{2-} + H^+ + O_2{(\rm aq)}} &3.7 $\times 10^{5}$& $-$5530 \\ 
	A11 & \ce{SO_3^{2-} + O_3{(\rm aq)} -> SO_4^{2-} + O_2{(\rm aq)}} &1.5 $\times 10^{9}$& $-$5280 \\ 
	A12 &\ce{SO_5^- + SO_5^- -> SO_4^- + SO_4^- +  O_2{(\rm aq)}} & 2.2$\times 10^8$ & $-$2600 \\
	A13 & \ce{SO_5^- + HO_2{(\rm aq)} -> SO_5O_2H^-} & 1.7$\times 10^9$ & 0 \\
	A14 & \ce{SO_3^- + O_2{(\rm aq)} -> SO_5^-} & 2.5$\times10^9$ & 0 \\ 
	A15 & \ce{SO_4^- + H_2O(aq) -> SO_4^{2-} + OH{(\rm aq)} + H^+} & 11 & -1110 \\
	A16 & \ce{HSO_5^- + HSO_3^- + H^+ -> 2SO_4^{2-} + 3H^+ } &7.14$\times 10^6$& 0 \\
	\hline
%\end{tabular}
    \label{tab:aqchem}
\end{longtable}
\end{ThreePartTable}
\end{landscape}
%\end{center}

%\section{\edits{Change of maximum supersaturation and cloud droplets}}
\begin{figure}
    \centering
    \includegraphics[scale=0.5]{chap3_figs/fig_sup2.pdf}
    \caption{Temporal variation of (a) total number concentration, (b)
      mass concentrations of selected aerosol species, (c) mixing
      ratios of selected gas-phase species and (d) aerosol mixing
      state metrics for the low-emission case.}
    \label{fig:sup2}
\end{figure}

\begin{figure}
    \centering
    \includegraphics[scale=0.5]{chap3_figs/fig_sup3.pdf}
    \caption{(a) Number and (b) mass size distribution with respect to
      dry diameter before (grey) and after (blue) the first cloud
      cycle for the low-emission case.  The solid lines are the
      averaged distributions of all 25 cases for each cloud cycle. The
      shaded area represents the 1$\sigma$ region of the average.  are
      the distributions at the end of each cycle. The composition of
      the particles was evaluated at RH = 99\%.}
    \label{fig:mass-num-dist}
\end{figure}


\begin{figure}
    \centering
    \includegraphics[scale=0.5]{chap3_figs/fig_sup7.pdf}
    \caption{Normalized CCN spectrum before (grey) and after (blue)
      the first cloud cycle (high-emission case). The orange line is
      the CCN spectrum when assuming $\kappa$ = 0.1 for sulfate,
      nitrate and ammonium for the particles that underwent cloud
      processing. Solid lines are the mean distributions of all 25
      cloud parcel cases and the shaded band represents the 1$\sigma$
      range.}
    \label{fig:sup-new-kappa}
\end{figure}

\begin{figure}
    \centering
    \includegraphics[scale=0.5]{chap3_figs/fig_sup5.pdf}
    \caption{Normalized CCN spectrum before (grey) and after (blue)
      the first cloud cycle for the low-emission case. Solid lines are
      the mean distributions of all 25 cases at each cycle and the
      shaded band represents the 1$\sigma$ range.}
    \label{fig:sup-ccn-low-emi}
\end{figure}


\begin{figure}
    \centering
    \includegraphics[scale=0.5]{chap3_figs/fig_sup8.pdf}
    \caption{(a) Maximum supersaturation reached in the cloud parcel
      and (b) cloud droplet number concentration in the four cycles.}
    \label{fig:max_ss-cycle}
\end{figure}

\chapter{Appendix to Chapter~\ref{chap4}}
\label{tab:AppC}
\section{Relative importance of core and coating for particles with same diameter }

\begin{figure}[H]
	\centering
	\includegraphics[scale=0.6]{chap4_figs/fig_sup1.pdf}
	\caption{Particle absorption cross section $\sigma_{\rm abs}$ as a function of dry diameter and core ratio}
	\label{fig_sup1}
\end{figure}

\section{Absorption changes when BC moves between particles of different sizes}

Figure~\ref{fig_sup2} shows the relative error when moving BC
from smaller particles to larger particles. Before composition
averaging, small particles $P_1$ have diameter $D_1$, BC fraction
$f_1$ and absorption cross section $\sigma_1$. The large particles
$P_2$ have diameter $D_2$, BC fraction $f_2$ and absorption cross
section $\sigma_2$. After composition-averaging, all particles have BC
mass fraction $f$, with $f_1$ > $f$ and $f_2$ < $f$. Since the overall
BC mass is conserved, we need to move BC mass from $N$ particles of
type $P1$ and $N$ is given by:
\begin{equation}
N = \frac{D_2^3(f-f_2)}{D_1^3(f_1-f)}
\end{equation}
%\textcolor{red} {This is a bit confusing, because in reality, N is
%given by whatever there is in the bin? And then f is the parameter
%that is calculated as a free parameter?}

The resulting relative change in absorption coefficient for a
population with $N$ particles $P_1$ and one particle $P_2$ is defined
as:
  \begin{equation}{\label{eq7:abs_err}}
   \epsilon(\beta_{\rm abs}) = \frac{(\sigma_2' + N\sigma_1') - (\sigma_2 + N\sigma_1)}{(\sigma_2 + N\sigma_1)}
  \end{equation}
  where $\sigma_2'$, $\sigma_1'$ are the absorption cross section of
  $P_1$ and $P_2$ after composition averaging, respectively.  As we
  can see from Equation~\ref{eq7:abs_err}, absorption relative error
  $\beta_{\rm abs}$ is determined by the parameters of the two
  particles, including $f_1, f_2, f, D_1, D_2,\sigma_1, \sigma_2$. The
  relations between $\Delta \beta$ and these parameters should be
  nonlinear and it is hard to get a unified trend for different size
  bins.
  
  For an initial trial, figure~\ref{fig_sup2} shows the relative
  error $\beta_{\rm abs}$ when moving BC from particles $P_1$ with
  $D_1 = 600$~\unit{nm} and $f_1$ ranging from 0.4--0.9 to a large
  particle $P_2$ with $D_2 = 1200$~\unit{nm} and $f_2$ ranging from
  0.01--0.15, to make them with unified $f$ at 0.2.  These two
  diameters are the lower and upper boundaries of Bin 5. For the
  ranges considered here, the error is mainly positive, especially for
  $P_2$ with large $f_2$ and $P_1$ with small $f_1$, and $\Delta
  \beta$ reaches 59\% when $f_1$ = 0.4 and $f_2$ = 0.15. These are the
  populations where the increasing absorption ability from the large
  particles overweighting the weakening absorption of those smaller
  particles after redistribution of the species. There are also some
  populations with small overestimation, or even underestimation when
  $P_2$ is with lower $f_2$ and $P_1$ with higher $f_1$. When $f_1$ is
  0.9 and $f_2$ is 0.01, $\Delta \beta$ is $-1.3$\%. These results
  confirm the nonlinear variations when redistributing the species in
  an aerosol population. It is worth to mention that the case in the
  figure is with the simple assumption that particles in the
  population are only with two diameters, but in the simulation cases,
  aerosol populations are more complex with different sizes and BC
  mass fractions in the same size bin.
  
\begin{figure}
	\centering
	\includegraphics[scale=0.6]{chap4_figs/fig_sup2.pdf}
	\caption{Explanation for particle optical property changes
          after redistribution of BC mass among a
          population with $P1$ and $P2$. $P1$ and $P2$ are two
          particles in the same size bin, but with different diameter
          $D_1$ and $D_2$. $f_1$ and $f_2$ are for the BC mass
          fraction before composition averaging, and $f$ is the
          unified BC mass fraction after internal mixture assumption
          is used. $\sigma_1$ and $\sigma_2$ are for the absorption
          cross section of the two particles.  $D_1$ = 600~\unit{nm}, $D_2$ =
          1200~\unit{nm} and $f$= 0.2 are the specified values for this
          figure. }
	\label{fig_sup2}
\end{figure}
  
\section{Two-dimensional distributions of BC mass fraction}    %% Appendix B
\begin{figure}[H]
	\centering
	\includegraphics[scale=0.50]{chap4_figs/fig_sup3.pdf}
	\caption{Two-dimensional distributions of BC mass fraction in (a) Reference scenario and (b) Sensitivity scenario at RH0. 
	 This population is from scenario 77 at 2h.}
	\label{fig_sup3}
\end{figure}

\section{Box plots of SSA in reference and sensitivity scenario}    %% Appendix B
\begin{figure}[H]
	\centering
	\includegraphics[scale=0.50]{chap4_figs/fig_sup4.pdf}
	\caption{Box plots of single scattering albedo at different RH levels. Dark blue is for populations
	in reference scenario and Dark orange is for sensitivity scenario.}
	\label{fig_sup4}
\end{figure}

\backmatter
\renewcommand{\bibname}{References}
\bibliographystyle{copernicus}
\bibliography{thesis_ref}
\end{document}
%\chapter{Appendix to Chapter~\ref{chap2:mon}}



